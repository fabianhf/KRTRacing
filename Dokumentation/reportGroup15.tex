\documentclass[pdf]{ifacconf}

\usepackage{amsmath}
\usepackage{natbib}            % you should have natbib.sty
\usepackage{graphicx}          % Include this line if your 
                               % document contains figures,
%\usepackage[dvips]{epsfig}    % or this line, depending on which
                               % you prefer.
                               
                               \usepackage{units}

% for German
\usepackage{ngerman}           % neue Deutsche Rechtschreibung, Silbentrennung
\usepackage[utf8]{inputenc}  % Eingabe von Umlaute im Editor
%\usepackage[T1]{fontenc}       % Trennung mit Umlauten

% to include tex code of figure created with fig2texPS - and not precompiled pdf-file, see also file ``plotFigureTest.m''
\usepackage{pstricks, pst-node, pst-plot, pst-circ}
\usepackage{moredefs}
\usepackage{pgfplots}

% to include the legend into the caption. The commands are
%\mlLineLegend{red}
%\mlLineLegendDashed{red}
%\mlLineLegendDoted{red}
%\mlLineLegendDashDoted{red}
%\mlPointLegend{red}
\newlength{\mlLegendThickness}
\setlength{\mlLegendThickness}{0.4mm}
\newlength{\mlLegendHeight}
\setlength{\mlLegendHeight}{0.6ex}
\newcommand{\mlLineLegend}[1]{\mbox{\color{#1}
\protect\rule[\mlLegendHeight]{3mm}{\mlLegendThickness}\hspace*{-1mm}
}}
\newcommand{\mlLineLegendDashed}[1]{\mbox{\color{#1}
\protect\rule[\mlLegendHeight]{1.5mm}{\mlLegendThickness}\hspace*{0mm}
\protect\rule[\mlLegendHeight]{1.5mm}{\mlLegendThickness}\hspace*{-1mm}
}}
\newcommand{\mlLineLegendDoted}[1]{\mbox{\color{#1}
\protect\rule[\mlLegendHeight]{0.4mm}{\mlLegendThickness}\hspace*{0mm}
\protect\rule[\mlLegendHeight]{0.4mm}{\mlLegendThickness}\hspace*{0mm}
\protect\rule[\mlLegendHeight]{0.4mm}{\mlLegendThickness}\hspace*{0mm}
\protect\rule[\mlLegendHeight]{0.4mm}{\mlLegendThickness}\hspace*{-1mm}
}}
\newcommand{\mlLineLegendDashDoted}[1]{\mbox{\color{#1}
\protect\rule[\mlLegendHeight]{1.5mm}{\mlLegendThickness}\hspace*{0mm}
\protect\rule[\mlLegendHeight]{0.4mm}{\mlLegendThickness}\hspace*{0mm}
\protect\rule[\mlLegendHeight]{1.5mm}{\mlLegendThickness}\hspace*{0mm}
\protect\rule[\mlLegendHeight]{0.4mm}{\mlLegendThickness}\hspace*{-1mm}
}}
\newcommand{\mlPointLegend}[1]{\mbox{\color{#1}
\protect\rule[\mlLegendHeight]{0.4mm}{\mlLegendThickness}\hspace*{-0.75mm}
}}

\begin{document}

\begin{frontmatter}

\title{Bericht für den Kurs \glqq Projektwettbewerb Konzepte der Regelungstechnik\grqq}

%\thanks[footnoteinfo]{Institute for Systems Theory and Automatic Control, University of Stuttgart, Germany. \textit{http://www.ist.uni-stuttgart.de}}

% include all authors, underline corresponding author
\author{\underline{Steffen Zeile},} 
\author{Fabian Haas-Fickinger (Gruppe 15)} 
% \author{}

\begin{abstract}                          % Abstract of not more than 250 words.
Im Rahmen des \glqq Projektwettbewerbs Konzepte der Regelungstechnik\grqq  wurde ein modifizierter LQR Regler entwickelt, um einer zuvor berechneten optimalen Trajektorie folgen.\end{abstract}

\end{frontmatter}

\section{Einführung}
Die Aufgabe des Projektwettbewerbs Konzepte der Regelungstechnik ist es einen Regler zu entwerfen, der ein simuliertes Fahrzeug auf einem Rundkurs derart steuert, dass ein Runde absolviert ohne dabei die Strecke zu verlassen. Ziel des Wettbewerbs ist es eine möglichst geringe Rundenzeit zu erzielen.\\
Hierbei werden Gaspedalstellung, Lenkwinkel, Bremskraft und Bremskraftverteilung abhängig von den Fahrzeugzuständen geregelt. Die Zustände des dahinterliegenden Fahrzeugmodels, sowie der Verlauf der Streckenränder stehen dem Regler vollständig zur Verfügung.

\section{Konzept}
Unser Ansatz besteht darin, vor dem Start der Runde eine zeitoptimale Trajektorie mit Hilfe eines nichtlinearen Programms zu berechnen und diese als Referenz für einen Regler zu nutzen. Daraus resultieren zwei Teilaufgaben, das Optimierungsproblem zu lösen und einen Zustandsregler zu entwerfen.
\subsection{Optimierungsproblem}
Zur Lösung des Optimierungsproblems werden Zustände und Steuerungsgrößen mit Hilfe eines Euler Schemas bzw. Kollokation diskretisiert. Das resultierende nichtlineare Programm wird mit einem gradientenbasierten Solver gelöst. Um das gegebene Fahrzeugmodell für die gradientenbasierte Optimierung nutzbar zu machen, sind einige Anpassungen notwendig. Dabei handelt es sich hauptsächlich um die Einführung eines streckenbezogenen Koordinatensystems sowie der Eliminierung nicht differenzierbarer Funktionen.\\
Das Streckenkoordinatensystem eingeführt basiert auf dem Verlauf der Streckenmittellinie. Die Fahrzeugposition ist somit gegeben als orthogonaler Abstand zur Streckenmitte $n$, Winkel zwischen Streckentangente und Fahrzeuglängsachse $\xi$ und Distanz zum Start entlang der Streckenmitte $s$.\\ Zudem wurde der Streckenfortschritt $s$ statt der Zeit $t$ als unabhängige Variable des Optimierungsproblems genutzt um die Dimension zu reduzieren. Für diesen Zweck musste die Fahrzeugdynamik als Änderung des Zustands über die Strecke formuliert werden.\\
Die diskrete Gangwahl wurde durch ein Drehmoment-/Geschwindigkeitskennfeld ersetzt. Zielgröße des Optimierungsproblems ist die Minimierung der Fahrzeit.  \\
Da sich verschiedene, teilweise grundlegend unterschiedliche lokale Minima ergeben, wurden zusätzlich Regularisierungsterme in Form von $\delta^2$, $\ddot{\psi}$ sowie $\dot{\beta}$ zur Zielgröße hinzugefügt. Dies führt neben einer Reduzierung von Lenkwinkelschwingungen zu einer glatten Dynamik, die das Auftreten instabiler Fahrzustände auf ein Maß begrenzt, das im anschließenden Reglerentwurf tolerabel ist. \\
\subsection{Zustandsregler}
Die Aufgabe des Zustandsreglers ist die Folgeregelung des Fahrzeuges entlang der vorberechneten, optimalen Trajektorie. Ziel ist es, einen Regler zu entwickeln, der auch für instabile Fahrzustände eine Bahnfolge innerhalb der Streckenbegrenzung garantieren kann. Der Reglerentwurf findet dabei für Längs und Querdynamik getrennt statt. Das verwendete Reifenmodell lässt diese Entkopplung zu. Die Zuordnung der Solltrajektorie zur aktuellen Fahrzeugposition erfolgt ebenfalls über die zurückgelegte Strecke $s$ entlang der Fahrbahnmittellinie. $s$ sowie $n$ und $xi$, die für die Querregelung benötigt werden, berechnet der Regler mit Hilfe des Systems aus \ref{eq:system}. $C$ bezeichnet die Krümmung an der jeweiligen Position.
\begin{align}
	\dot{s} &= \frac{v cos(\beta) cos(\xi) - v sin(\beta) sin(\xi)}{1- n C} \\
	\dot{n} &= v cos(\beta) sin(\xi) + v sin(\beta) cos(\xi) \\
	\dot{\xi} &= \dot{\psi} - C \dot{s}
	\label{eq:system}
\end{align}
\begin{enumerate}
	\item Längsregelung
	Der Entwurf der Längsregelung basiert auf der Idee der Vorausschau. Mit Hilfe der Vorausschaudistanz $ds$ wird eine Sollbeschleunigung berechnet, die notwendig ist, damit das Fahrzeug im Abstand $ds$ von der aktuellen Position die Sollgeschwindigkeit erreicht. Da die Sollbeschleunigung $a_{x,Soll}$ mit einem Differenzenquotienten basierend auf einer Vorausschaudistanz ermittelt wird, ist eine Transformation in die Zeitdomäne mit Hilfe von $\dot{s}$ notwendig. $a_{x,Soll}$ ergibt sich zu \ref{eq:axsoll}.
	\begin{equation}
		a_{x,Soll} = \dot{s} \frac{v_{Soll}-v}{ds}
		\label{eq:axsoll}
	\end{equation}
	Eine zusätzliche Vorsteuerung ist nicht notwendig. Die Sollbeschleunigung wird dann mit Hilfe der Modellgleichungen des Antriebsstranges auf ein Solldrehmoment am Rad umgerechnet. Aus dem Drehmoment-/Geschwindigkeitskennfeld, das bereits für die Optimierung erzeugt wurde, folgt die Gangwahl. Das aus der Optimierung resultierende Sollgeschwindigkeitsprofil entlang der Fahrbahnmittellinie zeigt \ref{fig1}.
	\begin{figure}[h]
		\begin{center}
			\scalebox{0.5}{% This file was created by matlab2tikz.
%
%The latest updates can be retrieved from
%  http://www.mathworks.com/matlabcentral/fileexchange/22022-matlab2tikz-matlab2tikz
%where you can also make suggestions and rate matlab2tikz.
%
\definecolor{mycolor1}{rgb}{0.00000,0.44700,0.74100}%
%
\begin{tikzpicture}

\begin{axis}[%
width=4.521in,
height=3.566in,
at={(0.758in,0.481in)},
scale only axis,
xmin=0,
xmax=1200,
xlabel style={font=\color{white!15!black}},
xlabel={s[m]},
ymin=5,
ymax=45,
ylabel style={font=\color{white!15!black}},
ylabel={v[m/s]},
axis background/.style={fill=white},
xmajorgrids,
ymajorgrids,
legend style={legend cell align=left, align=left, draw=white!15!black}
]
\addplot [color=mycolor1]
  table[row sep=crcr]{%
0	5\\
0.529729130384147	7.0347912714705\\
1.05945826076829	8.21100107746418\\
1.58918739115244	9.1161009813524\\
2.11891652153659	9.91461436125814\\
2.64864565192073	10.6296759489037\\
3.17837478230488	11.2766305077499\\
3.70810391268903	11.8659326680871\\
4.23783304307317	12.4053596552577\\
4.76756217345732	12.9014153498549\\
5.29729130384147	13.3589290090408\\
5.82702043422561	13.7815837034325\\
6.35674956460976	14.1729820896606\\
6.88647869499391	14.5359760966612\\
7.41620782537805	14.8773765955585\\
7.9459369557622	15.2073560143293\\
8.47566608614635	15.5270527933101\\
9.00539521653049	15.8370287210818\\
9.53512434691464	16.1377553258725\\
10.0648534772988	16.429828812675\\
10.5945826076829	16.7136834742615\\
11.1243117380671	16.9896384283395\\
11.6540408684512	17.2581034722438\\
12.1837699988354	17.5194262103927\\
12.7134991292195	17.7739068086394\\
13.2432282596037	18.0218201472246\\
13.7729573899878	18.2634573425157\\
14.302686520372	18.4990480133657\\
14.8324156507561	18.7287643064967\\
15.3621447811403	18.9528930297139\\
15.8918739115244	19.1716376873722\\
16.4216030419085	19.3851376049386\\
16.9513321722927	19.5935883386662\\
17.4810613026768	19.7971635224111\\
18.010790433061	19.99599552758\\
18.5405195634451	20.1902291039355\\
19.0702486938293	20.3800445391818\\
19.5999778242134	20.5655688730703\\
20.1297069545976	20.7468801189999\\
20.6594360849817	20.9241383540406\\
21.1891652153659	21.0975059952463\\
21.71889434575	21.2680454873739\\
22.2486234761342	21.4364395753804\\
22.7783526065183	21.6027527274359\\
23.3080817369025	21.7670250917263\\
23.8378108672866	21.9292905158548\\
24.3675399976708	22.0896055737901\\
24.8972691280549	22.2480097049437\\
25.426998258439	22.4045231784464\\
25.9567273888232	22.5591992988296\\
26.4864565192073	22.7120752560683\\
27.0161856495915	22.8631747863789\\
27.5459147799756	23.0125332661165\\
28.0756439103598	23.1601918043198\\
28.6053730407439	23.3061853777533\\
29.1351021711281	23.4505357647506\\
29.6648313015122	23.5932857697508\\
30.1945604318964	23.7344626165758\\
30.7242895622805	23.8740795152886\\
31.2540186926647	24.0121731845627\\
31.7837478230488	24.1487712793716\\
32.3134769534329	24.283890997669\\
32.8432060838171	24.4175549185525\\
33.3729352142012	24.5497941723162\\
33.9026643445854	24.6806295934471\\
34.4323934749695	24.810071164403\\
34.9621226053537	24.9381543808888\\
35.4918517357378	25.0649060081389\\
36.021580866122	25.1903370932706\\
36.5513099965061	25.3144733301731\\
37.0810391268903	25.4373368586124\\
37.6107682572744	25.5589408813948\\
38.1404973876586	25.6792998786242\\
38.6702265180427	25.7984393414778\\
39.1999556484269	25.9163758950189\\
39.729684778811	26.0331135906082\\
40.2594139091952	26.1486810901075\\
40.7891430395793	26.2630940141679\\
41.3188721699634	26.3763598810208\\
41.8486013003476	26.4884972845934\\
42.3783304307317	26.5995251006231\\
42.9080595611159	26.7094569508625\\
43.4377886915	26.8183053248992\\
43.9675178218842	26.9260928349026\\
44.4972469522683	27.0328325935996\\
45.0269760826525	27.1385274738238\\
45.5567052130366	27.2431993677775\\
46.0864343434208	27.3469141714696\\
46.6161634738049	27.4499340387932\\
47.1458926041891	27.5523079437205\\
47.6756217345732	27.6540461927819\\
48.2053508649573	27.7551540829581\\
48.7350799953415	27.85563402931\\
49.2648091257256	27.9554977587539\\
49.7945382561098	28.0547514724118\\
50.3242673864939	28.1533959820129\\
50.8539965168781	28.2514415294134\\
51.3837256472622	28.3488950828302\\
51.9134547776464	28.4457595305732\\
52.4431839080305	28.5420401622043\\
52.9729130384147	28.637745857519\\
53.5026421687988	28.7328817417916\\
54.032371299183	28.8274506101559\\
54.5621004295671	28.921464407457\\
55.0918295599513	29.0149284610356\\
55.6215586903354	29.1078435297556\\
56.1512878207195	29.2002173394186\\
56.6810169511037	29.2920562561103\\
57.2107460814879	29.3833629412059\\
57.740475211872	29.4741407460868\\
58.2702043422561	29.5643977923855\\
58.7999334726403	29.6541385510696\\
59.3296626030244	29.7433621994931\\
59.8593917334086	29.8320785460401\\
60.3891208637927	29.9202919079992\\
60.9188499941769	30.008003129187\\
61.448579124561	30.0952180056211\\
61.9783082549452	30.1819425389165\\
62.5080373853293	30.268179349845\\
63.0377665157135	30.3539302729317\\
63.5674956460976	30.4392029331754\\
64.0972247764818	30.5240012092576\\
64.6269539068659	30.6083243495186\\
65.15668303725	30.6921803627614\\
65.6864121676342	30.7755733723382\\
66.2161412980183	30.8585043627552\\
66.7458704284025	30.9409779958851\\
67.2755995587866	31.023001853074\\
67.8053286891708	31.1045791174107\\
68.3350578195549	31.1857104576602\\
68.8647869499391	31.2664030841816\\
69.3945160803232	31.3466603137055\\
69.9242452107074	31.4264815986235\\
70.4539743410915	31.5058732155436\\
70.9837034714756	31.5848391028095\\
71.5134326018598	31.6633803675832\\
72.0431617322439	31.7414998542853\\
72.5728908626281	31.8192029862995\\
73.1026199930123	31.8964923004048\\
73.6323491233964	31.9733673609666\\
74.1620782537805	32.0498350979363\\
74.6918073841647	32.1258983852725\\
75.2215365145488	32.2015569104889\\
75.751265644933	32.2768155865768\\
76.2809947753171	32.3516783899396\\
76.8107239057013	32.4261464890508\\
77.3404530360854	32.5002215813705\\
77.8701821664696	32.5739089715913\\
78.3999112968537	32.6472113365456\\
78.9296404272379	32.7201273488016\\
79.459369557622	32.7926635859176\\
79.9890986880061	32.8648226614757\\
80.5188278183903	32.9366045897921\\
81.0485569487744	33.0080130632195\\
81.5782860791586	33.0790520868614\\
82.1080152095427	33.1497238587955\\
82.6377443399269	33.2200305298878\\
83.167473470311	33.2899771903057\\
83.6972026006952	33.3595662387827\\
84.2269317310793	33.4287965490199\\
84.7566608614635	33.4976735292318\\
85.2863899918476	33.5661997467937\\
85.8161191222318	33.6343754485588\\
86.3458482526159	33.7022032322885\\
86.875577383	33.7696868953523\\
87.4053065133842	33.836827997044\\
87.9350356437683	33.9036264209085\\
88.4647647741525	33.970087168511\\
88.9944939045366	34.0362124610837\\
89.5242230349208	34.1020013285099\\
90.0539521653049	34.1674581932384\\
90.5836812956891	34.2325857957643\\
91.1134104260732	34.2973843985036\\
91.6431395564574	34.3618558396757\\
92.1728686868415	34.4260040172217\\
92.7025978172257	34.4898303811109\\
93.2323269476098	34.5533343156745\\
93.762056077994	34.6165207860114\\
94.2917852083781	34.6793915623279\\
94.8215143387623	34.7419462125075\\
95.3512434691464	34.8041881580315\\
95.8809725995306	34.8661200402101\\
96.4107017299147	34.9277425851191\\
96.9404308602988	34.9890566813832\\
97.470159990683	35.0500662481959\\
97.9998891210671	35.1107730021797\\
98.5296182514513	35.171175496624\\
99.0593473818354	35.2312789891316\\
99.5890765122196	35.2910856920333\\
100.118805642604	35.350595784904\\
100.648534772988	35.4098123812153\\
101.178263903372	35.4687380837434\\
101.707993033756	35.5273738121525\\
102.23772216414	35.5857197271703\\
102.767451294524	35.6437797491059\\
103.297180424909	35.7015553977062\\
103.826909555293	35.7590454113376\\
104.356638685677	35.8162540679566\\
104.886367816061	35.8731829261803\\
105.416096946445	35.9298320639219\\
105.945826076829	35.9862031955854\\
106.475555207213	36.0422992521262\\
107.005284337598	36.0981210566079\\
107.535013467982	36.1536683461699\\
108.064742598366	36.2089449903563\\
108.59447172875	36.2639522715695\\
109.124200859134	36.3186894427606\\
109.653929989518	36.3731596312531\\
110.183659119903	36.4273649275281\\
110.713388250287	36.4813051740273\\
111.243117380671	36.5349818244959\\
111.772846511055	36.5883975381427\\
112.302575641439	36.6415534919367\\
112.832304771823	36.6944488151066\\
113.362033902207	36.7470873219359\\
113.891763032592	36.7994703380815\\
114.421492162976	36.8515970919443\\
114.95122129336	36.9034703243537\\
115.480950423744	36.9550918702703\\
116.010679554128	37.0064621250979\\
116.540408684512	37.0575815794501\\
117.070137814896	37.1084532952903\\
117.599866945281	37.1590782520634\\
118.129596075665	37.2094553792072\\
118.659325206049	37.25958823786\\
119.189054336433	37.3094782286749\\
119.718783466817	37.3591246301359\\
120.248512597201	37.4085296879473\\
120.778241727585	37.4576960256275\\
121.30797085797	37.5066245436509\\
121.837699988354	37.5553151635695\\
122.367429118738	37.6037709493287\\
122.897158249122	37.651992823217\\
123.426887379506	37.699979700345\\
123.95661650989	37.7477346830203\\
124.486345640274	37.7952590641471\\
125.016074770659	37.8425524912\\
125.545803901043	37.8896163198773\\
126.075533031427	37.9364527193734\\
126.605262161811	37.9830620865418\\
127.134991292195	38.0294442199622\\
127.664720422579	38.0756018000694\\
128.194449552964	38.1215360789033\\
128.724178683348	38.167245776888\\
129.253907813732	38.212733765533\\
129.783636944116	38.2580010972473\\
130.3133660745	38.3030478344934\\
130.843095204884	38.3478746294509\\
131.372824335268	38.3924837025887\\
131.902553465653	38.4368756991724\\
132.432282596037	38.4810496263258\\
132.962011726421	38.5250085983283\\
133.491740856805	38.5687532160855\\
134.021469987189	38.6122829269017\\
134.551199117573	38.6555994509179\\
135.080928247957	38.6987044199607\\
135.610657378342	38.7415976894475\\
136.140386508726	38.7842796023627\\
136.67011563911	38.826752513789\\
137.199844769494	38.8690169585157\\
137.729573899878	38.911071967086\\
138.259303030262	38.9529202399685\\
138.789032160646	38.9945627838316\\
139.318761291031	39.0359989251018\\
139.848490421415	39.0772300646277\\
140.378219551799	39.11825795089\\
140.907948682183	39.1590824156556\\
141.437677812567	39.1997036066468\\
141.967406942951	39.2401236206665\\
142.497136073335	39.2803431600581\\
143.02686520372	39.3203611408997\\
143.556594334104	39.36017981354\\
144.086323464488	39.3998002676397\\
144.616052594872	39.4392218204804\\
145.145781725256	39.4784454313472\\
145.67551085564	39.5174729126169\\
146.205239986024	39.5563044105295\\
146.734969116409	39.5949396530787\\
147.264698246793	39.6333811025384\\
147.794427377177	39.6716295489738\\
148.324156507561	39.7096841994409\\
148.853885637945	39.7475468321469\\
149.383614768329	39.7852185588568\\
149.913343898713	39.8226991702772\\
150.443073029098	39.8599889670399\\
150.972802159482	39.8970899081307\\
151.502531289866	39.9340022680878\\
152.03226042025	39.9707250590378\\
152.561989550634	40.0072609149209\\
153.091718681018	40.0436101638177\\
153.621447811403	40.0797719936675\\
154.151176941787	40.1157481602391\\
154.680906072171	40.1515394006509\\
155.210635202555	40.1871456894168\\
155.740364332939	40.2225672041903\\
156.270093463323	40.2578053822553\\
156.799822593707	40.2928609146395\\
157.329551724092	40.32773242698\\
157.859280854476	40.3624220863626\\
158.38900998486	40.3969306138421\\
158.918739115244	40.4312570616266\\
159.448468245628	40.4654023979956\\
159.978197376012	40.4993679524951\\
160.507926506396	40.5331534043475\\
161.037655636781	40.566758350128\\
161.567384767165	40.6001847685268\\
162.097113897549	40.6334328634832\\
162.626843027933	40.6665011782629\\
163.156572158317	40.6993919509982\\
163.686301288701	40.7321054390887\\
164.216030419085	40.7646407609731\\
164.74575954947	40.7969986563742\\
165.275488679854	40.8291799651647\\
165.805217810238	40.8611842674382\\
166.334946940622	40.893010732383\\
166.864676071006	40.9246607320397\\
167.39440520139	40.9561339239633\\
167.924134331774	40.9874286775039\\
168.453863462159	41.0185459892807\\
168.983592592543	41.0494854064427\\
169.513321722927	41.0802455799077\\
170.043050853311	41.1108258208951\\
170.572779983695	41.1412258928669\\
171.102509114079	41.1714445470833\\
171.632238244464	41.2014791559562\\
172.161967374848	41.2313296040457\\
172.691696505232	41.2609939675089\\
173.221425635616	41.2904686710472\\
173.751154766	41.3197520639451\\
174.280883896384	41.3488412512068\\
174.810613026768	41.377731838797\\
175.340342157153	41.4064190458296\\
175.870071287537	41.4348986272498\\
176.399800417921	41.4631642735294\\
176.929529548305	41.4912063244503\\
177.459258678689	41.5190178175316\\
177.988987809073	41.5465874009438\\
178.518716939457	41.5738996986852\\
179.048446069842	41.6009387527192\\
179.578175200226	41.62768443487\\
180.10790433061	41.6541096460807\\
180.637633460994	41.6801802144951\\
181.167362591378	41.7058578398003\\
181.697091721762	41.7310908838978\\
182.226820852146	41.7558067850332\\
182.756549982531	41.7799182753213\\
183.286279112915	41.80329123591\\
183.816008243299	41.8257199762712\\
184.345737373683	41.8469507347068\\
184.875466504067	41.8664618999392\\
185.405195634451	41.8820348638236\\
185.934924764836	41.8832078720147\\
186.46465389522	41.8103077544846\\
186.994383025604	41.6840419550254\\
187.524112155988	41.5441785059661\\
188.053841286372	41.3984805981385\\
188.583570416756	41.2496428601691\\
189.11329954714	41.098576106113\\
189.643028677525	40.9457879668698\\
190.172757807909	40.7915913775695\\
190.702486938293	40.6361649966231\\
191.232216068677	40.479622041555\\
191.761945199061	40.3220620501312\\
192.291674329445	40.1635517513958\\
192.821403459829	40.0041343274236\\
193.351132590214	39.8438521454737\\
193.880861720598	39.6827341878741\\
194.410590850982	39.5207961053512\\
194.940319981366	39.3580495580992\\
195.47004911175	39.194508572633\\
195.999778242134	39.0301783439413\\
196.529507372518	38.8650539765338\\
197.059236502903	38.6991442684209\\
197.588965633287	38.5324481809023\\
198.118694763671	38.364957530351\\
198.648423894055	38.1966712961852\\
199.178153024439	38.0275872422328\\
199.707882154823	37.8576967067238\\
200.237611285207	37.6869893418625\\
200.767340415592	37.515463239032\\
201.297069545976	37.3431097644903\\
201.82679867636	37.1699110207742\\
202.356527806744	36.9958646754348\\
202.886256937128	36.8209603896155\\
203.415986067512	36.6451812152486\\
203.945715197896	36.468516298674\\
204.475444328281	36.2909566343829\\
205.005173458665	36.1124865003829\\
205.534902589049	35.9330858681363\\
206.064631719433	35.7527472333947\\
206.594360849817	35.5714548558553\\
207.124089980201	35.3891834305153\\
207.653819110585	35.2059231673655\\
208.18354824097	35.0216580073232\\
208.713277371354	34.8363640625708\\
209.243006501738	34.6500212956518\\
209.772735632122	34.4626147647665\\
210.302464762506	34.2741221981883\\
210.83219389289	34.0845140479241\\
211.361923023275	33.8937795804844\\
211.891652153659	33.7018966323141\\
212.421381284043	33.5088312838196\\
212.951110414427	33.3145642639204\\
213.480839544811	33.1190728948896\\
214.010568675195	32.9223266707215\\
214.540297805579	32.724297680898\\
215.070026935964	32.5249683822285\\
215.599756066348	32.3243114121495\\
216.129485196732	32.1222851265108\\
216.659214327116	31.9188743953269\\
217.1889434575	31.7140482100313\\
217.718672587884	31.5077547372865\\
218.248401718268	31.2998960737642\\
218.778130848653	31.0898812103105\\
219.307859979037	30.8760974327136\\
219.837589109421	30.6586303651119\\
220.367318239805	30.4377153714991\\
220.897047370189	30.2130050938336\\
221.426776500573	29.9841439909048\\
221.956505630957	29.751252604892\\
222.486234761342	29.5141677507404\\
223.015963891726	29.2727506721883\\
223.54569302211	29.026903826963\\
224.075422152494	28.7765951613111\\
224.605151282878	28.5218818252548\\
225.134880413262	28.2631668789147\\
225.664609543646	28.0014034753257\\
226.194338674031	27.7370940792707\\
226.724067804415	27.4699219902771\\
227.253796934799	27.199743265366\\
227.783526065183	26.9265474563924\\
228.313255195567	26.650272181127\\
228.842984325951	26.3707893808166\\
229.372713456336	26.0880160860265\\
229.90244258672	25.8018593082795\\
230.432171717104	25.5121905708107\\
230.961900847488	25.2189234116312\\
231.491629977872	24.921935568356\\
232.021359108256	24.6210634872487\\
232.55108823864	24.3161859760707\\
233.080817369025	24.0071584222011\\
233.610546499409	23.6937949391947\\
234.140275629793	23.3759040703479\\
234.670004760177	23.0533207899623\\
235.199733890561	22.7258408975924\\
235.729463020945	22.3932068950086\\
236.259192151329	22.0552644501773\\
236.788921281714	21.7118045199439\\
237.318650412098	21.362553420324\\
237.848379542482	21.0072690064616\\
238.378108672866	20.6456835207603\\
238.90783780325	20.2774637422149\\
239.437566933634	19.9022627919179\\
239.967296064018	19.5198211923838\\
240.497025194403	19.1299342185722\\
241.026754324787	18.7325205562154\\
241.556483455171	18.3276647182528\\
242.086212585555	17.9152538147716\\
242.615941715939	17.4950068914828\\
243.145670846323	17.0666312050775\\
243.675399976707	16.6297735153845\\
244.205129107092	16.1838610866128\\
244.734858237476	15.7280087133556\\
245.26458736786	15.2611202956513\\
245.794316498244	14.7815085751692\\
246.324045628628	14.2863553945932\\
246.853774759012	13.7721574495788\\
247.383503889397	13.2354557177253\\
247.913233019781	12.6730033311302\\
248.442962150165	12.0819901432343\\
248.972691280549	11.4603008099731\\
249.502420410933	10.8061102966086\\
250.032149541317	10.1174219739559\\
250.561878671701	9.39198890122335\\
251.091607802086	8.62492815706305\\
251.62133693247	7.80556392850734\\
252.151066062854	6.91459216701813\\
252.680795193238	5.92270506247001\\
253.210524323622	5.69245876529854\\
253.740253454006	6.26217481847052\\
254.26998258439	6.49917752917768\\
254.799711714775	6.59663696955438\\
255.329440845159	6.62795673802993\\
255.859169975543	6.62815952556407\\
256.388899105927	6.61439669455875\\
256.918628236311	6.59555172583066\\
257.448357366695	6.57616961764462\\
257.978086497079	6.55835505850377\\
258.507815627464	6.54294438389353\\
259.037544757848	6.53011065055497\\
259.567273888232	6.51967152748604\\
260.097003018616	6.51132912953372\\
260.626732149	6.50476459728078\\
261.156461279384	6.49963763263657\\
261.686190409768	6.49566264936916\\
262.215919540153	6.49259911368158\\
262.745648670537	6.4902279965918\\
263.275377800921	6.48803763907546\\
263.805106931305	6.48688107744583\\
264.334836061689	6.4860399576012\\
264.864565192073	6.48546877734539\\
265.394294322457	6.48515167248619\\
265.924023452842	6.48504377455308\\
266.453752583226	6.4849433330087\\
266.98348171361	6.48411455041014\\
267.513210843994	6.47896513846534\\
268.042939974378	6.42826374998859\\
268.572669104762	6.56657984766464\\
269.102398235146	6.98811279480685\\
269.632127365531	7.51607601923136\\
270.161856495915	8.29116663986908\\
270.691585626299	8.97480097452564\\
271.221314756683	9.65459215609141\\
271.751043887067	10.3693456091093\\
272.280773017451	11.066527058849\\
272.810502147836	11.732195300252\\
273.34023127822	12.3546565469001\\
273.869960408604	12.6140334905157\\
274.399689538988	12.0137880918208\\
274.929418669372	11.3931761799574\\
275.459147799756	10.7635713371337\\
275.98887693014	10.1272938374543\\
276.518606060525	9.48265109598431\\
277.048335190909	8.82557064099503\\
277.578064321293	8.14923193754053\\
278.107793451677	7.44074165766139\\
278.637522582061	6.68116935711209\\
279.167251712445	5.84574413132094\\
279.696980842829	6.08249491183188\\
280.226709973214	6.39526599180421\\
280.756439103598	6.54067883130637\\
281.286168233982	6.60387758330578\\
281.815897364366	6.62428334940173\\
282.34562649475	6.62268842413583\\
282.875355625134	6.6104452516204\\
283.405084755518	6.59383551164797\\
283.934813885903	6.57635256688803\\
284.464543016287	6.55981561546004\\
284.994272146671	6.545051494141\\
285.524001277055	6.53200268314204\\
286.053730407439	6.5216694468821\\
286.583459537823	6.51318213758404\\
287.113188668207	6.50632363034194\\
287.642917798592	6.50085838630583\\
288.172646928976	6.49655027391849\\
288.70237605936	6.49319274637375\\
289.232105189744	6.49060919546328\\
289.761834320128	6.48863960199691\\
290.291563450512	6.48715943745709\\
290.821292580897	6.4860712067557\\
291.351021711281	6.4852932117959\\
291.880750841665	6.48475865857973\\
292.410479972049	6.48438414594473\\
292.940209102433	6.48398101234615\\
293.469938232817	6.4832870060387\\
293.999667363201	6.4805959516213\\
294.529396493586	6.77418203968405\\
295.05912562397	7.88671409071411\\
295.588854754354	8.72081126610566\\
296.118583884738	9.49369454499743\\
296.648313015122	10.2173902641114\\
297.178042145506	10.895045019015\\
297.70777127589	11.5249656319648\\
298.237500406275	12.1064219840407\\
298.767229536659	12.6405668279496\\
299.296958667043	13.1294617829676\\
299.826687797427	13.5757362851213\\
300.356416927811	13.9835945360802\\
300.886146058195	14.3568830510891\\
301.415875188579	14.6999211322178\\
301.945604318964	15.0251867456699\\
302.475333449348	15.3380705935167\\
303.005062579732	15.6397651496282\\
303.534791710116	15.9312615374914\\
304.0645208405	16.2135244156161\\
304.594249970884	16.4873061774556\\
305.123979101269	16.7531345669088\\
305.653708231653	17.0115787240369\\
306.183437362037	17.2630810142285\\
306.713166492421	17.5079972274535\\
307.242895622805	17.7467087753166\\
307.772624753189	17.9795671539363\\
308.302353883573	18.2068556106496\\
308.832083013958	18.4288115611644\\
309.361812144342	18.6457446886413\\
309.891541274726	18.8579114503169\\
310.42127040511	19.0655303205887\\
310.950999535494	19.2688494019907\\
311.480728665878	19.46808361631\\
312.010457796262	19.6634071877292\\
312.540186926647	19.8549914330146\\
313.069916057031	20.0430184804692\\
313.599645187415	20.2276406285407\\
314.129374317799	20.4089513005306\\
314.659103448183	20.5870711265604\\
315.188832578567	20.7620530245267\\
315.718561708951	20.9339228320747\\
316.248290839336	21.1028006178827\\
316.77801996972	21.2696836916834\\
317.307749100104	21.4351055997156\\
317.837478230488	21.5990333108466\\
318.367207360872	21.761441928745\\
318.896936491256	21.9222977135563\\
319.42666562164	22.0815546318067\\
319.956394752025	22.2392044961447\\
320.486123882409	22.3952302289316\\
321.015853012793	22.5496096586217\\
321.545582143177	22.7023405103659\\
322.075311273561	22.8534286365837\\
322.605040403945	23.0028733466975\\
323.13476953433	23.150673393113\\
323.664498664714	23.2968596042865\\
324.194227795098	23.4414477579317\\
324.723956925482	23.5844377466577\\
325.253686055866	23.7258583437151\\
325.78341518625	23.8657288879158\\
326.313144316634	24.0040596339503\\
326.842873447019	24.1408701054546\\
327.372602577403	24.276187938466\\
327.902331707787	24.4100308030068\\
328.432060838171	24.5424080570854\\
328.961789968555	24.6733532528054\\
329.491519098939	24.8028865211896\\
330.021248229323	24.931017266397\\
330.550977359708	25.0577802984279\\
331.080706490092	25.183199895619\\
331.610435620476	25.3072892236023\\
332.14016475086	25.4300651300531\\
332.669893881244	25.5515543540648\\
333.199623011628	25.6717745335913\\
333.729352142012	25.7907322640086\\
334.259081272397	25.9084580396719\\
334.788810402781	26.0249694061959\\
335.318539533165	26.1402749287176\\
335.848268663549	26.2543968069313\\
336.377997793933	26.3673555063262\\
336.907726924317	26.4791633465752\\
337.437456054701	26.5898324644126\\
337.967185185086	26.6993887827111\\
338.49691431547	26.8078511141811\\
339.026643445854	26.9152242548734\\
339.556372576238	27.0215334773357\\
340.086101706622	27.1267932004721\\
340.615830837006	27.2310106215836\\
341.14555996739	27.3342344207828\\
341.675289097775	27.436733927257\\
342.205018228159	27.5385769543839\\
342.734747358543	27.6397687563289\\
343.264476488927	27.740322290329\\
343.794205619311	27.8402451299144\\
344.323934749695	27.9395389589224\\
344.853663880079	28.038216185717\\
345.383393010464	28.1362847995322\\
345.913122140848	28.2337485684545\\
346.442851271232	28.330614688243\\
346.972580401616	28.4268929949173\\
347.502309532	28.5225891554982\\
348.032038662384	28.6177054426964\\
348.561767792768	28.7122534887947\\
349.091496923153	28.8062413391322\\
349.621226053537	28.8996720251563\\
350.150955183921	28.9925553268278\\
350.680684314305	29.0848982597714\\
351.210413444689	29.1767041128161\\
351.740142575073	29.267977719165\\
352.269871705458	29.3587278305909\\
352.799600835842	29.4489595069381\\
353.329329966226	29.5386733366127\\
353.85905909661	29.6278796838111\\
354.388788226994	29.7165836680624\\
354.918517357378	29.8047866563253\\
355.448246487762	29.8924957555225\\
355.977975618147	29.9797174236029\\
356.507704748531	30.0664547398857\\
357.037433878915	30.1527107737071\\
357.567163009299	30.2384935171007\\
358.096892139683	30.3238074690485\\
358.626621270067	30.4086522724457\\
359.156350400451	30.4930369415237\\
359.686079530836	30.576965834796\\
360.21580866122	30.6604401974662\\
360.745537791604	30.7434651523747\\
361.275266921988	30.8260465041567\\
361.804996052372	30.908186995776\\
362.334725182756	30.9898897163314\\
362.864454313141	31.0711629514403\\
363.394183443525	31.1520105840031\\
363.923912573909	31.2324321656014\\
364.453641704293	31.3124345487071\\
364.983370834677	31.3920215856443\\
365.513099965061	31.471194308972\\
366.042829095445	31.5499561571424\\
366.57255822583	31.6283123124272\\
367.102287356214	31.7062651112606\\
367.632016486598	31.7838145796833\\
368.161745616982	31.860967231333\\
368.691474747366	31.9377258150022\\
369.22120387775	32.0140896715146\\
369.750933008134	32.0900637771609\\
370.280662138519	32.1656514793734\\
370.810391268903	32.2408533984559\\
371.340120399287	32.3156711685285\\
371.869849529671	32.390109191847\\
372.399578660055	32.4641692519258\\
372.929307790439	32.5378498201875\\
373.459036920823	32.6111563711686\\
373.988766051208	32.6840905299155\\
374.518495181592	32.7566511868202\\
375.048224311976	32.8288413894197\\
375.57795344236	32.9006635989429\\
376.107682572744	32.9721177241676\\
376.637411703128	33.0432038005686\\
377.167140833512	33.1139256185165\\
377.696869963897	33.1842854643873\\
378.226599094281	33.2542809565325\\
378.756328224665	33.323916310703\\
379.286057355049	33.3931919946075\\
379.815786485433	33.4621060417114\\
380.345515615817	33.5306595623189\\
380.875244746201	33.5988540052374\\
381.404973876586	33.6666883674828\\
381.93470300697	33.7341606419232\\
382.464432137354	33.8012732897239\\
382.994161267738	33.8680258759504\\
383.523890398122	33.9344145602169\\
384.053619528506	34.0004415196821\\
384.58334865889	34.0661064076122\\
385.113077789275	34.1314064109737\\
385.642806919659	34.1963405580443\\
386.172536050043	34.2609096073598\\
386.702265180427	34.3251118013094\\
387.231994310811	34.3889430086588\\
387.761723441195	34.4524052093313\\
388.291452571579	34.515496929453\\
388.821181701964	34.5782133331617\\
389.350910832348	34.6405552968951\\
389.880639962732	34.702521875674\\
390.410369093116	34.764109602518\\
390.9400982235	34.825316306144\\
391.469827353884	34.8861423183904\\
391.999556484269	34.9465856955525\\
392.529285614653	35.0066410454521\\
393.059014745037	35.0663102577214\\
393.588743875421	35.1255916863411\\
394.118473005805	35.1844804119221\\
394.648202136189	35.2429767935117\\
395.177931266574	35.3010807829412\\
395.707660396958	35.3587896923794\\
396.237389527342	35.4161010271772\\
396.767118657726	35.4730158449119\\
397.29684778811	35.529532907687\\
397.826576918494	35.5856464284053\\
398.356306048878	35.6413592714825\\
398.886035179262	35.6966702393611\\
399.415764309647	35.7515756625542\\
399.945493440031	35.8060758022274\\
400.475222570415	35.8601715845097\\
401.004951700799	35.9138615057491\\
401.534680831183	35.9671425311315\\
402.064409961567	36.0200170079055\\
402.594139091951	36.0724833336975\\
403.123868222336	36.1245370094647\\
403.65359735272	36.1761808112077\\
404.183326483104	36.2274153414934\\
404.713055613488	36.2782380006791\\
405.242784743872	36.3286489452097\\
405.772513874256	36.378649615792\\
406.302243004641	36.4282391749927\\
406.831972135025	36.4774140883282\\
407.361701265409	36.526177657567\\
407.891430395793	36.5745285944056\\
408.421159526177	36.6224620946477\\
408.950888656561	36.6699785716555\\
409.480617786945	36.717076739397\\
410.010346917329	36.7637512679481\\
410.540076047714	36.8099978113055\\
411.069805178098	36.8558133457147\\
411.599534308482	36.9011903182768\\
412.129263438866	36.9461144813043\\
412.65899256925	36.9905778305018\\
413.188721699634	37.0345652714415\\
413.718450830018	37.078053663784\\
414.248179960403	37.1210202807015\\
414.777909090787	37.1634331766992\\
415.307638221171	37.2052446631276\\
415.837367351555	37.2463896026907\\
416.367096481939	37.2867884655454\\
416.896825612323	37.3263232374061\\
417.426554742708	37.3648087145086\\
417.956283873092	37.4019974939485\\
418.486013003476	37.43749124417\\
419.01574213386	37.4705442186178\\
419.545471264244	37.4998013639821\\
420.075200394628	37.522816516959\\
420.604929525012	37.5077817802537\\
421.134658655397	37.3972065294544\\
421.664387785781	37.2387129514433\\
422.194116916165	37.0694247465053\\
422.723846046549	36.8949332872309\\
423.253575176933	36.7162987330758\\
423.783304307317	36.5341885890858\\
424.313033437702	36.3489698434576\\
424.842762568086	36.1608954015129\\
425.37249169847	35.9701455130216\\
425.902220828854	35.7768144049382\\
426.431949959238	35.580967352578\\
426.961679089622	35.3826845561718\\
427.491408220006	35.1820265729259\\
428.021137350391	34.9790315685547\\
428.550866480775	34.7737765932103\\
429.080595611159	34.5663367345867\\
429.610324741543	34.3567815862364\\
430.140053871927	34.1451969697792\\
430.669783002311	33.9316878015163\\
431.199512132695	33.7163453123291\\
431.72924126308	33.4992417681898\\
432.258970393464	33.2804704249468\\
432.788699523848	33.0600895843499\\
433.318428654232	32.8381251902136\\
433.848157784616	32.6146004623737\\
434.377886915	32.389515827348\\
434.907616045384	32.1628438398786\\
435.437345175769	31.9345438478297\\
435.967074306153	31.704581984733\\
436.496803436537	31.4729071544933\\
437.026532566921	31.2394454760521\\
437.556261697305	31.0041579440673\\
438.085990827689	30.7669872452129\\
438.615719958073	30.52786201154\\
439.145449088458	30.2867276406765\\
439.675178218842	30.0435320517271\\
440.204907349226	29.7982082842849\\
440.73463647961	29.5506831525877\\
441.264365609994	29.3009140181564\\
441.794094740378	29.0488476010845\\
442.323823870763	28.7944219497806\\
442.853553001147	28.537641815159\\
443.383282131531	28.2785348215852\\
443.913011261915	28.0171734724608\\
444.442740392299	27.7537263094856\\
444.972469522683	27.488464291767\\
445.502198653067	27.221745995703\\
446.031927783451	26.9540150730893\\
446.561656913836	26.6857612424375\\
447.09138604422	26.4174259898211\\
447.621115174604	26.149349453765\\
448.150844304988	25.8817278126632\\
448.680573435372	25.6146170065389\\
449.210302565756	25.3479466381212\\
449.740031696141	25.0815300698714\\
450.269760826525	24.8150896101684\\
450.799489956909	24.5482776106048\\
451.329219087293	24.2806888158722\\
451.858948217677	24.0119321658012\\
452.388677348061	23.7416406502151\\
452.918406478445	23.4694794271377\\
453.44813560883	23.1951839215536\\
453.977864739214	22.9185130316756\\
454.507593869598	22.6391944505759\\
455.037322999982	22.356891220753\\
455.567052130366	22.0712647512019\\
456.09678126075	21.7819927704394\\
456.626510391134	21.4887148639584\\
457.156239521519	21.1911387212549\\
457.685968651903	20.8889151389945\\
458.215697782287	20.5816410384769\\
458.745426912671	20.2689411908297\\
459.275156043055	19.9504342066529\\
459.804885173439	19.6256770850856\\
460.334614303824	19.2941766004629\\
460.864343434208	18.9554939139373\\
461.394072564592	18.6091054725278\\
461.923801694976	18.2543659107152\\
462.45353082536	17.8907378825697\\
462.983259955744	17.5175856258633\\
463.512989086128	17.1341534870334\\
464.042718216513	16.7396886102929\\
464.572447346897	16.3334279217049\\
465.102176477281	15.9144643138672\\
465.631905607665	15.4817397937199\\
466.161634738049	15.0343533782719\\
466.691363868433	14.5712642522458\\
467.221092998817	14.0913370205111\\
467.750822129202	13.5939816221451\\
468.280551259586	13.0795179370939\\
468.81028038997	12.5515207416157\\
469.340009520354	12.0259283781482\\
469.869738650738	11.5852884083228\\
470.399467781122	11.4357783391275\\
470.929196911506	11.4742372853528\\
471.458926041891	11.5294758384361\\
471.988655172275	11.544093118623\\
472.518384302659	11.5289799120608\\
473.048113433043	11.5093578817703\\
473.577842563427	11.5009704560743\\
474.107571693811	11.5084797711884\\
474.637300824195	11.5292611408025\\
475.16702995458	11.557584960427\\
475.696759084964	11.5877076012947\\
476.226488215348	11.6150486401203\\
476.756217345732	11.6369593402103\\
477.285946476116	11.6529872033505\\
477.8156756065	11.6650270888189\\
478.345404736884	11.6773796200732\\
478.875133867269	11.6963014049563\\
479.404862997653	11.7295336306557\\
479.934592128037	11.7856751399856\\
480.464321258421	11.8726235536337\\
480.994050388805	11.9960030196606\\
481.523779519189	12.1566577183214\\
482.053508649573	12.3494924892554\\
482.583237779958	12.5655405805346\\
483.112966910342	12.7916607808226\\
483.642696040726	13.0117031113445\\
484.17242517111	13.2084900229298\\
484.702154301494	13.3625158653061\\
485.231883431878	13.4472571967365\\
485.761612562262	13.4411702598684\\
486.291341692647	13.3358235699622\\
486.821070823031	13.1345268506743\\
487.350799953415	12.8528461121753\\
487.880529083799	12.5138160003561\\
488.410258214183	12.1466162943953\\
488.939987344567	11.7891258711152\\
489.469716474951	11.4920791024406\\
489.999445605336	11.3132091395501\\
490.52917473572	11.264598026308\\
491.058903866104	11.2752024102098\\
491.588632996488	11.2969149331383\\
492.118362126872	11.319064099808\\
492.648091257256	11.3414363624237\\
493.177820387641	11.3638061815829\\
493.707549518025	11.3861464914165\\
494.237278648409	11.408347651442\\
494.767007778793	11.4302756154343\\
495.296736909177	11.4517798276083\\
495.826466039561	11.4726751644234\\
496.356195169945	11.4927534701048\\
496.88592430033	11.5117531031273\\
497.415653430714	11.5293348377187\\
497.945382561098	11.5450894883422\\
498.475111691482	11.5585213874562\\
499.004840821866	11.5690258858356\\
499.53456995225	11.5757973237262\\
500.064299082634	11.5774303358773\\
500.594028213019	11.6018724579708\\
501.123757343403	12.0162441517323\\
501.653486473787	12.5049928564143\\
502.183215604171	12.9677962390891\\
502.712944734555	13.398840397926\\
503.242673864939	13.7988202222262\\
503.772402995324	14.1694719209551\\
504.302132125708	14.5126953685106\\
504.831861256092	14.8338764542094\\
505.361590386476	15.1433582245873\\
505.89131951686	15.4423167881326\\
506.421048647244	15.7311240934194\\
506.950777777628	16.0102941513507\\
507.480506908013	16.2802969997684\\
508.010236038397	16.5415296652781\\
508.539965168781	16.7943794104901\\
509.069694299165	17.0392624002705\\
509.599423429549	17.2765166484348\\
510.129152559933	17.5064206487851\\
510.658881690317	17.7293963812172\\
511.188610820702	17.945757276387\\
511.718339951086	18.1557617415503\\
512.24806908147	18.3597439436884\\
512.777798211854	18.5580144799918\\
513.307527342238	18.7508343138795\\
513.837256472622	18.9384721143863\\
514.366985603006	19.1212109257853\\
514.896714733391	19.2992712159766\\
515.426443863775	19.4728125457528\\
515.956172994159	19.6420771454417\\
516.485902124543	19.8072462896205\\
517.015631254927	19.9684656990376\\
517.545360385311	20.1259176805699\\
518.075089515695	20.2797827506889\\
518.60481864608	20.43019705732\\
519.134547776464	20.5772828693786\\
519.664276906848	20.7211970962295\\
520.194006037232	20.8620636075391\\
520.723735167616	20.9999759431869\\
521.253464298	21.1352357055587\\
521.783193428384	21.2687272967718\\
522.312922558769	21.4007257630302\\
522.842651689153	21.5313599713031\\
523.372380819537	21.6607351654766\\
523.902109949921	21.788900934366\\
524.431839080305	21.9158998446081\\
524.961568210689	22.0417899207578\\
525.491297341074	22.1666171092486\\
526.021026471458	22.290419424709\\
526.550755601842	22.4132545146927\\
527.080484732226	22.5351810833605\\
527.61021386261	22.6562627358362\\
528.139942992994	22.7765821944219\\
528.669672123378	22.8962440646987\\
529.199401253763	23.0153655227118\\
529.729130384147	23.1340762224029\\
530.258859514531	23.2525154938138\\
530.788588644915	23.3707993237203\\
531.318317775299	23.4890118209104\\
531.848046905683	23.6071963902547\\
532.377776036067	23.7253497284657\\
532.907505166452	23.843420159705\\
533.437234296836	23.9613056824477\\
533.96696342722	24.0788610442103\\
534.496692557604	24.1959065933511\\
535.026421687988	24.3122388952785\\
535.556150818372	24.4276842551308\\
536.085879948756	24.5420984478151\\
536.615609079141	24.655373375339\\
537.145338209525	24.7674563082328\\
537.675067339909	24.8783330122339\\
538.204796470293	24.9880107196275\\
538.734525600677	25.0965128708525\\
539.264254731061	25.2038826735754\\
539.793983861445	25.310158104502\\
540.32371299183	25.4153675575237\\
540.853442122214	25.5195570725551\\
541.383171252598	25.6227608531079\\
541.912900382982	25.7250031482527\\
542.442629513366	25.826311523361\\
542.97235864375	25.9267135694545\\
543.502087774134	26.0262280422324\\
544.031816904519	26.1248678579662\\
544.561546034903	26.2226576775933\\
545.091275165287	26.3196136103323\\
545.621004295671	26.4157439859173\\
546.150733426055	26.5110695023305\\
546.680462556439	26.605606092998\\
547.210191686824	26.6993662919058\\
547.739920817208	26.7923656108493\\
548.269649947592	26.8846223210949\\
548.799379077976	26.9761483882155\\
549.32910820836	27.0669504381844\\
549.858837338744	27.1570479768248\\
550.388566469128	27.2464534992905\\
550.918295599513	27.3352097227249\\
551.448024729897	27.4235502203957\\
551.977753860281	27.511525306388\\
552.507482990665	27.5991430092044\\
553.037212121049	27.686411097623\\
553.566941251433	27.7733407873277\\
554.096670381817	27.8599411715545\\
554.626399512202	27.9462185063013\\
555.156128642586	28.0321861852057\\
555.68585777297	28.1178545346962\\
556.215586903354	28.2032323454389\\
556.745316033738	28.2883316627941\\
557.275045164122	28.3731648678965\\
557.804774294507	28.4577426368809\\
558.334503424891	28.5420753701016\\
558.864232555275	28.6261763141662\\
559.393961685659	28.7100567113023\\
559.923690816043	28.7937274172053\\
560.453419946427	28.8772029460903\\
560.983149076811	28.9604954606047\\
561.512878207196	29.0436157743616\\
562.04260733758	29.12657631858\\
562.572336467964	29.2093901287565\\
563.102065598348	29.2920691540372\\
563.631794728732	29.3746247645575\\
564.161523859116	29.4570697942169\\
564.6912529895	29.5394154394604\\
565.220982119885	29.6216716728464\\
565.750711250269	29.7038491354705\\
566.280440380653	29.7859568937529\\
566.810169511037	29.868002244237\\
567.339898641421	29.9499910207097\\
567.869627771805	30.0319272844507\\
568.399356902189	30.1138124533988\\
568.929086032574	30.1956451007648\\
569.458815162958	30.2774222066826\\
569.988544293342	30.3591376005536\\
570.518273423726	30.4407816309257\\
571.04800255411	30.5223425087371\\
571.577731684494	30.6038053422241\\
572.107460814878	30.6851515431785\\
572.637189945263	30.7663597947282\\
573.166919075647	30.8474080813477\\
573.696648206031	30.9282723466721\\
574.226377336415	31.0089265683848\\
574.756106466799	31.0893470489887\\
575.285835597183	31.1695070395855\\
575.815564727567	31.2493773793607\\
576.345293857952	31.3289319131523\\
576.875022988336	31.4081456163137\\
577.40475211872	31.4869917598341\\
577.934481249104	31.5654433078313\\
578.464210379488	31.643478660314\\
578.993939509872	31.7210739500096\\
579.523668640257	31.7982009502411\\
580.053397770641	31.8748402724855\\
580.583126901025	31.9509688076174\\
581.112856031409	32.0265587992618\\
581.642585161793	32.1015843023682\\
582.172314292177	32.176018583602\\
582.702043422561	32.2498260111246\\
583.231772552945	32.3229592157356\\
583.76150168333	32.3953680362677\\
584.291230813714	32.4669762529532\\
584.820959944098	32.5376634847252\\
585.350689074482	32.6072582200542\\
585.880418204866	32.6754561418451\\
586.41014733525	32.7416421459969\\
586.939876465635	32.8044131878663\\
587.469605596019	32.859837686623\\
587.999334726403	32.865657489975\\
588.529063856787	32.7029310276922\\
589.058792987171	32.5154688599877\\
589.588522117555	32.3219879670832\\
590.118251247939	32.1249851227892\\
590.647980378323	31.9254401798439\\
591.177709508708	31.7237880036976\\
591.707438639092	31.5202399484459\\
592.237167769476	31.3149159094529\\
592.76689689986	31.1079005958686\\
593.296626030244	30.8992370712561\\
593.826355160628	30.6889334478815\\
594.356084291013	30.4770186591864\\
594.885813421397	30.2634951690138\\
595.415542551781	30.0483472230551\\
595.945271682165	29.8315695323239\\
596.475000812549	29.6131529395194\\
597.004729942933	29.3930731047559\\
597.534459073318	29.171299225813\\
598.064188203702	28.9478174912407\\
598.593917334086	28.7225981831136\\
599.12364646447	28.4955904687382\\
599.653375594854	28.2667732001145\\
600.183104725238	28.0361072704718\\
600.712833855622	27.8035365090433\\
601.242562986006	27.5690138349524\\
601.772292116391	27.3324952517781\\
602.302021246775	27.0939188598\\
602.831750377159	26.8532062097394\\
603.361479507543	26.6103028324096\\
603.891208637927	26.365127379829\\
604.420937768311	26.1175649226664\\
604.950666898696	25.8675223894845\\
605.48039602908	25.6148682205798\\
606.010125159464	25.3594063844733\\
606.539854289848	25.1006175833811\\
607.069583420232	24.8366483672217\\
607.599312550616	24.5668804269411\\
608.129041681	24.290705645122\\
608.658770811384	24.0076733822183\\
609.188499941769	23.7179389209521\\
609.718229072153	23.4253061278556\\
610.247958202537	23.1308571512692\\
610.777687332921	22.8329126494736\\
611.307416463305	22.5308800439072\\
611.837145593689	22.2245815354271\\
612.366874724074	21.913794180747\\
612.896603854458	21.5982744012186\\
613.426332984842	21.2777214695944\\
613.956062115226	20.9519087326115\\
614.48579124561	20.6205299616746\\
615.015520375994	20.2832103286255\\
615.545249506378	19.939602645963\\
616.074978636763	19.5893440198049\\
616.604707767147	19.2320025168469\\
617.134436897531	18.8671023142164\\
617.664166027915	18.4942279455063\\
618.193895158299	18.112873915893\\
618.723624288683	17.7223971308045\\
619.253353419067	17.3222688307606\\
619.783082549452	16.9118278810905\\
620.312811679836	16.4902693471236\\
620.84254081022	16.056792579364\\
621.372269940604	15.6105865939892\\
621.901999070988	15.1507888812103\\
622.431728201372	14.6770473532917\\
622.961457331756	14.2044689615579\\
623.491186462141	14.1702756849513\\
624.020915592525	14.1794268849382\\
624.550644722909	14.1860985981506\\
625.080373853293	14.1898041125526\\
625.610102983677	14.189971198385\\
626.139832114061	14.1859113858355\\
626.669561244446	14.1768289918273\\
627.19929037483	14.1617245702009\\
627.729019505214	14.1393625491897\\
628.258748635598	14.1084357752845\\
628.788477765982	14.0674620240614\\
629.318206896366	14.0148365333296\\
629.84793602675	13.9492338892591\\
630.377665157135	13.8697752651864\\
630.907394287519	13.7762339785978\\
631.437123417903	13.6693062211657\\
631.966852548287	13.5504342608022\\
632.496581678671	13.4208451094114\\
633.026310809055	13.2804440608706\\
633.556039939439	13.1271909492513\\
634.085769069824	12.9560578646375\\
634.615498200208	12.7580184743586\\
635.145227330592	12.5190379657393\\
635.674956460976	12.2171586208217\\
636.20468559136	11.8221050118157\\
636.734414721744	11.3248351519925\\
637.264143852128	10.7532939392426\\
637.793872982513	10.043525417138\\
638.323602112897	9.16239735819785\\
638.853331243281	8.1927865588506\\
639.383060373665	7.16178011692174\\
639.912789504049	6.84497655593135\\
640.442518634433	6.99846351144934\\
640.972247764817	7.96601709349301\\
641.501976895202	8.58986859732862\\
642.031706025586	9.08519676938617\\
642.56143515597	9.50045493447318\\
643.091164286354	9.86863087955938\\
643.620893416738	10.2147135490811\\
644.150622547122	10.5581429896515\\
644.680351677507	10.9134630146696\\
645.210080807891	11.2904288459932\\
645.739809938275	11.6922342422186\\
646.269539068659	12.115161872536\\
646.799268199043	12.5502997744848\\
647.328997329427	12.986337906024\\
647.858726459811	13.4126297857827\\
648.388455590196	13.8212357246217\\
648.91818472058	14.2068507586474\\
649.447913850964	14.5671236300814\\
649.977642981348	14.9073316071726\\
650.507372111732	15.2353368764041\\
651.037101242116	15.5512005823643\\
651.5668303725	15.8553276343647\\
652.096559502885	16.1483315085239\\
652.626288633269	16.4308995702895\\
653.156017763653	16.7039168290826\\
653.685746894037	16.9681046190534\\
654.215476024421	17.2240613920644\\
654.745205154805	17.4724003665421\\
655.274934285189	17.7136813659402\\
655.804663415574	17.9483220078485\\
656.334392545958	18.1766747623925\\
656.864121676342	18.3991196717903\\
657.393850806726	18.6159531615643\\
657.92357993711	18.8274020320684\\
658.453309067494	19.0337910532538\\
658.983038197878	19.2353559697068\\
659.512767328263	19.4322851590341\\
660.042496458647	19.6247917471131\\
660.572225589031	19.8130886594511\\
661.101954719415	19.9973500994373\\
661.631683849799	20.1777397779265\\
662.161412980183	20.3544687679388\\
662.691142110568	20.5276925320325\\
663.220871240952	20.6975286527935\\
663.750600371336	20.8641419129822\\
664.28032950172	21.0276759857555\\
664.810058632104	21.1886672310727\\
665.339787762488	21.3482304662618\\
665.869516892872	21.5065268383387\\
666.399246023257	21.6636117169024\\
666.928975153641	21.8195161807593\\
667.458704284025	21.974279239944\\
667.988433414409	22.1279165237678\\
668.518162544793	22.280421523407\\
669.047891675177	22.4317838701385\\
669.577620805561	22.5819867365282\\
670.107349935946	22.7310067611708\\
670.63707906633	22.8788183344885\\
671.166808196714	23.0254042405523\\
671.696537327098	23.1707398681005\\
672.226266457482	23.3147956207593\\
672.755995587866	23.4575649344844\\
673.28572471825	23.5990292631706\\
673.815453848635	23.7391637626712\\
674.345182979019	23.8779615469717\\
674.874912109403	24.0154197095321\\
675.404641239787	24.1515296119492\\
675.934370370171	24.2862845634443\\
676.464099500555	24.4196962436155\\
676.993828630939	24.5517687932114\\
677.523557761324	24.6824953983229\\
678.053286891708	24.8118965611539\\
678.583016022092	24.9399838311561\\
679.112745152476	25.066766479124\\
679.64247428286	25.1922596195194\\
680.172203413244	25.3164811928232\\
680.701932543629	25.4394410774996\\
681.231661674013	25.5611450950009\\
681.761390804397	25.6816174468494\\
682.291119934781	25.8008719571196\\
682.820849065165	25.9189118678587\\
683.350578195549	26.0357604299342\\
683.880307325933	26.1514337702255\\
684.410036456317	26.2659402319996\\
684.939765586702	26.3792937966318\\
685.469494717086	26.4915149208184\\
685.99922384747	26.6026161887607\\
686.528952977854	26.7126054832712\\
687.058682108238	26.8215102380655\\
687.588411238622	26.9293445517329\\
688.118140369007	27.0361131009331\\
688.647869499391	27.14183474393\\
689.177598629775	27.2465249097196\\
689.707327760159	27.3502508005663\\
690.237056890543	27.4532782649348\\
690.766786020927	27.5556544967851\\
691.296515151312	27.6573867063939\\
691.826244281695	27.7584754797959\\
692.35597341208	27.8589342112693\\
692.885702542464	27.9587695734894\\
693.415431672848	28.0579837985896\\
693.945160803232	28.1565856189493\\
694.474889933616	28.2545835705713\\
695.004619064001	28.3519819805271\\
695.534348194385	28.4487846132573\\
696.064077324769	28.5450019202322\\
696.593806455153	28.6406398303238\\
697.123535585537	28.7356988019253\\
697.653264715921	28.8301921562881\\
698.182993846305	28.9241270622529\\
698.71272297669	29.0175057742561\\
699.242452107074	29.1103345680792\\
699.772181237458	29.2026211028776\\
700.301910367842	29.2943692513337\\
700.831639498226	29.3855808391855\\
701.36136862861	29.4762652640582\\
701.891097758994	29.566427365978\\
702.420826889378	29.6560672695582\\
702.950556019763	29.7451935867377\\
703.480285150147	29.8338117732531\\
704.010014280531	29.9219238106963\\
704.539743410915	30.0095341345493\\
705.069472541299	30.0966499110337\\
705.599201671683	30.1832748424788\\
706.128930802068	30.2694093967021\\
706.658659932452	30.3550623034561\\
707.188389062836	30.4402376599246\\
707.71811819322	30.5249356780378\\
708.247847323604	30.6091629670182\\
708.777576453988	30.6929245751231\\
709.307305584372	30.776222368096\\
709.837034714757	30.8590591596756\\
710.366763845141	30.9414420136353\\
710.896492975525	31.0233762350851\\
711.426222105909	31.1048616100239\\
711.955951236293	31.1859062810245\\
712.485680366677	31.2665136925522\\
713.015409497061	31.3466841540686\\
713.545138627446	31.4264225854315\\
714.07486775783	31.5057337690441\\
714.604596888214	31.5846195593669\\
715.134326018598	31.6630815404833\\
715.664055148982	31.7411259865305\\
716.193784279366	31.8187559864892\\
716.72351340975	31.8959704752087\\
717.253242540135	31.9727764193129\\
717.782971670519	32.0491770723201\\
718.312700800903	32.1251728628052\\
718.842429931287	32.2007673604619\\
719.372159061671	32.2759651873878\\
719.901888192055	32.3507682881961\\
720.43161732244	32.4251770721502\\
720.961346452824	32.4991974775342\\
721.491075583208	32.57283219027\\
722.020804713592	32.6460804801477\\
722.550533843976	32.7189478351387\\
723.08026297436	32.7914373898843\\
723.609992104744	32.8635498177624\\
724.139721235128	32.9352876704006\\
724.669450365513	33.0066554269463\\
725.199179495897	33.0776552108031\\
725.728908626281	33.1482874535604\\
726.258637756665	33.218559319293\\
726.788366887049	33.2884731365546\\
727.318096017433	33.3580284348669\\
727.847825147818	33.4272295380508\\
728.377554278202	33.496079512937\\
728.907283408586	33.5645791260616\\
729.43701253897	33.6327299564918\\
729.966741669354	33.7005362901241\\
730.496470799738	33.7680001284304\\
731.026199930122	33.8351203695715\\
731.555929060507	33.90190250306\\
732.085658190891	33.9683485851533\\
732.615387321275	34.0344583946344\\
733.145116451659	34.1002351206983\\
733.674845582043	34.1656819301298\\
734.204574712427	34.2307998432075\\
734.734303842811	34.2955894357825\\
735.264032973196	34.3600551118371\\
735.79376210358	34.4241988905497\\
736.323491233964	34.4880193486626\\
736.853220364348	34.5515215780128\\
737.382949494732	34.6147076421958\\
737.912678625116	34.6775773508142\\
738.4424077555	34.7401333097355\\
738.972136885885	34.8023786259662\\
739.501866016269	34.864314352189\\
740.031595146653	34.9259406072635\\
740.561324277037	34.9872615368766\\
741.091053407421	35.0482790044868\\
741.620782537805	35.1089919682312\\
742.150511668189	35.1694043476883\\
742.680240798574	35.2295186133676\\
743.209969928958	35.2893353844032\\
743.739699059342	35.3488568352442\\
744.269428189726	35.4080866345382\\
744.79915732011	35.4670258661028\\
745.328886450494	35.5256740620004\\
745.858615580879	35.5840352680501\\
746.388344711263	35.6421112099575\\
746.918073841647	35.6999010401585\\
747.447802972031	35.7574079020633\\
747.977532102415	35.814634126287\\
748.507261232799	35.8715798399491\\
749.036990363183	35.9282461996807\\
749.566719493568	35.9846362389669\\
750.096448623952	36.0407514579952\\
750.626177754336	36.0965905474084\\
751.15590688472	36.1521577514176\\
751.685636015104	36.207454550205\\
752.215365145488	36.2624802736687\\
752.745094275872	36.3172375424541\\
753.274823406257	36.3717284348759\\
753.804552536641	36.4259536648379\\
754.334281667025	36.479913344861\\
754.864010797409	36.5336109539599\\
755.393739927793	36.587047646884\\
755.923469058177	36.6402222799896\\
756.453198188561	36.6931385643384\\
756.982927318946	36.7457979854649\\
757.51265644933	36.7982002342176\\
758.042385579714	36.8503470486629\\
758.572114710098	36.9022408826422\\
759.101843840482	36.953882287676\\
759.631572970866	37.0052711871916\\
760.16130210125	37.0564107343078\\
760.691031231635	37.1073023211393\\
761.220760362019	37.1579447274263\\
761.750489492403	37.2083412624835\\
762.280218622787	37.258493293254\\
762.809947753171	37.3084009269245\\
763.339676883555	37.3580651535098\\
763.86940601394	37.407488683843\\
764.399135144324	37.456672937661\\
764.928864274708	37.5056177558336\\
765.458593405092	37.5543263070008\\
765.988322535476	37.6027998862656\\
766.51805166586	37.651037542864\\
767.047780796244	37.6990418555096\\
767.577509926629	37.7468143892439\\
768.107239057013	37.7943552626794\\
768.636968187397	37.8416651052814\\
769.166697317781	37.8887463642925\\
769.696426448165	37.9356000780166\\
770.226155578549	37.9822249397652\\
770.755884708933	38.0286244839212\\
771.285613839318	38.0747994973814\\
771.815342969702	38.1207496522119\\
772.345072100086	38.1664765568758\\
772.87480123047	38.2119822630025\\
773.404530360854	38.2572667794227\\
773.934259491238	38.3023305137124\\
774.463988621623	38.3471759066996\\
774.993717752007	38.3918040502963\\
775.523446882391	38.4362138052963\\
776.053176012775	38.4804081458248\\
776.582905143159	38.5243883974168\\
777.112634273543	38.5681539526941\\
777.642363403927	38.6117065297971\\
778.172092534311	38.6550477637558\\
778.701821664696	38.6981783426543\\
779.23155079508	38.7410979349093\\
779.761279925464	38.78380921325\\
780.291009055848	38.8263132887524\\
780.820738186232	38.8686089683562\\
781.350467316616	38.9106990932899\\
781.880196447001	38.9525846158392\\
782.409925577385	38.9942655664537\\
782.939654707769	39.0357428534533\\
783.469383838153	39.0770183598207\\
783.999112968537	39.1180929498199\\
784.528842098921	39.1589657235606\\
785.058571229305	39.1996397075267\\
785.58830035969	39.2401156900467\\
786.118029490074	39.2803929819848\\
786.647758620458	39.3204738844747\\
787.177487750842	39.36035953069\\
787.707216881226	39.4000503028101\\
788.23694601161	39.4395466419556\\
788.766675141994	39.4788507061494\\
789.296404272378	39.5179635891714\\
789.826133402763	39.5568843275332\\
790.355862533147	39.5956160532367\\
790.885591663531	39.634159830193\\
791.415320793915	39.6725153456508\\
791.945049924299	39.7106844980078\\
792.474779054683	39.7486684934528\\
793.004508185068	39.7864677356215\\
793.534237315452	39.8240825049641\\
794.063966445836	39.8615147318037\\
794.59369557622	39.8987655669071\\
795.123424706604	39.9358338942332\\
795.653153836988	39.972722303883\\
796.182882967372	40.0094319944012\\
796.712612097757	40.045962430858\\
797.242341228141	40.0823149844585\\
797.772070358525	40.1184914348605\\
798.301799488909	40.1544919215605\\
798.831528619293	40.1903165541859\\
799.361257749677	40.2259676383458\\
799.890986880062	40.2614458077153\\
800.420716010446	40.2967505960981\\
800.95044514083	40.3318840853763\\
801.480174271214	40.3668472306008\\
802.009903401598	40.4016402033876\\
802.539632531982	40.4362636196843\\
803.069361662366	40.4707191328542\\
803.599090792751	40.5050074856067\\
804.128819923135	40.5391278934297\\
804.658549053519	40.5730826292477\\
805.188278183903	40.6068725936376\\
805.718007314287	40.6404968111272\\
806.247736444671	40.673956963551\\
806.777465575055	40.7072542538733\\
807.30719470544	40.7403882766451\\
807.836923835824	40.7733593807228\\
808.366652966208	40.8061693217726\\
808.896382096592	40.8388182553249\\
809.426111226976	40.8713051890922\\
809.95584035736	40.9036324063207\\
810.485569487744	40.9358000657366\\
811.015298618129	40.9678073717015\\
811.545027748513	40.9996554927287\\
812.074756878897	41.0313450240476\\
812.604486009281	41.0628755275807\\
813.134215139665	41.0942467510608\\
813.663944270049	41.1254597936399\\
814.193673400434	41.1565144402495\\
814.723402530818	41.187409159796\\
815.253131661202	41.2181453434137\\
815.782860791586	41.2487224859475\\
816.31258992197	41.2791393439904\\
816.842319052354	41.3093958844793\\
817.372048182738	41.3394918736598\\
817.901777313122	41.3694262357711\\
818.431506443507	41.3991973440917\\
818.961235573891	41.4288052788381\\
819.490964704275	41.4582487145583\\
820.020693834659	41.4875248572236\\
820.550422965043	41.5166333797562\\
821.080152095427	41.5455723287412\\
821.609881225812	41.5743387163032\\
822.139610356196	41.6029301842052\\
822.66933948658	41.6313445089695\\
823.199068616964	41.6595780757776\\
823.728797747348	41.6876258293678\\
824.258526877732	41.7154851298992\\
824.788256008116	41.7431508586194\\
825.3179851385	41.7706156134821\\
825.847714268885	41.7978738672071\\
826.377443399269	41.8249180477709\\
826.907172529653	41.8517376913365\\
827.436901660037	41.8783212665414\\
827.966630790421	41.9046569696776\\
828.496359920805	41.9307284120093\\
829.02608905119	41.9565133802053\\
829.555818181574	41.9819911364011\\
830.085547311958	42.0071330541199\\
830.615276442342	42.0319006175448\\
831.145005572726	42.0562511156942\\
831.67473470311	42.0801308786193\\
832.204463833495	42.103466258137\\
832.734192963879	42.1261542107149\\
833.263922094263	42.1480513150268\\
833.793651224647	42.1689524631717\\
834.323380355031	42.1885674729615\\
834.853109485415	42.2061506902002\\
835.382838615799	42.2191774619805\\
835.912567746183	42.2142558905251\\
836.442296876568	42.1336999441421\\
836.972026006952	42.005161429506\\
837.501755137336	41.86515969537\\
838.03148426772	41.7199030567959\\
838.561213398104	41.5718047455203\\
839.090942528488	41.4216432377782\\
839.620671658873	41.2698127723379\\
840.150400789257	41.1165914181097\\
840.680129919641	40.96214008676\\
841.209859050025	40.8065583950457\\
841.739588180409	40.6499242942263\\
842.269317310793	40.4923011931657\\
842.799046441177	40.3337310580106\\
843.328775571561	40.1742391410236\\
843.858504701946	40.0138558730731\\
844.38823383233	39.8525953913004\\
844.917962962714	39.6904586927579\\
845.447692093098	39.5274550286435\\
845.977421223482	39.3635874808105\\
846.507150353866	39.1988515309688\\
847.036879484251	39.0332438579099\\
847.566608614635	38.8667664746611\\
848.096337745019	38.6994135008633\\
848.626066875403	38.531168993293\\
849.155796005787	38.3620325967151\\
849.685525136171	38.1919948796449\\
850.215254266555	38.0210397861931\\
850.74498339694	37.8491613170019\\
851.274712527324	37.6763515020658\\
851.804441657708	37.5025944418165\\
852.334170788092	37.327872357377\\
852.863899918476	37.152174825325\\
853.39362904886	36.9754682075496\\
853.923358179245	36.797603767793\\
854.453087309629	36.6180287570595\\
854.982816440013	36.4357512424343\\
855.512545570397	36.2506539970894\\
856.042274700781	36.0625769150342\\
856.572003831165	35.8715628927248\\
857.101732961549	35.6774410515679\\
857.631462091933	35.4802704794804\\
858.161191222318	35.2800121539884\\
858.690920352702	35.0766267617411\\
859.220649483086	34.870065390844\\
859.75037861347	34.6603738161008\\
860.280107743854	34.4476655068148\\
860.809836874238	34.2322047977703\\
861.339566004622	34.0145309082137\\
861.869295135007	33.7952419501821\\
862.399024265391	33.5744777602791\\
862.928753395775	33.3521246688202\\
863.458482526159	33.1280304653173\\
863.988211656543	32.9021387332749\\
864.517940786927	32.6744463733871\\
865.047669917312	32.4449711381079\\
865.577399047696	32.213684680941\\
866.10712817808	31.9805376857313\\
866.636857308464	31.7454819651095\\
867.166586438848	31.5084899077722\\
867.696315569232	31.2695281828003\\
868.226044699616	31.0285471300955\\
868.755773830001	30.7855323599788\\
869.285502960385	30.5404466839589\\
869.815232090769	30.293234619047\\
870.344961221153	30.0438631658562\\
870.874690351537	29.7922955393388\\
871.404419481921	29.5384768078947\\
871.934148612305	29.2823489736366\\
872.46387774269	29.0238752371117\\
872.993606873074	28.7630002311175\\
873.523336003458	28.4996394741692\\
874.053065133842	28.2337553612028\\
874.582794264226	27.9652838907949\\
875.11252339461	27.6941394243279\\
875.642252524994	27.4202553790302\\
876.171981655379	27.1435671152446\\
876.701710785763	26.8639869069256\\
877.231439916147	26.5814126317763\\
877.761169046531	26.2957771323844\\
878.290898176915	26.0069848103203\\
878.820627307299	25.7149030764906\\
879.350356437684	25.4194506889546\\
879.880085568068	25.1205202513068\\
880.409814698452	24.8179757682742\\
880.939543828836	24.5116984699378\\
881.46927295922	24.2015867634536\\
881.999002089604	23.8875138854417\\
882.528731219988	23.5693300157879\\
883.058460350373	23.2469545186459\\
883.588189480757	22.9202744310096\\
884.117918611141	22.5891430144236\\
884.647647741525	22.2534841828807\\
885.177376871909	21.9132175793498\\
885.707106002293	21.5682548979978\\
886.236835132677	21.2185612770179\\
886.766564263062	20.8641889702393\\
887.296293393446	20.5051990217265\\
887.82602252383	20.1416064966372\\
888.355751654214	19.7734298606221\\
888.885480784598	19.4005134547238\\
889.415209914982	19.0225346185108\\
889.944939045366	18.639010520929\\
890.474668175751	18.2489582580618\\
891.004397306135	17.851091440247\\
891.534126436519	17.4447376761295\\
892.063855566903	17.0300703094459\\
892.593584697287	16.6077662905229\\
893.123313827671	16.1790083435184\\
893.653042958055	15.7455411231337\\
894.18277208844	15.3091642538105\\
894.712501218824	14.871387725184\\
895.242230349208	14.4330563659971\\
895.771959479592	13.9940858876237\\
896.301688609976	13.5534026901737\\
896.83141774036	13.1090473950326\\
897.361146870745	12.6585362500628\\
897.890876001129	12.1989778342824\\
898.420605131513	11.7270562794117\\
898.950334261897	11.2394480615611\\
899.480063392281	10.732507114332\\
900.009792522665	10.2020510640108\\
900.539521653049	9.6434928720286\\
901.069250783434	9.05181437411615\\
901.598979913818	8.42150601416053\\
902.128709044202	7.74919953469042\\
902.658438174586	7.25419435672528\\
903.18816730497	8.11647870219658\\
903.717896435354	8.82661246868502\\
904.247625565738	9.47746053127126\\
904.777354696123	10.0785959547096\\
905.307083826507	10.6348944216444\\
905.836812956891	11.1520368889438\\
906.366542087275	11.6351843538604\\
906.896271217659	12.0885096808915\\
907.426000348043	12.5153424937048\\
907.955729478427	12.9191592134546\\
908.485458608812	13.3027393214853\\
909.015187739196	13.6685581822786\\
909.54491686958	14.0190410585649\\
910.074645999964	14.3562302496277\\
910.604375130348	14.6822351011392\\
911.134104260732	15.0063839690083\\
911.663833391116	15.3330059653995\\
912.193562521501	15.6615565879375\\
912.723291651885	15.9901702554146\\
913.253020782269	16.3155661478295\\
913.782749912653	16.6333728792318\\
914.312479043037	16.9385374891514\\
914.842208173421	17.2249338253579\\
915.371937303806	17.2124588762606\\
915.90166643419	16.8305264965197\\
916.431395564574	16.4497805902803\\
916.961124694958	16.0702883253712\\
917.490853825342	15.68938779799\\
918.020582955726	15.3047108227915\\
918.55031208611	14.9136071209226\\
919.080041216494	14.5133340134396\\
919.609770346879	14.1011439509177\\
920.139499477263	13.677356009989\\
920.669228607647	13.2450396513442\\
921.198957738031	12.8025472704869\\
921.728686868415	12.3451456063424\\
922.258415998799	11.8692791809235\\
922.788145129184	11.3720079396384\\
923.317874259568	10.8496399807044\\
923.847603389952	10.2981676360338\\
924.377332520336	9.7127166564681\\
924.90706165072	9.08708443891108\\
925.436790781104	8.41404033711548\\
925.966519911488	7.68768457642099\\
926.496249041872	6.90321231593377\\
927.025978172257	7.65918039520023\\
927.555707302641	7.88279954598954\\
928.085436433025	8.10561122310505\\
928.615165563409	8.56285588228016\\
929.144894693793	9.12854334963308\\
929.674623824177	9.711088900423\\
930.204352954562	10.2669998159046\\
930.734082084946	10.7844219241124\\
931.26381121533	11.2566754104482\\
931.793540345714	11.6371277432898\\
932.323269476098	11.7297619405961\\
932.852998606482	11.4917072541479\\
933.382727736867	11.0507695654237\\
933.912456867251	10.5242156634979\\
934.442185997635	9.96211820245439\\
934.971915128019	9.39123521908575\\
935.501644258403	8.88554505991131\\
936.031373388787	8.55964189743935\\
936.561102519171	8.5170899292342\\
937.090831649555	8.88736512738638\\
937.62056077994	9.54740547008346\\
938.150289910324	10.2044751537782\\
938.680019040708	10.8215552231876\\
939.209748171092	11.3932971990697\\
939.739477301476	11.9221335290409\\
940.26920643186	12.41147662258\\
940.798935562245	12.8647728132752\\
941.328664692629	13.2848741844147\\
941.858393823013	13.6748875266848\\
942.388122953397	14.037280662411\\
942.917852083781	14.3740278131423\\
943.447581214165	14.6879030806358\\
943.977310344549	14.9883750995948\\
944.507039474934	15.2794840936719\\
945.036768605318	15.5616860510179\\
945.566497735702	15.8354569897007\\
946.096226866086	16.1011821175181\\
946.62595599647	16.3591864845786\\
947.155685126854	16.6098799378741\\
947.685414257238	16.853553363447\\
948.215143387622	17.09043070614\\
948.744872518007	17.3207970920363\\
949.274601648391	17.5449105342917\\
949.804330778775	17.7629824279386\\
950.334059909159	17.9752053209866\\
950.863789039543	18.1818343990061\\
951.393518169927	18.3830997241084\\
951.923247300312	18.5792012497801\\
952.452976430696	18.7704464175821\\
952.98270556108	18.9570876231959\\
953.512434691464	19.1393152978653\\
954.042163821848	19.3173381077437\\
954.571892952232	19.4913510925592\\
955.101622082617	19.6615045805273\\
955.631351213001	19.8279261626855\\
956.161080343385	19.9907771001404\\
956.690809473769	20.1501898391749\\
957.220538604153	20.3062630791995\\
957.750267734537	20.4591352514808\\
958.279996864921	20.6089125110758\\
958.809725995305	20.7556760388521\\
959.33945512569	20.8995250355616\\
959.869184256074	21.0405777562521\\
960.398913386458	21.1792690178283\\
960.928642516842	21.3164367545809\\
961.458371647226	21.4521987454091\\
961.98810077761	21.586603172447\\
962.517829907995	21.7196803101721\\
963.047559038379	21.8514708597083\\
963.577288168763	21.9819590850494\\
964.107017299147	22.1111290406623\\
964.636746429531	22.2389778787274\\
965.166475559915	22.3655051737851\\
965.696204690299	22.4906897060291\\
966.225933820683	22.6144843818166\\
966.755662951068	22.7368449312771\\
967.285392081452	22.8576781736773\\
967.815121211836	22.9768200427032\\
968.34485034222	23.0940278256541\\
968.874579472604	23.2088717068791\\
969.404308602988	23.320474840975\\
969.934037733373	23.4269405663435\\
970.463766863757	23.5242061142887\\
970.993495994141	23.5888706951742\\
971.523225124525	23.463242597276\\
972.052954254909	23.2146393707425\\
972.582683385293	22.9471035155876\\
973.112412515678	22.6678896940243\\
973.642141646062	22.379854132296\\
974.171870776446	22.0840876707746\\
974.70159990683	21.7810012532942\\
975.231329037214	21.4707857565103\\
975.761058167598	21.1536493444571\\
976.290787297982	20.8297040505083\\
976.820516428366	20.4989313144396\\
977.350245558751	20.1613019723729\\
977.879974689135	19.8165617828838\\
978.409703819519	19.4642401869778\\
978.939432949903	19.1037984255385\\
979.469162080287	18.7346889618336\\
979.998891210671	18.3563031670002\\
980.528620341056	17.9679746467947\\
981.05834947144	17.5691707366088\\
981.588078601824	17.1592201937321\\
982.117807732208	16.7372538586525\\
982.647536862592	16.3024665363904\\
983.177265992976	15.8538807444235\\
983.70699512336	15.3902777757637\\
984.236724253745	14.910316137456\\
984.766453384129	14.4125855506256\\
985.296182514513	13.8953334380161\\
985.825911644897	13.3563238579723\\
986.355640775281	12.7933726938488\\
986.885369905665	12.2036933278999\\
987.415099036049	11.583777912965\\
987.944828166434	10.9300299654961\\
988.474557296818	10.238884772164\\
989.004286427202	9.50941419321698\\
989.534015557586	8.76173701274382\\
990.06374468797	8.1124655020126\\
990.593473818354	8.13470678645353\\
991.123202948738	8.176762949454\\
991.652932079123	8.31497130770331\\
992.182661209507	9.12919675187799\\
992.712390339891	9.86698799517029\\
993.242119470275	10.5336222416814\\
993.771848600659	11.1330960077437\\
994.301577731043	11.6581653068714\\
994.831306861427	11.7808048588335\\
995.361035991812	11.285693005617\\
995.890765122196	10.6657541146299\\
996.42049425258	9.98454479217064\\
996.950223382964	9.24191906844121\\
997.479952513348	8.47294654443001\\
998.009681643732	8.82214261130058\\
998.539410774116	9.65415449032053\\
999.069139904501	10.4085889849139\\
999.598869034885	11.0914702934962\\
1000.12859816527	11.7103921198429\\
1000.65832729565	12.2711418749739\\
1001.18805642604	12.7793139771956\\
1001.71778555642	13.240120070184\\
1002.24751468681	13.6591823895912\\
1002.77724381719	14.0412256023422\\
1003.30697294757	14.3901613494381\\
1003.83670207796	14.710476170169\\
1004.36643120834	15.0134721228839\\
1004.89616033873	15.3033984418946\\
1005.42588946911	15.5811792840849\\
1005.95561859949	15.8478623398272\\
1006.48534772988	16.1042834024047\\
1007.01507686026	16.3511555692322\\
1007.54480599065	16.5892245031529\\
1008.07453512103	16.8191439543225\\
1008.60426425142	17.0414632691354\\
1009.1339933818	17.2567010024972\\
1009.66372251218	17.465405237094\\
1010.19345164257	17.668055923509\\
1010.72318077295	17.8650222795626\\
1011.25290990334	18.0567558042296\\
1011.78263903372	18.2436174699458\\
1012.3123681641	18.4259145364086\\
1012.84209729449	18.6039764351741\\
1013.37182642487	18.7781287738314\\
1013.90155555526	18.948761218395\\
1014.43128468564	19.1161479275642\\
1014.96101381603	19.2805571707908\\
1015.49074294641	19.442220465661\\
1016.02047207679	19.6013374120403\\
1016.55020120718	19.7581449226961\\
1017.07993033756	19.9128509972529\\
1017.60965946795	20.0656460158918\\
1018.13938859833	20.2167447165654\\
1018.66911772871	20.3663597970185\\
1019.1988468591	20.514696046528\\
1019.72857598948	20.6619651082327\\
1020.25830511987	20.8084037561357\\
1020.78803425025	20.954238582658\\
1021.31776338063	21.0997491637674\\
1021.84749251102	21.2459301247992\\
1022.3772216414	21.3933949791132\\
1022.90695077179	21.5421704851053\\
1023.43667990217	21.6920948878328\\
1023.96640903256	21.8427915302234\\
1024.49613816294	21.9937038341109\\
1025.02586729332	22.1441312409915\\
1025.55559642371	22.2933416312206\\
1026.08532555409	22.4406639282182\\
1026.61505468448	22.5855403870092\\
1027.14478381486	22.7275976100453\\
1027.67451294524	22.8665786533509\\
1028.20424207563	23.0022361277053\\
1028.73397120601	23.1342163968376\\
1029.2637003364	23.2617855666376\\
1029.79342946678	23.3804834373093\\
};
\addlegendentry{v}

\end{axis}

\begin{axis}[%
width=5.833in,
height=4.375in,
at={(0in,0in)},
scale only axis,
xmin=0,
xmax=1,
ymin=0,
ymax=1,
axis line style={draw=none},
ticks=none,
axis x line*=bottom,
axis y line*=left,
legend style={legend cell align=left, align=left, draw=white!15!black}
]
\end{axis}
\end{tikzpicture}%} % inclusion of tex-code
			% \includegraphics[width=\columnwidth]{fig1} % inclusion of pdf
			\caption{Sollgeschwindigkeit $v_{Soll}$ über zurückgelegte Distanz entlang der Fahrbahnmittellinie $s$.}
			\label{fig1}
		\end{center}
	\end{figure}
	\item Querregelung
	Strukturell lässt sich die verwendete Querregelung in eine Vorsteuerung sowie einen Regler aufteilen. Die einzige Ausgangsgröße ist der Lenkwinkel $\delta$. Die Vorsteuerung basiert auf dem Lenkwinkel der Optimierungstrajektorie. Die Zuordnung erfolgt erneut über die zurückgelegte Distanz entlang der Fahrbahnmitte. Der Regler wird als klassischer LQR ausgelegt. Die korrespondierenden Zustände $x_{LQR}$ zeigt \ref{eq:stateslqr}.
	\begin{equation}
		x_{LQR} = \begin{bmatrix}
		v \\
		\dot{\psi} \\
		\beta \\
		n \\
		\xi 
		\end{bmatrix}
		\label{eq:stateslqr}
	\end{equation}		
	Die Linearisierung erfolgt dabei jeweils bei Geradeausfahrt auf der Fahrbahnmitte bei verschiedenen Geschwindigkeiten $v_i$. Die notwendige Jacobimatrix der Fahrzeug- und Streckenmodelldynamik liegt als Nebenprodukt des Optimierungsproblems bereits vor. Die Stellgrößengewichtung $R$ wird zu $1$ gewählt, die Zustandsgewichte $Q$ zeigt \ref{eq:q}.
	\begin{equation}
		Q = diag(\begin{bmatrix}
		0 & 0.2 & 0.2 & 50 & 10
		\end{bmatrix})
		\label{eq:q}
	\end{equation}	 
	Die sich ergebenden $k$ werden in Abhängigkeit der aktuellen Fahrzeuggeschwindigkeit $v$ linear interpoliert. \ref{eq:k} zeigt $k$ exemplarisch für $v = 15 \frac{\text{m}}{\text{s}}$.
	\begin{equation}
		k_{15} = \begin{bmatrix}
		0 & 0.164 & -7.197 & 7.0711 & 10.991	
		\end{bmatrix}
		\label{eq:k}
	\end{equation}	 
\end{enumerate}

\section{Ergebnis}
Mit Hilfe der dargestellten Methode konnte eine Rundenzeit von $t_{\text{f}}=53.263 \text{s}$ unter Einhaltung der Streckenbegrenzung erreicht werden. Die benötigte Berechnungszeit beträgt $t_{\text{sim}}=316.297755 \text{s}$. \ref{fig2} zeigt die gefahrene Linie, \ref{fig3} die Fahrzeuggeschwindigkeit. 
\begin{figure}[h]
	\begin{center}
		\scalebox{0.8}{% This file was created by matlab2tikz.
%
%The latest updates can be retrieved from
%  http://www.mathworks.com/matlabcentral/fileexchange/22022-matlab2tikz-matlab2tikz
%where you can also make suggestions and rate matlab2tikz.
%
\definecolor{mycolor1}{rgb}{0.00000,0.44700,0.74100}%
\definecolor{mycolor2}{rgb}{0.85000,0.32500,0.09800}%
%
\begin{tikzpicture}

\begin{axis}[%
width=3.493in,
height=8.778in,
at={(0.647in,1.185in)},
scale only axis,
unbounded coords=jump,
xmin=-77.0352050453383,
xmax=102.035205045338,
ymin=-20,
ymax=429.999621219295,
axis background/.style={fill=white},
axis x line*=bottom,
axis y line*=left,
xmajorgrids,
ymajorgrids,
legend style={legend cell align=left, align=left, draw=white!15!black}
]
\addplot [color=mycolor1]
  table[row sep=crcr]{%
0	0\\
0	1.00401606425703\\
0	2.00803212851406\\
0	3.01204819277108\\
0	4.01606425702811\\
0	5.02008032128514\\
0	6.02409638554217\\
0	7.0281124497992\\
0	8.03212851405623\\
0	9.03614457831325\\
0	10.0401606425703\\
0	11.0441767068273\\
0	12.0481927710843\\
0	13.0522088353414\\
0	14.0562248995984\\
0	15.0602409638554\\
0	16.0642570281125\\
0	17.0682730923695\\
0	18.0722891566265\\
0	19.0763052208835\\
0	20.0803212851406\\
0	21.0843373493976\\
0	22.0883534136546\\
0	23.0923694779116\\
0	24.0963855421687\\
0	25.1004016064257\\
0	26.1044176706827\\
0	27.1084337349398\\
0	28.1124497991968\\
0	29.1164658634538\\
0	30.1204819277108\\
0	31.1244979919679\\
0	32.1285140562249\\
0	33.1325301204819\\
0	34.136546184739\\
0	35.140562248996\\
0	36.144578313253\\
0	37.14859437751\\
0	38.1526104417671\\
0	39.1566265060241\\
0	40.1606425702811\\
0	41.1646586345381\\
0	42.1686746987952\\
0	43.1726907630522\\
0	44.1767068273092\\
0	45.1807228915663\\
0	46.1847389558233\\
0	47.1887550200803\\
0	48.1927710843374\\
0	49.1967871485944\\
0	50.2008032128514\\
0	51.2048192771084\\
0	52.2088353413655\\
0	53.2128514056225\\
0	54.2168674698795\\
0	55.2208835341365\\
0	56.2248995983936\\
0	57.2289156626506\\
0	58.2329317269076\\
0	59.2369477911647\\
0	60.2409638554217\\
0	61.2449799196787\\
0	62.2489959839357\\
0	63.2530120481928\\
0	64.2570281124498\\
0	65.2610441767068\\
0	66.2650602409639\\
0	67.2690763052209\\
0	68.2730923694779\\
0	69.2771084337349\\
0	70.281124497992\\
0	71.285140562249\\
0	72.289156626506\\
0	73.2931726907631\\
0	74.2971887550201\\
0	75.3012048192771\\
0	76.3052208835341\\
0	77.3092369477912\\
0	78.3132530120482\\
0	79.3172690763052\\
0	80.3212851405623\\
0	81.3253012048193\\
0	82.3293172690763\\
0	83.3333333333333\\
0	84.3373493975904\\
0	85.3413654618474\\
0	86.3453815261044\\
0	87.3493975903614\\
0	88.3534136546185\\
0	89.3574297188755\\
0	90.3614457831325\\
0	91.3654618473896\\
0	92.3694779116466\\
0	93.3734939759036\\
0	94.3775100401606\\
0	95.3815261044177\\
0	96.3855421686747\\
0	97.3895582329317\\
0	98.3935742971887\\
0	99.3975903614458\\
0	100.401606425703\\
0	101.40562248996\\
0	102.409638554217\\
0	103.413654618474\\
0	104.417670682731\\
0	105.421686746988\\
0	106.425702811245\\
0	107.429718875502\\
0	108.433734939759\\
0	109.437751004016\\
0	110.441767068273\\
0	111.44578313253\\
0	112.449799196787\\
0	113.453815261044\\
0	114.457831325301\\
0	115.461847389558\\
0	116.465863453815\\
0	117.469879518072\\
0	118.473895582329\\
0	119.477911646586\\
0	120.481927710843\\
0	121.4859437751\\
0	122.489959839357\\
0	123.493975903614\\
0	124.497991967871\\
0	125.502008032129\\
0	126.506024096386\\
0	127.510040160643\\
0	128.5140562249\\
0	129.518072289157\\
0	130.522088353414\\
0	131.526104417671\\
0	132.530120481928\\
0	133.534136546185\\
0	134.538152610442\\
0	135.542168674699\\
0	136.546184738956\\
0	137.550200803213\\
0	138.55421686747\\
0	139.558232931727\\
0	140.562248995984\\
0	141.566265060241\\
0	142.570281124498\\
0	143.574297188755\\
0	144.578313253012\\
0	145.582329317269\\
0	146.586345381526\\
0	147.590361445783\\
0	148.59437751004\\
0	149.598393574297\\
0	150.602409638554\\
0	151.606425702811\\
0	152.610441767068\\
0	153.614457831325\\
0	154.618473895582\\
0	155.622489959839\\
0	156.626506024096\\
0	157.630522088353\\
0	158.63453815261\\
0	159.638554216867\\
0	160.642570281125\\
0	161.646586345382\\
0	162.650602409639\\
0	163.654618473896\\
0	164.658634538153\\
0	165.66265060241\\
0	166.666666666667\\
0	167.670682730924\\
0	168.674698795181\\
0	169.678714859438\\
0	170.682730923695\\
0	171.686746987952\\
0	172.690763052209\\
0	173.694779116466\\
0	174.698795180723\\
0	175.70281124498\\
0	176.706827309237\\
0	177.710843373494\\
0	178.714859437751\\
0	179.718875502008\\
0	180.722891566265\\
0	181.726907630522\\
0	182.730923694779\\
0	183.734939759036\\
0	184.738955823293\\
0	185.74297188755\\
0	186.746987951807\\
0	187.751004016064\\
0	188.755020080321\\
0	189.759036144578\\
0	190.763052208835\\
0	191.767068273092\\
0	192.771084337349\\
0	193.775100401606\\
0	194.779116465863\\
0	195.78313253012\\
0	196.787148594377\\
0	197.791164658635\\
0	198.795180722892\\
0	199.799196787149\\
0	200.803212851406\\
0	201.807228915663\\
0	202.81124497992\\
0	203.815261044177\\
0	204.819277108434\\
0	205.823293172691\\
0	206.827309236948\\
0	207.831325301205\\
0	208.835341365462\\
0	209.839357429719\\
0	210.843373493976\\
0	211.847389558233\\
0	212.85140562249\\
0	213.855421686747\\
0	214.859437751004\\
0	215.863453815261\\
0	216.867469879518\\
0	217.871485943775\\
0	218.875502008032\\
0	219.879518072289\\
0	220.883534136546\\
0	221.887550200803\\
0	222.89156626506\\
0	223.895582329317\\
0	224.899598393574\\
0	225.903614457831\\
0	226.907630522088\\
0	227.911646586345\\
0	228.915662650602\\
0	229.919678714859\\
0	230.923694779116\\
0	231.927710843373\\
0	232.931726907631\\
0	233.935742971888\\
0	234.939759036145\\
0	235.943775100402\\
0	236.947791164659\\
0	237.951807228916\\
0	238.955823293173\\
0	239.95983935743\\
0	240.963855421687\\
0	241.967871485944\\
0	242.971887550201\\
0	243.975903614458\\
0	244.979919678715\\
0	245.983935742972\\
0	246.987951807229\\
0	247.991967871486\\
0	248.995983935743\\
0	250\\
0	250\\
-0.202020202020202	251.999897966992\\
-0.404040404040404	252.813815813573\\
-0.606060606060606	253.428396514844\\
-0.808080808080808	253.938098725175\\
-1.01010101010101	254.379693613875\\
-1.21212121212121	254.77212598425\\
-1.41414141414141	255.126697996141\\
-1.61616161616162	255.450803055851\\
-1.81818181818182	255.749595745761\\
-2.02020202020202	256.026841975829\\
-2.22222222222222	256.285393610547\\
-2.42424242424242	256.527472493496\\
-2.62626262626263	256.754849897899\\
-2.82828282828283	256.968964974004\\
-3.03030303030303	257.171005797697\\
-3.23232323232323	257.361966528601\\
-3.43434343434343	257.542688768725\\
-3.63636363636364	257.713892158399\\
-3.83838383838384	257.876197450351\\
-4.04040404040404	258.030144207819\\
-4.24242424242424	258.176204583777\\
-4.44444444444444	258.314794192831\\
-4.64646464646465	258.446280792067\\
-4.84848484848485	258.57099128711\\
-5.05050505050505	258.689217441457\\
-5.25252525252525	258.801220570017\\
-5.45454545454545	258.907235428302\\
-5.65656565656566	259.007473458427\\
-5.85858585858586	259.102125516015\\
-6.06060606060606	259.191364174608\\
-6.26262626262626	259.275345683434\\
-6.46464646464646	259.354211638618\\
-6.66666666666667	259.428090415821\\
-6.86868686868687	259.497098402863\\
-7.07070707070707	259.561341063595\\
-7.27272727272727	259.620913858417\\
-7.47474747474747	259.675903042285\\
-7.67676767676768	259.726386357342\\
-7.87878787878788	259.772433634301\\
-8.08080808080808	259.814107314336\\
-8.28282828282828	259.851462901201\\
-8.48484848484848	259.884549351697\\
-8.68686868686869	259.91340941122\\
-8.88888888888889	259.938079899999\\
-9.09090909090909	259.958591954639\\
-9.29292929292929	259.974971228791\\
-9.49494949494949	259.987238056007\\
-9.6969696969697	259.995407577255\\
-9.8989898989899	259.999489834961\\
-10.1010101010101	259.999489834961\\
-10.3030303030303	259.995407577255\\
-10.5050505050505	259.987238056007\\
-10.7070707070707	259.974971228791\\
-10.9090909090909	259.958591954639\\
-11.1111111111111	259.938079899999\\
-11.3131313131313	259.91340941122\\
-11.5151515151515	259.884549351697\\
-11.7171717171717	259.851462901201\\
-11.9191919191919	259.814107314336\\
-12.1212121212121	259.772433634301\\
-12.3232323232323	259.726386357342\\
-12.5252525252525	259.675903042285\\
-12.7272727272727	259.620913858417\\
-12.9292929292929	259.561341063595\\
-13.1313131313131	259.497098402863\\
-13.3333333333333	259.428090415821\\
-13.5353535353535	259.354211638618\\
-13.7373737373737	259.275345683434\\
-13.9393939393939	259.191364174608\\
-14.1414141414141	259.102125516015\\
-14.3434343434343	259.007473458427\\
-14.5454545454545	258.907235428302\\
-14.7474747474747	258.801220570017\\
-14.9494949494949	258.689217441457\\
-15.1515151515152	258.57099128711\\
-15.3535353535354	258.446280792067\\
-15.5555555555556	258.314794192831\\
-15.7575757575758	258.176204583777\\
-15.959595959596	258.030144207819\\
-16.1616161616162	257.876197450351\\
-16.3636363636364	257.713892158399\\
-16.5656565656566	257.542688768725\\
-16.7676767676768	257.361966528601\\
-16.969696969697	257.171005797697\\
-17.1717171717172	256.968964974004\\
-17.3737373737374	256.754849897899\\
-17.5757575757576	256.527472493496\\
-17.7777777777778	256.285393610547\\
-17.979797979798	256.026841975829\\
-18.1818181818182	255.749595745761\\
-18.3838383838384	255.450803055851\\
-18.5858585858586	255.126697996141\\
-18.7878787878788	254.77212598425\\
-18.989898989899	254.379693613875\\
-19.1919191919192	253.938098725175\\
-19.3939393939394	253.428396514844\\
-19.5959595959596	252.813815813573\\
-19.7979797979798	251.999897966992\\
-20	250\\
-20	250\\
-20.1010101010101	249.000051016504\\
-20.2020202020202	248.593092093214\\
-20.3030303030303	248.285801742578\\
-20.4040404040404	248.030950637412\\
-20.5050505050505	247.810153193063\\
-20.6060606060606	247.613937007875\\
-20.7070707070707	247.436651001929\\
-20.8080808080808	247.274598472074\\
-20.9090909090909	247.12520212712\\
-21.010101010101	246.986579012086\\
-21.1111111111111	246.857303194726\\
-21.2121212121212	246.736263753252\\
-21.3131313131313	246.62257505105\\
-21.4141414141414	246.515517512998\\
-21.5151515151515	246.414497101152\\
-21.6161616161616	246.319016735699\\
-21.7171717171717	246.228655615637\\
-21.8181818181818	246.143053920801\\
-21.9191919191919	246.061901274825\\
-22.020202020202	245.984927896091\\
-22.1212121212121	245.911897708111\\
-22.2222222222222	245.842602903585\\
-22.3232323232323	245.776859603966\\
-22.4242424242424	245.714504356445\\
-22.5252525252525	245.655391279271\\
-22.6262626262626	245.599389714992\\
-22.7272727272727	245.546382285849\\
-22.8282828282828	245.496263270786\\
-22.9292929292929	245.448937241992\\
-23.030303030303	245.404317912696\\
-23.1313131313131	245.362327158283\\
-23.2323232323232	245.322894180691\\
-23.3333333333333	245.28595479209\\
-23.4343434343434	245.251450798568\\
-23.5353535353535	245.219329468202\\
-23.6363636363636	245.189543070792\\
-23.7373737373737	245.162048478857\\
-23.8383838383838	245.136806821329\\
-23.9393939393939	245.113783182849\\
-24.040404040404	245.092946342832\\
-24.1414141414141	245.074268549399\\
-24.2424242424242	245.057725324151\\
-24.3434343434343	245.04329529439\\
-24.4444444444444	245.03096005\\
-24.5454545454545	245.02070402268\\
-24.6464646464646	245.012514385605\\
-24.7474747474747	245.006380971997\\
-24.8484848484848	245.002296211373\\
-24.9494949494949	245.000255082519\\
-25.0505050505051	245.000255082519\\
-25.1515151515152	245.002296211373\\
-25.2525252525253	245.006380971997\\
-25.3535353535354	245.012514385605\\
-25.4545454545455	245.02070402268\\
-25.5555555555556	245.03096005\\
-25.6565656565657	245.04329529439\\
-25.7575757575758	245.057725324151\\
-25.8585858585859	245.074268549399\\
-25.959595959596	245.092946342832\\
-26.0606060606061	245.113783182849\\
-26.1616161616162	245.136806821329\\
-26.2626262626263	245.162048478857\\
-26.3636363636364	245.189543070792\\
-26.4646464646465	245.219329468202\\
-26.5656565656566	245.251450798568\\
-26.6666666666667	245.28595479209\\
-26.7676767676768	245.322894180691\\
-26.8686868686869	245.362327158283\\
-26.969696969697	245.404317912696\\
-27.0707070707071	245.448937241992\\
-27.1717171717172	245.496263270786\\
-27.2727272727273	245.546382285849\\
-27.3737373737374	245.599389714992\\
-27.4747474747475	245.655391279271\\
-27.5757575757576	245.714504356445\\
-27.6767676767677	245.776859603966\\
-27.7777777777778	245.842602903585\\
-27.8787878787879	245.911897708111\\
-27.979797979798	245.984927896091\\
-28.0808080808081	246.061901274825\\
-28.1818181818182	246.143053920801\\
-28.2828282828283	246.228655615637\\
-28.3838383838384	246.319016735699\\
-28.4848484848485	246.414497101152\\
-28.5858585858586	246.515517512998\\
-28.6868686868687	246.62257505105\\
-28.7878787878788	246.736263753252\\
-28.8888888888889	246.857303194726\\
-28.989898989899	246.986579012086\\
-29.0909090909091	247.12520212712\\
-29.1919191919192	247.274598472074\\
-29.2929292929293	247.436651001929\\
-29.3939393939394	247.613937007875\\
-29.4949494949495	247.810153193063\\
-29.5959595959596	248.030950637412\\
-29.6969696969697	248.285801742578\\
-29.7979797979798	248.593092093214\\
-29.8989898989899	249.000051016504\\
-30	250\\
-30	250\\
-30	251.006711409396\\
-30	252.013422818792\\
-30	253.020134228188\\
-30	254.026845637584\\
-30	255.03355704698\\
-30	256.040268456376\\
-30	257.046979865772\\
-30	258.053691275168\\
-30	259.060402684564\\
-30	260.06711409396\\
-30	261.073825503356\\
-30	262.080536912752\\
-30	263.087248322148\\
-30	264.093959731544\\
-30	265.10067114094\\
-30	266.107382550336\\
-30	267.114093959732\\
-30	268.120805369128\\
-30	269.127516778523\\
-30	270.134228187919\\
-30	271.140939597315\\
-30	272.147651006711\\
-30	273.154362416107\\
-30	274.161073825503\\
-30	275.167785234899\\
-30	276.174496644295\\
-30	277.181208053691\\
-30	278.187919463087\\
-30	279.194630872483\\
-30	280.201342281879\\
-30	281.208053691275\\
-30	282.214765100671\\
-30	283.221476510067\\
-30	284.228187919463\\
-30	285.234899328859\\
-30	286.241610738255\\
-30	287.248322147651\\
-30	288.255033557047\\
-30	289.261744966443\\
-30	290.268456375839\\
-30	291.275167785235\\
-30	292.281879194631\\
-30	293.288590604027\\
-30	294.295302013423\\
-30	295.302013422819\\
-30	296.308724832215\\
-30	297.315436241611\\
-30	298.322147651007\\
-30	299.328859060403\\
-30	300.335570469799\\
-30	301.342281879195\\
-30	302.348993288591\\
-30	303.355704697987\\
-30	304.362416107383\\
-30	305.369127516779\\
-30	306.375838926174\\
-30	307.38255033557\\
-30	308.389261744966\\
-30	309.395973154362\\
-30	310.402684563758\\
-30	311.409395973154\\
-30	312.41610738255\\
-30	313.422818791946\\
-30	314.429530201342\\
-30	315.436241610738\\
-30	316.442953020134\\
-30	317.44966442953\\
-30	318.456375838926\\
-30	319.463087248322\\
-30	320.469798657718\\
-30	321.476510067114\\
-30	322.48322147651\\
-30	323.489932885906\\
-30	324.496644295302\\
-30	325.503355704698\\
-30	326.510067114094\\
-30	327.51677852349\\
-30	328.523489932886\\
-30	329.530201342282\\
-30	330.536912751678\\
-30	331.543624161074\\
-30	332.55033557047\\
-30	333.557046979866\\
-30	334.563758389262\\
-30	335.570469798658\\
-30	336.577181208054\\
-30	337.58389261745\\
-30	338.590604026846\\
-30	339.597315436242\\
-30	340.604026845638\\
-30	341.610738255034\\
-30	342.61744966443\\
-30	343.624161073826\\
-30	344.630872483221\\
-30	345.637583892617\\
-30	346.644295302013\\
-30	347.651006711409\\
-30	348.657718120805\\
-30	349.664429530201\\
-30	350.671140939597\\
-30	351.677852348993\\
-30	352.684563758389\\
-30	353.691275167785\\
-30	354.697986577181\\
-30	355.704697986577\\
-30	356.711409395973\\
-30	357.718120805369\\
-30	358.724832214765\\
-30	359.731543624161\\
-30	360.738255033557\\
-30	361.744966442953\\
-30	362.751677852349\\
-30	363.758389261745\\
-30	364.765100671141\\
-30	365.771812080537\\
-30	366.778523489933\\
-30	367.785234899329\\
-30	368.791946308725\\
-30	369.798657718121\\
-30	370.805369127517\\
-30	371.812080536913\\
-30	372.818791946309\\
-30	373.825503355705\\
-30	374.832214765101\\
-30	375.838926174497\\
-30	376.845637583893\\
-30	377.852348993289\\
-30	378.859060402685\\
-30	379.865771812081\\
-30	380.872483221477\\
-30	381.879194630872\\
-30	382.885906040268\\
-30	383.892617449664\\
-30	384.89932885906\\
-30	385.906040268456\\
-30	386.912751677852\\
-30	387.919463087248\\
-30	388.926174496644\\
-30	389.93288590604\\
-30	390.939597315436\\
-30	391.946308724832\\
-30	392.953020134228\\
-30	393.959731543624\\
-30	394.96644295302\\
-30	395.973154362416\\
-30	396.979865771812\\
-30	397.986577181208\\
-30	398.993288590604\\
-30	400\\
-30	400\\
-29.748743718593	403.535489266199\\
-29.4974874371859	404.987294784241\\
-29.2462311557789	406.092641031649\\
-28.9949748743719	407.017206052145\\
-28.7437185929648	407.825332413265\\
-28.4924623115578	408.550100264911\\
-28.2412060301508	409.211207427047\\
-27.9899497487437	409.821517731504\\
-27.7386934673367	410.38998649655\\
-27.4874371859296	410.923148301148\\
-27.2361809045226	411.425946734575\\
-26.9849246231156	411.9022304345\\
-26.7336683417085	412.355066184065\\
-26.4824120603015	412.786945376884\\
-26.2311557788945	413.199925162376\\
-25.9798994974874	413.595727897958\\
-25.7286432160804	413.975813050426\\
-25.4773869346734	414.341430344553\\
-25.2261306532663	414.693659816295\\
-24.9748743718593	415.033442514223\\
-24.7236180904523	415.361604389581\\
-24.4723618090452	415.67887513751\\
-24.2211055276382	415.985903236771\\
-23.9698492462312	416.283268086452\\
-23.7185929648241	416.571489897328\\
-23.4673366834171	416.851037826238\\
-23.21608040201	417.122336721006\\
-22.964824120603	417.385772755785\\
-22.713567839196	417.641698172404\\
-22.4623115577889	417.890435295395\\
-22.2110552763819	418.132279952431\\
-21.9597989949749	418.36750440452\\
-21.7085427135678	418.59635986931\\
-21.4572864321608	418.819078704597\\
-21.2060301507538	419.035876306408\\
-20.9547738693467	419.246952766036\\
-20.7035175879397	419.452494322458\\
-20.4522613065327	419.652674640227\\
-20.2010050251256	419.847655937821\\
-19.9497487437186	420.037589987313\\
-19.6984924623116	420.222619002873\\
-19.4472361809045	420.402876432826\\
-19.1959798994975	420.578487667782\\
-18.9447236180904	420.74957067544\\
-18.6934673366834	420.916236571123\\
-18.4422110552764	421.078590131823\\
-18.1909547738693	421.236730260418\\
-17.9396984924623	421.390750405822\\
-17.6884422110553	421.540738944073\\
-17.4371859296482	421.686779524663\\
-17.1859296482412	421.828951385903\\
-16.9346733668342	421.967329642607\\
-16.6834170854271	422.101985548994\\
-16.4321608040201	422.232986739326\\
-16.1809045226131	422.36039744853\\
-15.929648241206	422.484278714771\\
-15.678391959799	422.604688565714\\
-15.427135678392	422.721682190023\\
-15.1758793969849	422.835312095481\\
-14.9246231155779	422.945628254936\\
-14.6733668341709	423.052678241184\\
-14.4221105527638	423.156507351747\\
-14.1708542713568	423.257158724434\\
-13.9195979899498	423.354673444467\\
-13.6683417085427	423.44909064386\\
-13.4170854271357	423.540447593712\\
-13.1658291457286	423.628779789967\\
-12.9145728643216	423.71412103316\\
-12.6633165829146	423.796503502616\\
-12.4120603015075	423.875957825537\\
-12.1608040201005	423.952513141333\\
-11.9095477386935	424.026197161572\\
-11.6582914572864	424.097036225848\\
-11.4070351758794	424.165055353858\\
-11.1557788944724	424.230278293952\\
-10.9045226130653	424.292727568386\\
-10.6532663316583	424.352424515504\\
-10.4020100502513	424.409389329047\\
-10.1507537688442	424.46364109475\\
-9.89949748743718	424.515197824423\\
-9.64824120603015	424.564076487639\\
-9.39698492462312	424.610293041178\\
-9.14572864321608	424.653862456354\\
-8.89447236180904	424.694798744333\\
-8.64321608040201	424.733114979547\\
-8.39195979899498	424.768823321305\\
-8.14070351758794	424.801935033675\\
-7.8894472361809	424.83246050373\\
-7.63819095477387	424.860409258219\\
-7.38693467336683	424.885789978722\\
-7.1356783919598	424.908610515364\\
-6.88442211055276	424.928877899121\\
-6.63316582914573	424.946598352772\\
-6.38190954773869	424.961777300542\\
-6.13065326633166	424.974419376461\\
-5.87939698492462	424.984528431469\\
-5.62814070351759	424.992107539313\\
-5.37688442211055	424.997159001222\\
-5.12562814070352	424.999684349413\\
-4.87437185929648	424.999684349413\\
-4.62311557788945	424.997159001222\\
-4.37185929648241	424.992107539313\\
-4.12060301507538	424.984528431469\\
-3.86934673366834	424.974419376461\\
-3.61809045226131	424.961777300542\\
-3.36683417085427	424.946598352772\\
-3.11557788944724	424.928877899121\\
-2.8643216080402	424.908610515364\\
-2.61306532663317	424.885789978722\\
-2.36180904522613	424.860409258219\\
-2.1105527638191	424.83246050373\\
-1.85929648241206	424.801935033675\\
-1.60804020100502	424.768823321305\\
-1.35678391959799	424.733114979547\\
-1.10552763819096	424.694798744333\\
-0.854271356783919	424.653862456354\\
-0.603015075376884	424.610293041178\\
-0.35175879396985	424.564076487639\\
-0.100502512562816	424.515197824423\\
0.150753768844222	424.46364109475\\
0.402010050251256	424.409389329047\\
0.653266331658291	424.352424515504\\
0.904522613065328	424.292727568386\\
1.15577889447236	424.230278293952\\
1.4070351758794	424.165055353858\\
1.65829145728643	424.097036225848\\
1.90954773869347	424.026197161572\\
2.1608040201005	423.952513141333\\
2.41206030150754	423.875957825537\\
2.66331658291458	423.796503502616\\
2.91457286432161	423.71412103316\\
3.16582914572864	423.628779789967\\
3.41708542713568	423.540447593712\\
3.66834170854271	423.44909064386\\
3.91959798994975	423.354673444467\\
4.17085427135678	423.257158724434\\
4.42211055276382	423.156507351747\\
4.67336683417086	423.052678241184\\
4.92462311557789	422.945628254936\\
5.17587939698493	422.835312095481\\
5.42713567839196	422.721682190023\\
5.67839195979899	422.604688565714\\
5.92964824120603	422.484278714771\\
6.18090452261306	422.36039744853\\
6.4321608040201	422.232986739326\\
6.68341708542714	422.101985548994\\
6.93467336683417	421.967329642607\\
7.18592964824121	421.828951385903\\
7.43718592964824	421.686779524663\\
7.68844221105527	421.540738944073\\
7.93969849246231	421.390750405822\\
8.19095477386934	421.236730260418\\
8.44221105527638	421.078590131823\\
8.69346733668342	420.916236571123\\
8.94472361809045	420.74957067544\\
9.19597989949749	420.578487667782\\
9.44723618090453	420.402876432826\\
9.69849246231156	420.222619002873\\
9.94974874371859	420.037589987313\\
10.2010050251256	419.847655937821\\
10.4522613065327	419.652674640227\\
10.7035175879397	419.452494322458\\
10.9547738693467	419.246952766036\\
11.2060301507538	419.035876306408\\
11.4572864321608	418.819078704597\\
11.7085427135678	418.59635986931\\
11.9597989949749	418.36750440452\\
12.2110552763819	418.132279952431\\
12.4623115577889	417.890435295395\\
12.713567839196	417.641698172404\\
12.964824120603	417.385772755785\\
13.21608040201	417.122336721006\\
13.4673366834171	416.851037826238\\
13.7185929648241	416.571489897328\\
13.9698492462312	416.283268086452\\
14.2211055276382	415.985903236771\\
14.4723618090452	415.67887513751\\
14.7236180904523	415.361604389581\\
14.9748743718593	415.033442514223\\
15.2261306532663	414.693659816295\\
15.4773869346734	414.341430344553\\
15.7286432160804	413.975813050426\\
15.9798994974874	413.595727897958\\
16.2311557788945	413.199925162376\\
16.4824120603015	412.786945376884\\
16.7336683417085	412.355066184065\\
16.9849246231156	411.9022304345\\
17.2361809045226	411.425946734575\\
17.4874371859297	410.923148301148\\
17.7386934673367	410.38998649655\\
17.9899497487437	409.821517731504\\
18.2412060301508	409.211207427047\\
18.4924623115578	408.550100264911\\
18.7437185929648	407.825332413265\\
18.9949748743719	407.017206052145\\
19.2462311557789	406.092641031649\\
19.4974874371859	404.987294784241\\
19.748743718593	403.535489266199\\
20	400\\
20	400\\
20	398.989898989899\\
20	397.979797979798\\
20	396.969696969697\\
20	395.959595959596\\
20	394.949494949495\\
20	393.939393939394\\
20	392.929292929293\\
20	391.919191919192\\
20	390.909090909091\\
20	389.89898989899\\
20	388.888888888889\\
20	387.878787878788\\
20	386.868686868687\\
20	385.858585858586\\
20	384.848484848485\\
20	383.838383838384\\
20	382.828282828283\\
20	381.818181818182\\
20	380.808080808081\\
20	379.79797979798\\
20	378.787878787879\\
20	377.777777777778\\
20	376.767676767677\\
20	375.757575757576\\
20	374.747474747475\\
20	373.737373737374\\
20	372.727272727273\\
20	371.717171717172\\
20	370.707070707071\\
20	369.69696969697\\
20	368.686868686869\\
20	367.676767676768\\
20	366.666666666667\\
20	365.656565656566\\
20	364.646464646465\\
20	363.636363636364\\
20	362.626262626263\\
20	361.616161616162\\
20	360.606060606061\\
20	359.59595959596\\
20	358.585858585859\\
20	357.575757575758\\
20	356.565656565657\\
20	355.555555555556\\
20	354.545454545455\\
20	353.535353535354\\
20	352.525252525253\\
20	351.515151515152\\
20	350.505050505051\\
20	349.494949494949\\
20	348.484848484848\\
20	347.474747474747\\
20	346.464646464646\\
20	345.454545454545\\
20	344.444444444444\\
20	343.434343434343\\
20	342.424242424242\\
20	341.414141414141\\
20	340.40404040404\\
20	339.393939393939\\
20	338.383838383838\\
20	337.373737373737\\
20	336.363636363636\\
20	335.353535353535\\
20	334.343434343434\\
20	333.333333333333\\
20	332.323232323232\\
20	331.313131313131\\
20	330.30303030303\\
20	329.292929292929\\
20	328.282828282828\\
20	327.272727272727\\
20	326.262626262626\\
20	325.252525252525\\
20	324.242424242424\\
20	323.232323232323\\
20	322.222222222222\\
20	321.212121212121\\
20	320.20202020202\\
20	319.191919191919\\
20	318.181818181818\\
20	317.171717171717\\
20	316.161616161616\\
20	315.151515151515\\
20	314.141414141414\\
20	313.131313131313\\
20	312.121212121212\\
20	311.111111111111\\
20	310.10101010101\\
20	309.090909090909\\
20	308.080808080808\\
20	307.070707070707\\
20	306.060606060606\\
20	305.050505050505\\
20	304.040404040404\\
20	303.030303030303\\
20	302.020202020202\\
20	301.010101010101\\
20	300\\
20	300\\
20.1020408163265	298.995014510021\\
20.2040816326531	298.586080973413\\
20.3061224489796	298.277352753456\\
20.4081632653061	298.021355160238\\
20.5102040816327	297.799606259914\\
20.6122448979592	297.602583648067\\
20.7142857142857	297.424606231811\\
20.8163265306122	297.261957578572\\
20.9183673469388	297.112046592228\\
21.0204081632653	296.972980209349\\
21.1224489795918	296.84332483709\\
21.2244897959184	296.721963590978\\
21.3265306122449	296.608006094164\\
21.4285714285714	296.500728938881\\
21.530612244898	296.399534946045\\
21.6326530612245	296.303924433413\\
21.734693877551	296.213474425453\\
21.8367346938776	296.127823273263\\
21.9387755102041	296.046659055549\\
22.0408163265306	295.969710682014\\
22.1428571428571	295.896740966759\\
22.2448979591837	295.827541163212\\
22.3469387755102	295.761926600552\\
22.4489795918367	295.699733162101\\
22.5510204081633	295.640814415654\\
22.6530612244898	295.585039254534\\
22.7551020408163	295.532289943063\\
22.8571428571429	295.482460485474\\
22.9591836734694	295.435455255848\\
23.0612244897959	295.391187840557\\
23.1632653061224	295.349580055059\\
23.265306122449	295.310561104867\\
23.3673469387755	295.274066866568\\
23.469387755102	295.240039269514\\
23.5714285714286	295.2084257625\\
23.6734693877551	295.179178852646\\
23.7755102040816	295.152255706033\\
23.8775510204082	295.127617801505\\
23.9795918367347	295.105230630526\\
24.0816326530612	295.085063437228\\
24.1836734693878	295.067088993767\\
24.2857142857143	295.051283406946\\
24.3877551020408	295.03762595274\\
24.4897959183673	295.026098935937\\
24.5918367346939	295.016687572623\\
24.6938775510204	295.009379893617\\
24.7959183673469	295.00416666739\\
24.8979591836735	295.001041341259\\
25	295\\
25	295\\
25.1020408163265	294.998958658741\\
25.2040816326531	294.99583333261\\
25.3061224489796	294.990620106383\\
25.4081632653061	294.983312427377\\
25.5102040816327	294.973901064063\\
25.6122448979592	294.96237404726\\
25.7142857142857	294.948716593054\\
25.8163265306122	294.932911006233\\
25.9183673469388	294.914936562772\\
26.0204081632653	294.894769369474\\
26.1224489795918	294.872382198495\\
26.2244897959184	294.847744293967\\
26.3265306122449	294.820821147354\\
26.4285714285714	294.7915742375\\
26.530612244898	294.759960730486\\
26.6326530612245	294.725933133432\\
26.734693877551	294.689438895133\\
26.8367346938776	294.650419944941\\
26.9387755102041	294.608812159443\\
27.0408163265306	294.564544744152\\
27.1428571428571	294.517539514526\\
27.2448979591837	294.467710056937\\
27.3469387755102	294.414960745466\\
27.4489795918367	294.359185584346\\
27.5510204081633	294.300266837899\\
27.6530612244898	294.238073399448\\
27.7551020408163	294.172458836788\\
27.8571428571429	294.103259033241\\
27.9591836734694	294.030289317986\\
28.0612244897959	293.953340944451\\
28.1632653061224	293.872176726737\\
28.265306122449	293.786525574547\\
28.3673469387755	293.696075566587\\
28.469387755102	293.600465053955\\
28.5714285714286	293.499271061119\\
28.6734693877551	293.391993905836\\
28.7755102040816	293.278036409022\\
28.8775510204082	293.15667516291\\
28.9795918367347	293.027019790651\\
29.0816326530612	292.887953407772\\
29.1836734693878	292.738042421428\\
29.2857142857143	292.575393768189\\
29.3877551020408	292.397416351933\\
29.4897959183673	292.200393740086\\
29.5918367346939	291.978644839762\\
29.6938775510204	291.722647246544\\
29.7959183673469	291.413919026587\\
29.8979591836735	291.004985489979\\
30	290\\
30	290\\
30	288.995901639344\\
30	287.991803278689\\
30	286.987704918033\\
30	285.983606557377\\
30	284.979508196721\\
30	283.975409836066\\
30	282.97131147541\\
30	281.967213114754\\
30	280.963114754098\\
30	279.959016393443\\
30	278.954918032787\\
30	277.950819672131\\
30	276.946721311475\\
30	275.94262295082\\
30	274.938524590164\\
30	273.934426229508\\
30	272.930327868852\\
30	271.926229508197\\
30	270.922131147541\\
30	269.918032786885\\
30	268.91393442623\\
30	267.909836065574\\
30	266.905737704918\\
30	265.901639344262\\
30	264.897540983607\\
30	263.893442622951\\
30	262.889344262295\\
30	261.885245901639\\
30	260.881147540984\\
30	259.877049180328\\
30	258.872950819672\\
30	257.868852459016\\
30	256.864754098361\\
30	255.860655737705\\
30	254.856557377049\\
30	253.852459016393\\
30	252.848360655738\\
30	251.844262295082\\
30	250.840163934426\\
30	249.83606557377\\
30	248.831967213115\\
30	247.827868852459\\
30	246.823770491803\\
30	245.819672131148\\
30	244.815573770492\\
30	243.811475409836\\
30	242.80737704918\\
30	241.803278688525\\
30	240.799180327869\\
30	239.795081967213\\
30	238.790983606557\\
30	237.786885245902\\
30	236.782786885246\\
30	235.77868852459\\
30	234.774590163934\\
30	233.770491803279\\
30	232.766393442623\\
30	231.762295081967\\
30	230.758196721311\\
30	229.754098360656\\
30	228.75\\
30	227.745901639344\\
30	226.741803278689\\
30	225.737704918033\\
30	224.733606557377\\
30	223.729508196721\\
30	222.725409836066\\
30	221.72131147541\\
30	220.717213114754\\
30	219.713114754098\\
30	218.709016393443\\
30	217.704918032787\\
30	216.700819672131\\
30	215.696721311475\\
30	214.69262295082\\
30	213.688524590164\\
30	212.684426229508\\
30	211.680327868852\\
30	210.676229508197\\
30	209.672131147541\\
30	208.668032786885\\
30	207.66393442623\\
30	206.659836065574\\
30	205.655737704918\\
30	204.651639344262\\
30	203.647540983607\\
30	202.643442622951\\
30	201.639344262295\\
30	200.635245901639\\
30	199.631147540984\\
30	198.627049180328\\
30	197.622950819672\\
30	196.618852459016\\
30	195.614754098361\\
30	194.610655737705\\
30	193.606557377049\\
30	192.602459016393\\
30	191.598360655738\\
30	190.594262295082\\
30	189.590163934426\\
30	188.58606557377\\
30	187.581967213115\\
30	186.577868852459\\
30	185.573770491803\\
30	184.569672131148\\
30	183.565573770492\\
30	182.561475409836\\
30	181.55737704918\\
30	180.553278688525\\
30	179.549180327869\\
30	178.545081967213\\
30	177.540983606557\\
30	176.536885245902\\
30	175.532786885246\\
30	174.52868852459\\
30	173.524590163934\\
30	172.520491803279\\
30	171.516393442623\\
30	170.512295081967\\
30	169.508196721311\\
30	168.504098360656\\
30	167.5\\
30	166.495901639344\\
30	165.491803278689\\
30	164.487704918033\\
30	163.483606557377\\
30	162.479508196721\\
30	161.475409836066\\
30	160.47131147541\\
30	159.467213114754\\
30	158.463114754098\\
30	157.459016393443\\
30	156.454918032787\\
30	155.450819672131\\
30	154.446721311475\\
30	153.44262295082\\
30	152.438524590164\\
30	151.434426229508\\
30	150.430327868852\\
30	149.426229508197\\
30	148.422131147541\\
30	147.418032786885\\
30	146.41393442623\\
30	145.409836065574\\
30	144.405737704918\\
30	143.401639344262\\
30	142.397540983607\\
30	141.393442622951\\
30	140.389344262295\\
30	139.385245901639\\
30	138.381147540984\\
30	137.377049180328\\
30	136.372950819672\\
30	135.368852459016\\
30	134.364754098361\\
30	133.360655737705\\
30	132.356557377049\\
30	131.352459016393\\
30	130.348360655738\\
30	129.344262295082\\
30	128.340163934426\\
30	127.33606557377\\
30	126.331967213115\\
30	125.327868852459\\
30	124.323770491803\\
30	123.319672131148\\
30	122.315573770492\\
30	121.311475409836\\
30	120.30737704918\\
30	119.303278688525\\
30	118.299180327869\\
30	117.295081967213\\
30	116.290983606557\\
30	115.286885245902\\
30	114.282786885246\\
30	113.27868852459\\
30	112.274590163934\\
30	111.270491803279\\
30	110.266393442623\\
30	109.262295081967\\
30	108.258196721311\\
30	107.254098360656\\
30	106.25\\
30	105.245901639344\\
30	104.241803278689\\
30	103.237704918033\\
30	102.233606557377\\
30	101.229508196721\\
30	100.225409836066\\
30	99.2213114754098\\
30	98.2172131147541\\
30	97.2131147540984\\
30	96.2090163934426\\
30	95.2049180327869\\
30	94.2008196721312\\
30	93.1967213114754\\
30	92.1926229508197\\
30	91.1885245901639\\
30	90.1844262295082\\
30	89.1803278688525\\
30	88.1762295081967\\
30	87.172131147541\\
30	86.1680327868852\\
30	85.1639344262295\\
30	84.1598360655738\\
30	83.155737704918\\
30	82.1516393442623\\
30	81.1475409836065\\
30	80.1434426229508\\
30	79.1393442622951\\
30	78.1352459016393\\
30	77.1311475409836\\
30	76.1270491803279\\
30	75.1229508196721\\
30	74.1188524590164\\
30	73.1147540983607\\
30	72.1106557377049\\
30	71.1065573770492\\
30	70.1024590163935\\
30	69.0983606557377\\
30	68.094262295082\\
30	67.0901639344262\\
30	66.0860655737705\\
30	65.0819672131148\\
30	64.077868852459\\
30	63.0737704918033\\
30	62.0696721311475\\
30	61.0655737704918\\
30	60.0614754098361\\
30	59.0573770491803\\
30	58.0532786885246\\
30	57.0491803278688\\
30	56.0450819672131\\
30	55.0409836065574\\
30	54.0368852459016\\
30	53.0327868852459\\
30	52.0286885245902\\
30	51.0245901639344\\
30	50.0204918032787\\
30	49.016393442623\\
30	48.0122950819672\\
30	47.0081967213115\\
30	46.0040983606557\\
30	45\\
30	45\\
30.1006711409396	43.2650649725221\\
30.2013422818792	42.550565424041\\
30.3020134228188	42.0051393059107\\
30.4026845637584	41.5476990202385\\
30.503355704698	41.1467800250326\\
30.6040268456376	40.7862182141363\\
30.7046979865772	40.4563956658805\\
30.8053691275168	40.1510357400836\\
30.9060402684564	39.8657718120805\\
31.006711409396	39.5974196516784\\
31.1073825503356	39.3435717632697\\
31.2080536912752	39.1023549600496\\
31.3087248322148	38.8722773985772\\
31.4093959731544	38.6521277434492\\
31.510067114094	38.440906256673\\
31.6107382550336	38.237776258894\\
31.7114093959732	38.0420290487406\\
31.8120805369128	37.8530579801481\\
31.9127516778523	37.6703389331809\\
32.0134228187919	37.4934153494065\\
32.1140939597315	37.321886591006\\
32.2147651006711	37.1553987629081\\
32.3154362416107	36.9936373889663\\
32.4161073825503	36.8363215036054\\
32.5167785234899	36.6831988379908\\
32.6174496644295	36.5340418624314\\
32.7181208053691	36.3886445057417\\
32.8187919463087	36.2468194150534\\
32.9194630872483	36.1083956509725\\
33.0201342281879	35.9732167363249\\
33.1208053691275	35.8411389942935\\
33.2214765100671	35.7120301250973\\
33.3221476510067	35.5857679805992\\
33.4228187919463	35.4622395041662\\
33.5234899328859	35.3413398092992\\
33.6241610738255	35.2229713754263\\
33.7248322147651	35.1070433431253\\
33.8255033557047	34.9934708941295\\
33.9261744966443	34.8821747039579\\
34.0268456375839	34.7730804570199\\
34.1275167785235	34.6661184156818\\
34.2281879194631	34.5612230361216\\
34.3288590604027	34.4583326248999\\
34.4295302013423	34.3573890310852\\
34.5302013422819	34.2583373695296\\
34.6308724832215	34.161125771521\\
34.7315436241611	34.0657051595682\\
34.8322147651007	33.9720290435201\\
34.9328859060403	33.8800533355957\\
35.0335570469799	33.7897361822214\\
35.1342281879195	33.7010378108449\\
35.2348993288591	33.6139203901219\\
35.3355704697987	33.5283479020774\\
35.4362416107383	33.444286025008\\
35.5369127516779	33.3617020260417\\
35.6375838926174	33.2805646623985\\
35.738255033557	33.2008440905051\\
35.8389261744966	33.1225117822127\\
35.9395973154362	33.0455404474502\\
36.0402684563758	32.9699039627197\\
36.1409395973154	32.8955773049027\\
36.241610738255	32.8225364899027\\
36.3422818791946	32.7507585157007\\
36.4429530201342	32.6802213094403\\
36.5436241610738	32.6109036782005\\
36.6442953020134	32.5427852631479\\
36.744966442953	32.4758464967876\\
36.8456375838926	32.410068563063\\
36.9463087248322	32.3454333600756\\
37.0469798657718	32.2819234652185\\
37.1476510067114	32.2195221025347\\
37.248322147651	32.1582131121326\\
37.3489932885906	32.0979809215004\\
37.4496644295302	32.038810518579\\
37.5503355704698	31.9806874264655\\
37.6510067114094	31.9235976796272\\
37.751677852349	31.8675278015192\\
37.8523489932886	31.8124647835065\\
37.9530201342282	31.758396065\\
38.0536912751678	31.7053095147228\\
38.1543624161074	31.6531934130296\\
38.255033557047	31.6020364352101\\
38.3557046979866	31.5518276357104\\
38.4563758389262	31.5025564332126\\
38.5570469798658	31.4542125965176\\
38.6577181208054	31.4067862311799\\
38.758389261745	31.3602677668475\\
38.8590604026846	31.3146479452619\\
38.9597315436242	31.26991780888\\
39.0604026845638	31.2260686900775\\
39.1610738255034	31.1830922009019\\
39.261744966443	31.1409802233398\\
39.3624161073826	31.0997249000712\\
39.4630872483221	31.0593186256802\\
39.5637583892617	31.0197540382983\\
39.6644295302013	30.9810240116543\\
39.7651006711409	30.9431216475097\\
39.8657718120805	30.9060402684564\\
39.9664429530201	30.8697734110596\\
40.0671140939597	30.8343148193252\\
40.1677852348993	30.7996584384761\\
40.2684563758389	30.7657984090199\\
40.3691275167785	30.7327290610943\\
40.4697986577181	30.7004449090754\\
40.5704697986577	30.6689406464352\\
40.6711409395973	30.6382111408373\\
40.7718120805369	30.6082514294577\\
40.8724832214765	30.5790567145208\\
40.9731543624161	30.5506223590401\\
41.0738255033557	30.5229438827537\\
41.1744966442953	30.4960169582458\\
41.2751677852349	30.4698374072467\\
41.3758389261745	30.4444011971005\\
41.4765100671141	30.4197044373973\\
41.5771812080537	30.3957433767583\\
41.6778523489933	30.3725143997709\\
41.7785234899329	30.3500140240652\\
41.8791946308725	30.3282388975275\\
41.9798657718121	30.3071857956439\\
42.0805369127517	30.2868516189704\\
42.1812080536913	30.2672333907231\\
42.2818791946309	30.2483282544852\\
42.3825503355705	30.2301334720256\\
42.4832214765101	30.2126464212253\\
42.5838926174497	30.195864594108\\
42.6845637583893	30.1797855949708\\
42.7852348993289	30.1644071386126\\
42.8859060402685	30.1497270486558\\
42.9865771812081	30.135743255959\\
43.0872483221477	30.1224537971185\\
43.1879194630872	30.1098568130544\\
43.2885906040268	30.0979505476805\\
43.3892617449664	30.086733346655\\
43.489932885906	30.0762036562096\\
43.5906040268456	30.0663600220557\\
43.6912751677852	30.0572010883655\\
43.7919463087248	30.0487255968263\\
43.8926174496644	30.0409323857664\\
43.993288590604	30.0338203893515\\
44.0939597315436	30.0273886368498\\
44.1946308724832	30.0216362519652\\
44.2953020134228	30.016562452237\\
44.3959731543624	30.0121665485051\\
44.496644295302	30.0084479444406\\
44.5973154362416	30.0054061361399\\
44.6979865771812	30.003040711783\\
44.7986577181208	30.0013513513541\\
44.8993288590604	30.0003378264248\\
45	30\\
45	30\\
45.0671140939597	29.9997747823834\\
45.1342281879195	29.9990990990973\\
45.2013422818792	29.9979728588113\\
45.2684563758389	29.9963959092401\\
45.3355704697987	29.9943680370396\\
45.4026845637584	29.9918889676633\\
45.4697986577181	29.9889583651753\\
45.5369127516779	29.9855758320232\\
45.6040268456376	29.9817409087668\\
45.6711409395973	29.9774530737657\\
45.738255033557	29.9727117428224\\
45.8053691275168	29.9675162687825\\
45.8724832214765	29.9618659410896\\
45.9395973154362	29.9557599852962\\
46.006711409396	29.9491975625269\\
46.0738255033557	29.9421777688967\\
46.1409395973154	29.9346996348796\\
46.2080536912752	29.9267621246304\\
46.2751677852349	29.9183641352543\\
46.3422818791946	29.9095044960273\\
46.4093959731544	29.9001819675628\\
46.4765100671141	29.8903952409249\\
46.5436241610738	29.8801429366861\\
46.6107382550336	29.869423603928\\
46.6778523489933	29.8582357191831\\
46.744966442953	29.8465776853162\\
46.8120805369128	29.8344478303432\\
46.8791946308725	29.8218444061846\\
46.9463087248322	29.8087655873531\\
47.0134228187919	29.7952094695707\\
47.0805369127517	29.781174068315\\
47.1476510067114	29.7666573172899\\
47.2147651006711	29.7516570668194\\
47.2818791946309	29.7361710821611\\
47.3489932885906	29.7201970417351\\
47.4161073825503	29.7037325352663\\
47.4832214765101	29.6867750618356\\
47.5503355704698	29.6693220278361\\
47.6174496644295	29.6513707448309\\
47.6845637583893	29.6329184273066\\
47.751677852349	29.6139621903195\\
47.8187919463087	29.5944990470282\\
47.8859060402685	29.5745259061085\\
47.9530201342282	29.5540395690432\\
48.0201342281879	29.5330367272831\\
48.0872483221477	29.5115139592704\\
48.1543624161074	29.4894677273201\\
48.2214765100671	29.4668943743493\\
48.2885906040268	29.4437901204499\\
48.3557046979866	29.4201510592936\\
48.4228187919463	29.3959731543624\\
48.489932885906	29.3712522349935\\
48.5570469798658	29.3459839922304\\
48.6241610738255	29.3201639744678\\
48.6912751677852	29.2937875828799\\
48.758389261745	29.2668500666192\\
48.8255033557047	29.2393465177734\\
48.8926174496644	29.2112718660654\\
48.9597315436242	29.1826208732816\\
49.0268456375839	29.1533881274133\\
49.0939597315436	29.123568036492\\
49.1610738255034	29.0931548221017\\
49.2281879194631	29.0621425125467\\
49.2953020134228	29.030524935655\\
49.3624161073826	28.9982957111916\\
49.4295302013423	28.9654482428597\\
49.496644295302	28.9319757098599\\
49.5637583892617	28.8978710579803\\
49.6308724832215	28.8631269901848\\
49.6979865771812	28.8277359566666\\
49.7651006711409	28.791690144329\\
49.8322147651007	28.7549814656539\\
49.8993288590604	28.7176015469152\\
49.9664429530201	28.6795417156896\\
50.0335570469799	28.640792987614\\
50.1006711409396	28.6013460523331\\
50.1677852348993	28.5611912585782\\
50.2348993288591	28.5203185983102\\
50.3020134228188	28.4787176898544\\
50.3691275167785	28.4363777599496\\
50.4362416107383	28.3932876246247\\
50.503355704698	28.3494356688083\\
50.5704697986577	28.3048098245681\\
50.6375838926174	28.2593975478663\\
50.7046979865772	28.2131857937065\\
50.7718120805369	28.1661609895329\\
50.8389261744966	28.1183090067315\\
50.9060402684564	28.0696151300649\\
50.9731543624161	28.0200640248535\\
51.0402684563758	27.9696397016999\\
51.1073825503356	27.9183254785249\\
51.1744966442953	27.8661039396633\\
51.241610738255	27.8129568917344\\
51.3087248322148	27.7588653159722\\
51.3758389261745	27.7038093166613\\
51.4429530201342	27.6477680652817\\
51.510067114094	27.5907197399188\\
51.5771812080537	27.5326414594368\\
51.6442953020134	27.4735092118524\\
51.7114093959732	27.4132977762696\\
51.7785234899329	27.3519806376532\\
51.8456375838926	27.2895298936212\\
51.9127516778523	27.2259161523193\\
51.9798657718121	27.1611084203136\\
52.0469798657718	27.0950739792765\\
52.1140939597315	27.0277782500667\\
52.1812080536913	26.9591846425856\\
52.248322147651	26.8892543895455\\
52.3154362416107	26.8179463619867\\
52.3825503355705	26.745216864028\\
52.4496644295302	26.6710194039137\\
52.5167785234899	26.5953044379165\\
52.5838926174497	26.5180190830491\\
52.6510067114094	26.4391067938006\\
52.7181208053691	26.3585069972226\\
52.7852348993289	26.2761546796006\\
52.8523489932886	26.1919799166018\\
52.9194630872483	26.1059073371376\\
52.9865771812081	26.0178555091167\\
53.0536912751678	25.927736232685\\
53.1208053691275	25.8354537232978\\
53.1879194630872	25.7409036628389\\
53.255033557047	25.6439720917124\\
53.3221476510067	25.5445341080061\\
53.3892617449664	25.4424523309297\\
53.4563758389262	25.3375750740224\\
53.5234899328859	25.2297341580612\\
53.5906040268456	25.1187422726626\\
53.6577181208054	25.0043897670623\\
53.7248322147651	24.8864407112127\\
53.7919463087248	24.7646280132346\\
53.8590604026846	24.6386473008396\\
53.9261744966443	24.5081491607374\\
53.993288590604	24.372729162218\\
54.0604026845638	24.2319148377005\\
54.1275167785235	24.0851484009485\\
54.1946308724832	23.9317633599669\\
54.261744966443	23.7709521578202\\
54.3288590604027	23.6017202322144\\
54.3959731543624	23.4228187919463\\
54.4630872483221	23.2326428399442\\
54.5302013422819	23.0290695560797\\
54.5973154362416	22.8091878572425\\
54.6644295302013	22.5688133166449\\
54.7315436241611	22.3015339865077\\
54.7986577181208	21.9965737960595\\
54.8657718120805	21.6329563839727\\
54.9328859060403	21.1566233516519\\
55	20\\
55	20\\
54.859437751004	16.8663759504567\\
54.718875502008	15.5728469784672\\
54.578313253012	14.583335659599\\
54.4377510040161	13.7516957672189\\
54.2971887550201	13.0212577420767\\
54.1566265060241	12.3629281967516\\
54.0160642570281	11.7594070321527\\
53.8755020080321	11.1994111728924\\
53.7349397590361	10.6750957402125\\
53.5943775100402	10.1807434236136\\
53.4538152610442	9.71203399694926\\
53.3132530120482	9.26560790005965\\
53.1726907630522	8.83879094636649\\
53.0321285140562	8.42941290033762\\
52.8915662650602	8.03568352846052\\
52.7510040160643	7.65610531721277\\
52.6104417670683	7.28941040877139\\
52.4698795180723	6.93451401279426\\
52.3293172690763	6.59047931820811\\
52.1887550200803	6.25649061348398\\
52.0481927710843	5.93183238273664\\
51.9076305220884	5.61587282919483\\
51.7670682730924	5.30805073066585\\
51.6265060240964	5.00786483826286\\
51.4859437751004	4.71486524131317\\
51.3453815261044	4.42864627005886\\
51.2048192771084	4.14884061391717\\
51.0642570281125	3.87511440997592\\
50.9236947791165	3.60716311287015\\
50.7831325301205	3.34470799916621\\
50.6425702811245	3.08749319094459\\
50.5020080321285	2.83528310725844\\
50.3614457831325	2.58786027054963\\
50.2208835341365	2.34502340935694\\
50.0803212851406	2.10658580978131\\
49.9397590361446	1.87237387693581\\
49.7991967871486	1.64222587455677\\
49.6586345381526	1.415990816503\\
49.5180722891566	1.19352748833201\\
49.3775100401606	0.974703580751743\\
49.2369477911647	0.759394919683221\\
49.0963855421687	0.547484780073358\\
48.9558232931727	0.338863272573537\\
48.8152610441767	0.133426793834012\\
48.6746987951807	-0.0689224674784441\\
48.5341365461847	-0.268276975699841\\
48.3935742971888	-0.464724250036255\\
48.2530120481928	-0.658347228291859\\
48.1124497991968	-0.849224596795931\\
47.9718875502008	-1.03743109029525\\
47.8313253012048	-1.22303776509188\\
47.6907630522088	-1.40611224829081\\
47.5502008032129	-1.58671896566613\\
47.4096385542169	-1.76491935034881\\
47.2690763052209	-1.94077203427539\\
47.1285140562249	-2.11433302410923\\
46.9879518072289	-2.2856558631486\\
46.8473895582329	-2.45479178056402\\
46.7068273092369	-2.62178982915798\\
46.566265060241	-2.78669701270941\\
46.425702811245	-2.9495584038508\\
46.285140562249	-3.11041725332566\\
46.144578313253	-3.26931509138548\\
46.004016064257	-3.42629182200765\\
45.863453815261	-3.58138581054682\\
45.7228915662651	-3.73463396537157\\
45.5823293172691	-3.88607181398388\\
45.4417670682731	-4.03573357407122\\
45.3012048192771	-4.18365221989824\\
45.1606425702811	-4.32985954440669\\
45.0200803212851	-4.47438621735875\\
44.8795180722892	-4.6172618398277\\
44.7389558232932	-4.7585149953132\\
44.5983935742972	-4.89817329773361\\
44.4578313253012	-5.03626343652573\\
44.3172690763052	-5.17281121906272\\
44.1767068273092	-5.30784161058265\\
44.0361445783133	-5.44137877180445\\
43.8955823293173	-5.57344609439291\\
43.7550200803213	-5.7040662344216\\
43.6144578313253	-5.83326114397017\\
43.4738955823293	-5.96105210098197\\
43.3333333333333	-6.08745973749754\\
43.1927710843374	-6.21250406637117\\
43.0522088353414	-6.3362045065687\\
42.9116465863454	-6.45857990713802\\
42.7710843373494	-6.57964856993638\\
42.6305220883534	-6.6994282711924\\
42.4899598393574	-6.81793628197535\\
42.3493975903614	-6.93518938763836\\
42.2088353413655	-7.05120390629808\\
42.0682730923695	-7.16599570640843\\
41.9277108433735	-7.27958022348218\\
41.7871485943775	-7.39197247601045\\
41.6465863453815	-7.50318708062653\\
41.5060240963855	-7.61323826655766\\
41.3654618473896	-7.72213988940503\\
41.2248995983936	-7.82990544428982\\
41.0843373493976	-7.93654807840077\\
40.9437751004016	-8.0420806029759\\
40.8032128514056	-8.14651550474964\\
40.6626506024096	-8.24986495689387\\
40.5220883534137	-8.3521408294803\\
40.3815261044177	-8.45335469948914\\
40.2409638554217	-8.55351786038829\\
40.1004016064257	-8.65264133130485\\
39.9598393574297	-8.75073586581046\\
39.8192771084337	-8.84781196033963\\
39.6787148594377	-8.94387986226007\\
39.5381526104418	-9.03894957761192\\
39.3975903614458	-9.1330308785328\\
39.2570281124498	-9.22613331038367\\
39.1164658634538	-9.31826619859034\\
38.9759036144578	-9.40943865521418\\
38.8353413654618	-9.49965958526508\\
38.6947791164659	-9.58893769276876\\
38.5542168674699	-9.67728148660009\\
38.4136546184739	-9.76469928609316\\
38.2730923694779	-9.8511992264386\\
38.1325301204819	-9.93678926387757\\
37.9919678714859	-10.0214771807021\\
37.85140562249	-10.10527059007\\
37.710843373494	-10.1881769406429\\
37.570281124498	-10.2702035210553\\
37.429718875502	-10.3513574642217\\
37.289156626506	-10.4316457514894\\
37.14859437751	-10.5110752166427\\
37.0080321285141	-10.5896525497665\\
36.8674698795181	-10.6673843009728\\
36.7269076305221	-10.7442768839988\\
36.5863453815261	-10.8203365796795\\
36.4457831325301	-10.8955695393011\\
36.3052208835341	-10.9699817878403\\
36.1646586345382	-11.0435792270938\\
36.0240963855422	-11.1163676387019\\
35.8835341365462	-11.1883526870717\\
35.7429718875502	-11.259539922203\\
35.6024096385542	-11.329934782421\\
35.4618473895582	-11.399542597019\\
35.3212851405623	-11.4683685888161\\
35.1807228915663	-11.5364178766308\\
35.0401606425703	-11.603695477676\\
34.8995983935743	-11.6702063098774\\
34.7590361445783	-11.7359551941175\\
34.6184738955823	-11.8009468564094\\
34.4779116465863	-11.8651859300027\\
34.3373493975904	-11.9286769574221\\
34.1967871485944	-11.9914243924447\\
34.0562248995984	-12.0534326020149\\
33.9156626506024	-12.1147058681008\\
33.7751004016064	-12.1752483894944\\
33.6345381526104	-12.2350642835563\\
33.4939759036145	-12.2941575879086\\
33.3534136546185	-12.3525322620757\\
33.2128514056225	-12.4101921890775\\
33.0722891566265	-12.4671411769738\\
32.9317269076305	-12.5233829603635\\
32.7911646586345	-12.57892120184\\
32.6506024096386	-12.6337594934026\\
32.5100401606426	-12.6879013578283\\
32.3694779116466	-12.7413502500017\\
32.2289156626506	-12.7941095582084\\
32.0883534136546	-12.8461826053894\\
31.9477911646586	-12.8975726503599\\
31.8072289156627	-12.9482828889935\\
31.6666666666667	-12.9983164553722\\
31.5261044176707	-13.0476764229038\\
31.3855421686747	-13.0963658054072\\
31.2449799196787	-13.1443875581677\\
31.1044176706827	-13.1917445789616\\
30.9638554216867	-13.238439709053\\
30.8232931726908	-13.2844757341615\\
30.6827309236948	-13.3298553854039\\
30.5421686746988	-13.3745813402086\\
30.4016064257028	-13.4186562232053\\
30.2610441767068	-13.4620826070893\\
30.1204819277108	-13.5048630134624\\
29.9799196787149	-13.5469999136495\\
29.8393574297189	-13.588495729494\\
29.6987951807229	-13.6293528341297\\
29.5582329317269	-13.6695735527323\\
29.4176706827309	-13.7091601632498\\
29.2771084337349	-13.7481148971128\\
29.136546184739	-13.7864399399246\\
28.995983935743	-13.8241374321335\\
28.855421686747	-13.8612094696851\\
28.714859437751	-13.8976581046574\\
28.574297188755	-13.9334853458779\\
28.433734939759	-13.9686931595239\\
28.2931726907631	-14.003283469706\\
28.1526104417671	-14.0372581590349\\
28.0120481927711	-14.0706190691733\\
27.8714859437751	-14.1033680013712\\
27.7309236947791	-14.1355067169876\\
27.5903614457831	-14.1670369379958\\
27.4497991967871	-14.1979603474761\\
27.3092369477912	-14.2282785900934\\
27.1686746987952	-14.2579932725613\\
27.0281124497992	-14.2871059640935\\
26.8875502008032	-14.3156181968417\\
26.7469879518072	-14.3435314663208\\
26.6064257028112	-14.3708472318218\\
26.4658634538153	-14.3975669168128\\
26.3253012048193	-14.4236919093277\\
26.1847389558233	-14.4492235623435\\
26.0441767068273	-14.4741631941465\\
25.9036144578313	-14.498512088687\\
25.7630522088353	-14.5222714959233\\
25.6224899598394	-14.5454426321549\\
25.4819277108434	-14.5680266803456\\
25.3413654618474	-14.5900247904361\\
25.2008032128514	-14.6114380796464\\
25.0602409638554	-14.6322676327687\\
24.9196787148594	-14.6525145024507\\
24.7791164658635	-14.672179709469\\
24.6385542168675	-14.6912642429932\\
24.4979919678715	-14.7097690608417\\
24.3574297188755	-14.7276950897272\\
24.2168674698795	-14.7450432254944\\
24.0763052208835	-14.7618143333486\\
23.9357429718875	-14.7780092480757\\
23.7951807228916	-14.7936287742539\\
23.6546184738956	-14.8086736864572\\
23.5140562248996	-14.8231447294503\\
23.3734939759036	-14.8370426183759\\
23.2329317269076	-14.850368038934\\
23.0923694779116	-14.863121647553\\
22.9518072289157	-14.8753040715536\\
22.8112449799197	-14.8869159093044\\
22.6706827309237	-14.8979577303708\\
22.5301204819277	-14.9084300756555\\
22.3895582329317	-14.9183334575324\\
22.2489959839357	-14.9276683599727\\
22.1084337349398	-14.9364352386641\\
21.9678714859438	-14.9446345211223\\
21.8273092369478	-14.9522666067963\\
21.6867469879518	-14.9593318671658\\
21.5461847389558	-14.9658306458322\\
21.4056224899598	-14.9717632586022\\
21.2650602409639	-14.9771299935648\\
21.1244979919679	-14.9819311111617\\
20.9839357429719	-14.9861668442501\\
20.8433734939759	-14.9898373981598\\
20.7028112449799	-14.9929429507427\\
20.5622489959839	-14.995483652416\\
20.421686746988	-14.9974596261988\\
20.281124497992	-14.9988709677416\\
20.140562248996	-14.9997177453498\\
20	-15\\
20	-15\\
18.75	-15\\
17.5	-15\\
16.25	-15\\
15	-15\\
15	-15\\
14.8993288590604	-14.9996621735752\\
14.7986577181208	-14.9986486486459\\
14.6979865771812	-14.996959288217\\
14.5973154362416	-14.9945938638601\\
14.496644295302	-14.9915520555594\\
14.3959731543624	-14.9878334514949\\
14.2953020134228	-14.983437547763\\
14.1946308724832	-14.9783637480348\\
14.0939597315436	-14.9726113631502\\
13.993288590604	-14.9661796106485\\
13.8926174496644	-14.9590676142336\\
13.7919463087248	-14.9512744031737\\
13.6912751677852	-14.9427989116345\\
13.5906040268456	-14.9336399779443\\
13.489932885906	-14.9237963437904\\
13.3892617449664	-14.913266653345\\
13.2885906040268	-14.9020494523195\\
13.1879194630872	-14.8901431869456\\
13.0872483221477	-14.8775462028815\\
12.9865771812081	-14.864256744041\\
12.8859060402685	-14.8502729513442\\
12.7852348993289	-14.8355928613874\\
12.6845637583893	-14.8202144050292\\
12.5838926174497	-14.804135405892\\
12.4832214765101	-14.7873535787747\\
12.3825503355705	-14.7698665279744\\
12.2818791946309	-14.7516717455148\\
12.1812080536913	-14.7327666092769\\
12.0805369127517	-14.7131483810296\\
11.9798657718121	-14.6928142043561\\
11.8791946308725	-14.6717611024725\\
11.7785234899329	-14.6499859759348\\
11.6778523489933	-14.6274856002291\\
11.5771812080537	-14.6042566232417\\
11.4765100671141	-14.5802955626027\\
11.3758389261745	-14.5555988028995\\
11.2751677852349	-14.5301625927533\\
11.1744966442953	-14.5039830417542\\
11.0738255033557	-14.4770561172463\\
10.9731543624161	-14.4493776409599\\
10.8724832214765	-14.4209432854792\\
10.7718120805369	-14.3917485705423\\
10.6711409395973	-14.3617888591627\\
10.5704697986577	-14.3310593535648\\
10.4697986577181	-14.2995550909246\\
10.3691275167785	-14.2672709389057\\
10.2684563758389	-14.2342015909801\\
10.1677852348993	-14.2003415615239\\
10.0671140939597	-14.1656851806748\\
9.96644295302013	-14.1302265889404\\
9.86577181208054	-14.0939597315436\\
9.76510067114094	-14.0568783524903\\
9.66442953020134	-14.0189759883457\\
9.56375838926174	-13.9802459617017\\
9.46308724832215	-13.9406813743198\\
9.36241610738255	-13.9002750999288\\
9.26174496644295	-13.8590197766602\\
9.16107382550335	-13.8169077990981\\
9.06040268456376	-13.7739313099225\\
8.95973154362416	-13.73008219112\\
8.85906040268456	-13.6853520547381\\
8.75838926174497	-13.6397322331525\\
8.65771812080537	-13.5932137688201\\
8.55704697986577	-13.5457874034824\\
8.45637583892617	-13.4974435667874\\
8.35570469798658	-13.4481723642896\\
8.25503355704698	-13.3979635647899\\
8.15436241610738	-13.3468065869704\\
8.05369127516779	-13.2946904852772\\
7.95302013422819	-13.241603935\\
7.85234899328859	-13.1875352164935\\
7.75167785234899	-13.1324721984808\\
7.6510067114094	-13.0764023203728\\
7.5503355704698	-13.0193125735345\\
7.4496644295302	-12.961189481421\\
7.3489932885906	-12.9020190784996\\
7.24832214765101	-12.8417868878674\\
7.14765100671141	-12.7804778974653\\
7.04697986577181	-12.7180765347815\\
6.94630872483221	-12.6545666399244\\
6.84563758389262	-12.589931436937\\
6.74496644295302	-12.5241535032124\\
6.64429530201342	-12.4572147368521\\
6.54362416107383	-12.3890963217995\\
6.44295302013423	-12.3197786905597\\
6.34228187919463	-12.2492414842993\\
6.24161073825503	-12.1774635100973\\
6.14093959731544	-12.1044226950973\\
6.04026845637584	-12.0300960372803\\
5.93959731543624	-11.9544595525498\\
5.83892617449664	-11.8774882177874\\
5.73825503355705	-11.7991559094949\\
5.63758389261745	-11.7194353376015\\
5.53691275167785	-11.6382979739583\\
5.43624161073826	-11.555713974992\\
5.33557046979866	-11.4716520979226\\
5.23489932885906	-11.3860796098781\\
5.13422818791946	-11.2989621891551\\
5.03355704697987	-11.2102638177786\\
4.93288590604027	-11.1199466644043\\
4.83221476510067	-11.0279709564799\\
4.73154362416107	-10.9342948404318\\
4.63087248322148	-10.838874228479\\
4.53020134228188	-10.7416626304704\\
4.42953020134228	-10.6426109689148\\
4.32885906040269	-10.5416673751001\\
4.22818791946309	-10.4387769638784\\
4.12751677852349	-10.3338815843182\\
4.02684563758389	-10.2269195429801\\
3.9261744966443	-10.1178252960421\\
3.8255033557047	-10.0065291058705\\
3.7248322147651	-9.89295665687468\\
3.6241610738255	-9.77702862457368\\
3.52348993288591	-9.65866019070082\\
3.42281879194631	-9.53776049583383\\
3.32214765100671	-9.41423201940083\\
3.22147651006711	-9.28796987490266\\
3.12080536912752	-9.15886100570646\\
3.02013422818792	-9.02678326367512\\
2.91946308724832	-8.89160434902747\\
2.81879194630872	-8.75318058494664\\
2.71812080536913	-8.61135549425834\\
2.61744966442953	-8.46595813756858\\
2.51677852348993	-8.31680116200921\\
2.41610738255033	-8.16367849639461\\
2.31543624161074	-8.00636261103366\\
2.21476510067114	-7.84460123709187\\
2.11409395973154	-7.67811340899397\\
2.01342281879195	-7.50658465059352\\
1.91275167785235	-7.32966106681909\\
1.81208053691275	-7.14694201985189\\
1.71140939597315	-6.95797095125939\\
1.61073825503356	-6.76222374110604\\
1.51006711409396	-6.55909374332695\\
1.40939597315436	-6.34787225655078\\
1.30872483221476	-6.12772260142276\\
1.20805369127517	-5.89764503995039\\
1.10738255033557	-5.65642823673026\\
1.00671140939597	-5.40258034832164\\
0.906040268456376	-5.13422818791947\\
0.805369127516778	-4.84896425991636\\
0.704697986577182	-4.5436043341195\\
0.604026845637584	-4.2137817858637\\
0.503355704697986	-3.85321997496738\\
0.40268456375839	-3.45230097976153\\
0.302013422818792	-2.9948606940893\\
0.201342281879194	-2.44943457595901\\
0.100671140939598	-1.73493502747794\\
0	-0\\
};

\addplot [color=mycolor2]
  table[row sep=crcr]{%
-5	0\\
-5	1.00401606425703\\
-5	2.00803212851406\\
-5	3.01204819277108\\
-5	4.01606425702811\\
-5	5.02008032128514\\
-5	6.02409638554217\\
-5	7.0281124497992\\
-5	8.03212851405623\\
-5	9.03614457831325\\
-5	10.0401606425703\\
-5	11.0441767068273\\
-5	12.0481927710843\\
-5	13.0522088353414\\
-5	14.0562248995984\\
-5	15.0602409638554\\
-5	16.0642570281125\\
-5	17.0682730923695\\
-5	18.0722891566265\\
-5	19.0763052208835\\
-5	20.0803212851406\\
-5	21.0843373493976\\
-5	22.0883534136546\\
-5	23.0923694779116\\
-5	24.0963855421687\\
-5	25.1004016064257\\
-5	26.1044176706827\\
-5	27.1084337349398\\
-5	28.1124497991968\\
-5	29.1164658634538\\
-5	30.1204819277108\\
-5	31.1244979919679\\
-5	32.1285140562249\\
-5	33.1325301204819\\
-5	34.136546184739\\
-5	35.140562248996\\
-5	36.144578313253\\
-5	37.14859437751\\
-5	38.1526104417671\\
-5	39.1566265060241\\
-5	40.1606425702811\\
-5	41.1646586345381\\
-5	42.1686746987952\\
-5	43.1726907630522\\
-5	44.1767068273092\\
-5	45.1807228915663\\
-5	46.1847389558233\\
-5	47.1887550200803\\
-5	48.1927710843374\\
-5	49.1967871485944\\
-5	50.2008032128514\\
-5	51.2048192771084\\
-5	52.2088353413655\\
-5	53.2128514056225\\
-5	54.2168674698795\\
-5	55.2208835341365\\
-5	56.2248995983936\\
-5	57.2289156626506\\
-5	58.2329317269076\\
-5	59.2369477911647\\
-5	60.2409638554217\\
-5	61.2449799196787\\
-5	62.2489959839357\\
-5	63.2530120481928\\
-5	64.2570281124498\\
-5	65.2610441767068\\
-5	66.2650602409639\\
-5	67.2690763052209\\
-5	68.2730923694779\\
-5	69.2771084337349\\
-5	70.281124497992\\
-5	71.285140562249\\
-5	72.289156626506\\
-5	73.2931726907631\\
-5	74.2971887550201\\
-5	75.3012048192771\\
-5	76.3052208835341\\
-5	77.3092369477912\\
-5	78.3132530120482\\
-5	79.3172690763052\\
-5	80.3212851405623\\
-5	81.3253012048193\\
-5	82.3293172690763\\
-5	83.3333333333333\\
-5	84.3373493975904\\
-5	85.3413654618474\\
-5	86.3453815261044\\
-5	87.3493975903614\\
-5	88.3534136546185\\
-5	89.3574297188755\\
-5	90.3614457831325\\
-5	91.3654618473896\\
-5	92.3694779116466\\
-5	93.3734939759036\\
-5	94.3775100401606\\
-5	95.3815261044177\\
-5	96.3855421686747\\
-5	97.3895582329317\\
-5	98.3935742971887\\
-5	99.3975903614458\\
-5	100.401606425703\\
-5	101.40562248996\\
-5	102.409638554217\\
-5	103.413654618474\\
-5	104.417670682731\\
-5	105.421686746988\\
-5	106.425702811245\\
-5	107.429718875502\\
-5	108.433734939759\\
-5	109.437751004016\\
-5	110.441767068273\\
-5	111.44578313253\\
-5	112.449799196787\\
-5	113.453815261044\\
-5	114.457831325301\\
-5	115.461847389558\\
-5	116.465863453815\\
-5	117.469879518072\\
-5	118.473895582329\\
-5	119.477911646586\\
-5	120.481927710843\\
-5	121.4859437751\\
-5	122.489959839357\\
-5	123.493975903614\\
-5	124.497991967871\\
-5	125.502008032129\\
-5	126.506024096386\\
-5	127.510040160643\\
-5	128.5140562249\\
-5	129.518072289157\\
-5	130.522088353414\\
-5	131.526104417671\\
-5	132.530120481928\\
-5	133.534136546185\\
-5	134.538152610442\\
-5	135.542168674699\\
-5	136.546184738956\\
-5	137.550200803213\\
-5	138.55421686747\\
-5	139.558232931727\\
-5	140.562248995984\\
-5	141.566265060241\\
-5	142.570281124498\\
-5	143.574297188755\\
-5	144.578313253012\\
-5	145.582329317269\\
-5	146.586345381526\\
-5	147.590361445783\\
-5	148.59437751004\\
-5	149.598393574297\\
-5	150.602409638554\\
-5	151.606425702811\\
-5	152.610441767068\\
-5	153.614457831325\\
-5	154.618473895582\\
-5	155.622489959839\\
-5	156.626506024096\\
-5	157.630522088353\\
-5	158.63453815261\\
-5	159.638554216867\\
-5	160.642570281125\\
-5	161.646586345382\\
-5	162.650602409639\\
-5	163.654618473896\\
-5	164.658634538153\\
-5	165.66265060241\\
-5	166.666666666667\\
-5	167.670682730924\\
-5	168.674698795181\\
-5	169.678714859438\\
-5	170.682730923695\\
-5	171.686746987952\\
-5	172.690763052209\\
-5	173.694779116466\\
-5	174.698795180723\\
-5	175.70281124498\\
-5	176.706827309237\\
-5	177.710843373494\\
-5	178.714859437751\\
-5	179.718875502008\\
-5	180.722891566265\\
-5	181.726907630522\\
-5	182.730923694779\\
-5	183.734939759036\\
-5	184.738955823293\\
-5	185.74297188755\\
-5	186.746987951807\\
-5	187.751004016064\\
-5	188.755020080321\\
-5	189.759036144578\\
-5	190.763052208835\\
-5	191.767068273092\\
-5	192.771084337349\\
-5	193.775100401606\\
-5	194.779116465863\\
-5	195.78313253012\\
-5	196.787148594377\\
-5	197.791164658635\\
-5	198.795180722892\\
-5	199.799196787149\\
-5	200.803212851406\\
-5	201.807228915663\\
-5	202.81124497992\\
-5	203.815261044177\\
-5	204.819277108434\\
-5	205.823293172691\\
-5	206.827309236948\\
-5	207.831325301205\\
-5	208.835341365462\\
-5	209.839357429719\\
-5	210.843373493976\\
-5	211.847389558233\\
-5	212.85140562249\\
-5	213.855421686747\\
-5	214.859437751004\\
-5	215.863453815261\\
-5	216.867469879518\\
-5	217.871485943775\\
-5	218.875502008032\\
-5	219.879518072289\\
-5	220.883534136546\\
-5	221.887550200803\\
-5	222.89156626506\\
-5	223.895582329317\\
-5	224.899598393574\\
-5	225.903614457831\\
-5	226.907630522088\\
-5	227.911646586345\\
-5	228.915662650602\\
-5	229.919678714859\\
-5	230.923694779116\\
-5	231.927710843373\\
-5	232.931726907631\\
-5	233.935742971888\\
-5	234.939759036145\\
-5	235.943775100402\\
-5	236.947791164659\\
-5	237.951807228916\\
-5	238.955823293173\\
-5	239.95983935743\\
-5	240.963855421687\\
-5	241.967871485944\\
-5	242.971887550201\\
-5	243.975903614458\\
-5	244.979919678715\\
-5	245.983935742972\\
-5	246.987951807229\\
-5	247.991967871486\\
-5	248.995983935743\\
-5	250\\
-5	250\\
-5.1010101010101	250.999948983496\\
-5.2020202020202	251.406907906786\\
-5.3030303030303	251.714198257422\\
-5.4040404040404	251.969049362588\\
-5.50505050505051	252.189846806937\\
-5.60606060606061	252.386062992125\\
-5.70707070707071	252.563348998071\\
-5.80808080808081	252.725401527926\\
-5.90909090909091	252.87479787288\\
-6.01010101010101	253.013420987914\\
-6.11111111111111	253.142696805274\\
-6.21212121212121	253.263736246748\\
-6.31313131313131	253.37742494895\\
-6.41414141414141	253.484482487002\\
-6.51515151515152	253.585502898848\\
-6.61616161616162	253.680983264301\\
-6.71717171717172	253.771344384363\\
-6.81818181818182	253.856946079199\\
-6.91919191919192	253.938098725175\\
-7.02020202020202	254.015072103909\\
-7.12121212121212	254.088102291889\\
-7.22222222222222	254.157397096415\\
-7.32323232323232	254.223140396034\\
-7.42424242424242	254.285495643555\\
-7.52525252525253	254.344608720729\\
-7.62626262626263	254.400610285008\\
-7.72727272727273	254.453617714151\\
-7.82828282828283	254.503736729214\\
-7.92929292929293	254.551062758008\\
-8.03030303030303	254.595682087304\\
-8.13131313131313	254.637672841717\\
-8.23232323232323	254.677105819309\\
-8.33333333333333	254.71404520791\\
-8.43434343434343	254.748549201432\\
-8.53535353535354	254.780670531798\\
-8.63636363636364	254.810456929208\\
-8.73737373737374	254.837951521143\\
-8.83838383838384	254.863193178671\\
-8.93939393939394	254.886216817151\\
-9.04040404040404	254.907053657168\\
-9.14141414141414	254.925731450601\\
-9.24242424242424	254.942274675849\\
-9.34343434343434	254.95670470561\\
-9.44444444444444	254.96903995\\
-9.54545454545455	254.97929597732\\
-9.64646464646465	254.987485614395\\
-9.74747474747475	254.993619028003\\
-9.84848484848485	254.997703788627\\
-9.94949494949495	254.999744917481\\
-10.0505050505051	254.999744917481\\
-10.1515151515152	254.997703788627\\
-10.2525252525253	254.993619028003\\
-10.3535353535354	254.987485614395\\
-10.4545454545455	254.97929597732\\
-10.5555555555556	254.96903995\\
-10.6565656565657	254.95670470561\\
-10.7575757575758	254.942274675849\\
-10.8585858585859	254.925731450601\\
-10.959595959596	254.907053657168\\
-11.0606060606061	254.886216817151\\
-11.1616161616162	254.863193178671\\
-11.2626262626263	254.837951521143\\
-11.3636363636364	254.810456929208\\
-11.4646464646465	254.780670531798\\
-11.5656565656566	254.748549201432\\
-11.6666666666667	254.71404520791\\
-11.7676767676768	254.677105819309\\
-11.8686868686869	254.637672841717\\
-11.969696969697	254.595682087304\\
-12.0707070707071	254.551062758008\\
-12.1717171717172	254.503736729214\\
-12.2727272727273	254.453617714151\\
-12.3737373737374	254.400610285008\\
-12.4747474747475	254.344608720729\\
-12.5757575757576	254.285495643555\\
-12.6767676767677	254.223140396034\\
-12.7777777777778	254.157397096415\\
-12.8787878787879	254.088102291889\\
-12.979797979798	254.015072103909\\
-13.0808080808081	253.938098725175\\
-13.1818181818182	253.856946079199\\
-13.2828282828283	253.771344384363\\
-13.3838383838384	253.680983264301\\
-13.4848484848485	253.585502898848\\
-13.5858585858586	253.484482487002\\
-13.6868686868687	253.37742494895\\
-13.7878787878788	253.263736246748\\
-13.8888888888889	253.142696805274\\
-13.989898989899	253.013420987914\\
-14.0909090909091	252.87479787288\\
-14.1919191919192	252.725401527926\\
-14.2929292929293	252.563348998071\\
-14.3939393939394	252.386062992125\\
-14.4949494949495	252.189846806937\\
-14.5959595959596	251.969049362588\\
-14.6969696969697	251.714198257422\\
-14.7979797979798	251.406907906786\\
-14.8989898989899	250.999948983496\\
-15	250\\
-15	250\\
-15.2020202020202	248.000102033008\\
-15.4040404040404	247.186184186427\\
-15.6060606060606	246.571603485156\\
-15.8080808080808	246.061901274825\\
-16.010101010101	245.620306386125\\
-16.2121212121212	245.22787401575\\
-16.4141414141414	244.873302003859\\
-16.6161616161616	244.549196944149\\
-16.8181818181818	244.250404254239\\
-17.020202020202	243.973158024171\\
-17.2222222222222	243.714606389453\\
-17.4242424242424	243.472527506504\\
-17.6262626262626	243.245150102101\\
-17.8282828282828	243.031035025996\\
-18.030303030303	242.828994202304\\
-18.2323232323232	242.638033471399\\
-18.4343434343434	242.457311231275\\
-18.6363636363636	242.286107841601\\
-18.8383838383838	242.123802549649\\
-19.040404040404	241.969855792181\\
-19.2424242424242	241.823795416223\\
-19.4444444444444	241.685205807169\\
-19.6464646464646	241.553719207933\\
-19.8484848484848	241.42900871289\\
-20.0505050505051	241.310782558543\\
-20.2525252525253	241.198779429983\\
-20.4545454545455	241.092764571698\\
-20.6565656565657	240.992526541573\\
-20.8585858585859	240.897874483985\\
-21.0606060606061	240.808635825392\\
-21.2626262626263	240.724654316566\\
-21.4646464646465	240.645788361382\\
-21.6666666666667	240.571909584179\\
-21.8686868686869	240.502901597137\\
-22.0707070707071	240.438658936405\\
-22.2727272727273	240.379086141583\\
-22.4747474747475	240.324096957715\\
-22.6767676767677	240.273613642658\\
-22.8787878787879	240.227566365699\\
-23.0808080808081	240.185892685664\\
-23.2828282828283	240.148537098799\\
-23.4848484848485	240.115450648303\\
-23.6868686868687	240.08659058878\\
-23.8888888888889	240.061920100001\\
-24.0909090909091	240.041408045361\\
-24.2929292929293	240.025028771209\\
-24.4949494949495	240.012761943993\\
-24.6969696969697	240.004592422745\\
-24.8989898989899	240.000510165039\\
-25.1010101010101	240.000510165039\\
-25.3030303030303	240.004592422745\\
-25.5050505050505	240.012761943993\\
-25.7070707070707	240.025028771209\\
-25.9090909090909	240.041408045361\\
-26.1111111111111	240.061920100001\\
-26.3131313131313	240.08659058878\\
-26.5151515151515	240.115450648303\\
-26.7171717171717	240.148537098799\\
-26.9191919191919	240.185892685664\\
-27.1212121212121	240.227566365699\\
-27.3232323232323	240.273613642658\\
-27.5252525252525	240.324096957715\\
-27.7272727272727	240.379086141583\\
-27.9292929292929	240.438658936405\\
-28.1313131313131	240.502901597137\\
-28.3333333333333	240.571909584179\\
-28.5353535353535	240.645788361382\\
-28.7373737373737	240.724654316566\\
-28.9393939393939	240.808635825392\\
-29.1414141414141	240.897874483985\\
-29.3434343434343	240.992526541573\\
-29.5454545454545	241.092764571698\\
-29.7474747474747	241.198779429983\\
-29.9494949494949	241.310782558543\\
-30.1515151515152	241.42900871289\\
-30.3535353535354	241.553719207933\\
-30.5555555555556	241.685205807169\\
-30.7575757575758	241.823795416223\\
-30.959595959596	241.969855792181\\
-31.1616161616162	242.123802549649\\
-31.3636363636364	242.286107841601\\
-31.5656565656566	242.457311231275\\
-31.7676767676768	242.638033471399\\
-31.969696969697	242.828994202304\\
-32.1717171717172	243.031035025996\\
-32.3737373737374	243.245150102101\\
-32.5757575757576	243.472527506504\\
-32.7777777777778	243.714606389453\\
-32.979797979798	243.973158024171\\
-33.1818181818182	244.250404254239\\
-33.3838383838384	244.549196944149\\
-33.5858585858586	244.873302003859\\
-33.7878787878788	245.22787401575\\
-33.989898989899	245.620306386125\\
-34.1919191919192	246.061901274825\\
-34.3939393939394	246.571603485156\\
-34.5959595959596	247.186184186427\\
-34.7979797979798	248.000102033008\\
-35	250\\
-35	250\\
-35	251.006711409396\\
-35	252.013422818792\\
-35	253.020134228188\\
-35	254.026845637584\\
-35	255.03355704698\\
-35	256.040268456376\\
-35	257.046979865772\\
-35	258.053691275168\\
-35	259.060402684564\\
-35	260.06711409396\\
-35	261.073825503356\\
-35	262.080536912752\\
-35	263.087248322148\\
-35	264.093959731544\\
-35	265.10067114094\\
-35	266.107382550336\\
-35	267.114093959732\\
-35	268.120805369128\\
-35	269.127516778523\\
-35	270.134228187919\\
-35	271.140939597315\\
-35	272.147651006711\\
-35	273.154362416107\\
-35	274.161073825503\\
-35	275.167785234899\\
-35	276.174496644295\\
-35	277.181208053691\\
-35	278.187919463087\\
-35	279.194630872483\\
-35	280.201342281879\\
-35	281.208053691275\\
-35	282.214765100671\\
-35	283.221476510067\\
-35	284.228187919463\\
-35	285.234899328859\\
-35	286.241610738255\\
-35	287.248322147651\\
-35	288.255033557047\\
-35	289.261744966443\\
-35	290.268456375839\\
-35	291.275167785235\\
-35	292.281879194631\\
-35	293.288590604027\\
-35	294.295302013423\\
-35	295.302013422819\\
-35	296.308724832215\\
-35	297.315436241611\\
-35	298.322147651007\\
-35	299.328859060403\\
-35	300.335570469799\\
-35	301.342281879195\\
-35	302.348993288591\\
-35	303.355704697987\\
-35	304.362416107383\\
-35	305.369127516779\\
-35	306.375838926174\\
-35	307.38255033557\\
-35	308.389261744966\\
-35	309.395973154362\\
-35	310.402684563758\\
-35	311.409395973154\\
-35	312.41610738255\\
-35	313.422818791946\\
-35	314.429530201342\\
-35	315.436241610738\\
-35	316.442953020134\\
-35	317.44966442953\\
-35	318.456375838926\\
-35	319.463087248322\\
-35	320.469798657718\\
-35	321.476510067114\\
-35	322.48322147651\\
-35	323.489932885906\\
-35	324.496644295302\\
-35	325.503355704698\\
-35	326.510067114094\\
-35	327.51677852349\\
-35	328.523489932886\\
-35	329.530201342282\\
-35	330.536912751678\\
-35	331.543624161074\\
-35	332.55033557047\\
-35	333.557046979866\\
-35	334.563758389262\\
-35	335.570469798658\\
-35	336.577181208054\\
-35	337.58389261745\\
-35	338.590604026846\\
-35	339.597315436242\\
-35	340.604026845638\\
-35	341.610738255034\\
-35	342.61744966443\\
-35	343.624161073826\\
-35	344.630872483221\\
-35	345.637583892617\\
-35	346.644295302013\\
-35	347.651006711409\\
-35	348.657718120805\\
-35	349.664429530201\\
-35	350.671140939597\\
-35	351.677852348993\\
-35	352.684563758389\\
-35	353.691275167785\\
-35	354.697986577181\\
-35	355.704697986577\\
-35	356.711409395973\\
-35	357.718120805369\\
-35	358.724832214765\\
-35	359.731543624161\\
-35	360.738255033557\\
-35	361.744966442953\\
-35	362.751677852349\\
-35	363.758389261745\\
-35	364.765100671141\\
-35	365.771812080537\\
-35	366.778523489933\\
-35	367.785234899329\\
-35	368.791946308725\\
-35	369.798657718121\\
-35	370.805369127517\\
-35	371.812080536913\\
-35	372.818791946309\\
-35	373.825503355705\\
-35	374.832214765101\\
-35	375.838926174497\\
-35	376.845637583893\\
-35	377.852348993289\\
-35	378.859060402685\\
-35	379.865771812081\\
-35	380.872483221477\\
-35	381.879194630872\\
-35	382.885906040268\\
-35	383.892617449664\\
-35	384.89932885906\\
-35	385.906040268456\\
-35	386.912751677852\\
-35	387.919463087248\\
-35	388.926174496644\\
-35	389.93288590604\\
-35	390.939597315436\\
-35	391.946308724832\\
-35	392.953020134228\\
-35	393.959731543624\\
-35	394.96644295302\\
-35	395.973154362416\\
-35	396.979865771812\\
-35	397.986577181208\\
-35	398.993288590604\\
-35	400\\
-35	400\\
-34.6984924623116	404.242587119438\\
-34.3969849246231	405.984753741089\\
-34.0954773869347	407.311169237979\\
-33.7939698492462	408.420647262574\\
-33.4924623115578	409.390398895918\\
-33.1909547738693	410.260120317894\\
-32.8894472361809	411.053448912456\\
-32.5879396984925	411.785821277804\\
-32.286432160804	412.467983795861\\
-31.9849246231156	413.107777961377\\
-31.6834170854271	413.71113608149\\
-31.3819095477387	414.282676521399\\
-31.0804020100502	414.826079420877\\
-30.7788944723618	415.344334452261\\
-30.4773869346734	415.839910194851\\
-30.1758793969849	416.31487347755\\
-29.8743718592965	416.770975660511\\
-29.572864321608	417.209716413464\\
-29.2713567839196	417.632391779554\\
-28.9698492462312	418.040131017067\\
-28.6683417085427	418.433925267497\\
-28.3668341708543	418.814650165012\\
-28.0653266331658	419.183083884125\\
-27.7638190954774	419.539921703742\\
-27.4623115577889	419.885787876793\\
-27.1608040201005	420.221245391486\\
-26.8592964824121	420.546804065207\\
-26.5577889447236	420.862927306942\\
-26.2562814070352	421.170037806885\\
-25.9547738693467	421.468522354474\\
-25.6532663316583	421.758735942917\\
-25.3517587939699	422.041005285424\\
-25.0502512562814	422.315631843172\\
-24.748743718593	422.582894445516\\
-24.4472361809045	422.843051567689\\
-24.1457286432161	423.096343319243\\
-23.8442211055276	423.34299318695\\
-23.5427135678392	423.583209568273\\
-23.2412060301508	423.817187125385\\
-22.9396984924623	424.045107984776\\
-22.6381909547739	424.267142803448\\
-22.3366834170854	424.483451719391\\
-22.035175879397	424.694185201339\\
-21.7336683417085	424.899484810528\\
-21.4321608040201	425.099483885348\\
-21.1306532663317	425.294308158188\\
-20.8291457286432	425.484076312501\\
-20.5276381909548	425.668900486986\\
-20.2261306532663	425.848886732888\\
-19.9246231155779	426.024135429596\\
-19.6231155778894	426.194741663084\\
-19.321608040201	426.360795571129\\
-19.0201005025126	426.522382658793\\
-18.7185929648241	426.679584087191\\
-18.4170854271357	426.832476938236\\
-18.1155778894472	426.981134457725\\
-17.8140703517588	427.125626278856\\
-17.5125628140704	427.266018628027\\
-17.2110552763819	427.402374514577\\
-16.9095477386935	427.534753905924\\
-16.608040201005	427.663213889421\\
-16.3065326633166	427.787808822096\\
-16.0050251256281	427.908590469321\\
-15.7035175879397	428.02560813336\\
-15.4020100502513	428.138908772631\\
-15.1005025125628	428.248537112454\\
-14.7989949748744	428.354535747961\\
-14.4974874371859	428.456945239792\\
-14.1959798994975	428.555804203139\\
-13.894472361809	428.651149390644\\
-13.5929648241206	428.743015769599\\
-13.2914572864322	428.831436593886\\
-12.9899497487437	428.916443471017\\
-12.6884422110553	428.99806642463\\
-12.3869346733668	429.076333952743\\
-12.0854271356784	429.151273082063\\
-11.78391959799	429.222909418605\\
-11.4824120603015	429.291267194856\\
-11.1809045226131	429.3563693137\\
-10.8793969849246	429.418237389308\\
-10.5778894472362	429.476891785166\\
-10.2763819095477	429.532351649413\\
-9.9748743718593	429.584634947625\\
-9.67336683417085	429.633758493199\\
-9.37185929648241	429.679737975457\\
-9.07035175879397	429.722587985565\\
-8.76884422110553	429.762322040409\\
-8.46733668341708	429.798952604476\\
-8.16582914572864	429.832491109863\\
-7.8643216080402	429.862947974467\\
-7.56281407035176	429.890332618437\\
-7.26130653266332	429.914653478945\\
-6.95979899497488	429.935918023326\\
-6.65829145728643	429.954132760651\\
-6.35678391959799	429.969303251753\\
-6.05527638190955	429.981434117763\\
-5.75376884422111	429.990529047176\\
-5.45226130653266	429.996590801466\\
-5.15075376884422	429.999621219295\\
-4.84924623115578	429.999621219295\\
-4.54773869346734	429.996590801466\\
-4.24623115577889	429.990529047176\\
-3.94472361809045	429.981434117763\\
-3.64321608040201	429.969303251753\\
-3.34170854271357	429.954132760651\\
-3.04020100502512	429.935918023326\\
-2.73869346733668	429.914653478945\\
-2.43718592964824	429.890332618437\\
-2.1356783919598	429.862947974467\\
-1.83417085427136	429.832491109863\\
-1.53266331658291	429.798952604476\\
-1.23115577889448	429.762322040409\\
-0.929648241206031	429.722587985565\\
-0.628140703517587	429.679737975457\\
-0.326633165829143	429.633758493199\\
-0.0251256281407066	429.584634947625\\
0.276381909547737	429.532351649413\\
0.577889447236181	429.476891785166\\
0.879396984924625	429.418237389308\\
1.18090452261306	429.3563693137\\
1.48241206030151	429.291267194856\\
1.78391959798995	429.222909418605\\
2.08542713567839	429.151273082063\\
2.38693467336683	429.076333952743\\
2.68844221105527	428.99806642463\\
2.98994974874372	428.916443471017\\
3.29145728643216	428.831436593886\\
3.5929648241206	428.743015769599\\
3.89447236180904	428.651149390644\\
4.19597989949749	428.555804203139\\
4.49748743718593	428.456945239792\\
4.79899497487437	428.354535747961\\
5.10050251256281	428.248537112454\\
5.40201005025126	428.138908772631\\
5.7035175879397	428.02560813336\\
6.00502512562814	427.908590469321\\
6.30653266331658	427.787808822096\\
6.60804020100502	427.663213889421\\
6.90954773869347	427.534753905924\\
7.21105527638191	427.402374514577\\
7.51256281407035	427.266018628027\\
7.81407035175879	427.125626278856\\
8.11557788944724	426.981134457725\\
8.41708542713568	426.832476938236\\
8.71859296482412	426.679584087191\\
9.02010050251256	426.522382658793\\
9.32160804020101	426.360795571129\\
9.62311557788945	426.194741663084\\
9.92462311557789	426.024135429596\\
10.2261306532663	425.848886732888\\
10.5276381909548	425.668900486986\\
10.8291457286432	425.484076312501\\
11.1306532663317	425.294308158188\\
11.4321608040201	425.099483885348\\
11.7336683417085	424.899484810528\\
12.035175879397	424.694185201339\\
12.3366834170854	424.483451719391\\
12.6381909547739	424.267142803448\\
12.9396984924623	424.045107984776\\
13.2412060301508	423.817187125385\\
13.5427135678392	423.583209568273\\
13.8442211055276	423.34299318695\\
14.1457286432161	423.096343319243\\
14.4472361809045	422.843051567689\\
14.748743718593	422.582894445516\\
15.0502512562814	422.315631843172\\
15.3517587939699	422.041005285424\\
15.6532663316583	421.758735942917\\
15.9547738693467	421.468522354474\\
16.2562814070352	421.170037806885\\
16.5577889447236	420.862927306942\\
16.8592964824121	420.546804065207\\
17.1608040201005	420.221245391486\\
17.4623115577889	419.885787876793\\
17.7638190954774	419.539921703742\\
18.0653266331658	419.183083884125\\
18.3668341708543	418.814650165012\\
18.6683417085427	418.433925267497\\
18.9698492462312	418.040131017067\\
19.2713567839196	417.632391779554\\
19.572864321608	417.209716413464\\
19.8743718592965	416.770975660511\\
20.1758793969849	416.31487347755\\
20.4773869346734	415.839910194851\\
20.7788944723618	415.344334452261\\
21.0804020100502	414.826079420877\\
21.3819095477387	414.282676521399\\
21.6834170854271	413.71113608149\\
21.9849246231156	413.107777961377\\
22.286432160804	412.467983795861\\
22.5879396984925	411.785821277804\\
22.8894472361809	411.053448912456\\
23.1909547738693	410.260120317894\\
23.4924623115578	409.390398895918\\
23.7939698492462	408.420647262574\\
24.0954773869347	407.311169237979\\
24.3969849246231	405.984753741089\\
24.6984924623116	404.242587119438\\
25	400\\
25	400\\
25	398.989361702128\\
25	397.978723404255\\
25	396.968085106383\\
25	395.957446808511\\
25	394.946808510638\\
25	393.936170212766\\
25	392.925531914894\\
25	391.914893617021\\
25	390.904255319149\\
25	389.893617021277\\
25	388.882978723404\\
25	387.872340425532\\
25	386.86170212766\\
25	385.851063829787\\
25	384.840425531915\\
25	383.829787234043\\
25	382.81914893617\\
25	381.808510638298\\
25	380.797872340426\\
25	379.787234042553\\
25	378.776595744681\\
25	377.765957446808\\
25	376.755319148936\\
25	375.744680851064\\
25	374.734042553192\\
25	373.723404255319\\
25	372.712765957447\\
25	371.702127659574\\
25	370.691489361702\\
25	369.68085106383\\
25	368.670212765957\\
25	367.659574468085\\
25	366.648936170213\\
25	365.63829787234\\
25	364.627659574468\\
25	363.617021276596\\
25	362.606382978723\\
25	361.595744680851\\
25	360.585106382979\\
25	359.574468085106\\
25	358.563829787234\\
25	357.553191489362\\
25	356.542553191489\\
25	355.531914893617\\
25	354.521276595745\\
25	353.510638297872\\
25	352.5\\
25	351.489361702128\\
25	350.478723404255\\
25	349.468085106383\\
25	348.457446808511\\
25	347.446808510638\\
25	346.436170212766\\
25	345.425531914894\\
25	344.414893617021\\
25	343.404255319149\\
25	342.393617021277\\
25	341.382978723404\\
25	340.372340425532\\
25	339.36170212766\\
25	338.351063829787\\
25	337.340425531915\\
25	336.329787234043\\
25	335.31914893617\\
25	334.308510638298\\
25	333.297872340426\\
25	332.287234042553\\
25	331.276595744681\\
25	330.265957446808\\
25	329.255319148936\\
25	328.244680851064\\
25	327.234042553192\\
25	326.223404255319\\
25	325.212765957447\\
25	324.202127659574\\
25	323.191489361702\\
25	322.18085106383\\
25	321.170212765957\\
25	320.159574468085\\
25	319.148936170213\\
25	318.13829787234\\
25	317.127659574468\\
25	316.117021276596\\
25	315.106382978723\\
25	314.095744680851\\
25	313.085106382979\\
25	312.074468085106\\
25	311.063829787234\\
25	310.053191489362\\
25	309.042553191489\\
25	308.031914893617\\
25	307.021276595745\\
25	306.010638297872\\
25	305\\
25	305\\
25.1020408163265	303.995014510021\\
25.2040816326531	303.586080973413\\
25.3061224489796	303.277352753456\\
25.4081632653061	303.021355160238\\
25.5102040816327	302.799606259914\\
25.6122448979592	302.602583648067\\
25.7142857142857	302.424606231811\\
25.8163265306122	302.261957578572\\
25.9183673469388	302.112046592228\\
26.0204081632653	301.972980209349\\
26.1224489795918	301.84332483709\\
26.2244897959184	301.721963590978\\
26.3265306122449	301.608006094164\\
26.4285714285714	301.500728938881\\
26.530612244898	301.399534946045\\
26.6326530612245	301.303924433413\\
26.734693877551	301.213474425453\\
26.8367346938776	301.127823273263\\
26.9387755102041	301.046659055549\\
27.0408163265306	300.969710682014\\
27.1428571428571	300.896740966759\\
27.2448979591837	300.827541163212\\
27.3469387755102	300.761926600552\\
27.4489795918367	300.699733162101\\
27.5510204081633	300.640814415654\\
27.6530612244898	300.585039254534\\
27.7551020408163	300.532289943063\\
27.8571428571429	300.482460485474\\
27.9591836734694	300.435455255848\\
28.0612244897959	300.391187840557\\
28.1632653061224	300.349580055059\\
28.265306122449	300.310561104867\\
28.3673469387755	300.274066866568\\
28.469387755102	300.240039269514\\
28.5714285714286	300.2084257625\\
28.6734693877551	300.179178852646\\
28.7755102040816	300.152255706033\\
28.8775510204082	300.127617801505\\
28.9795918367347	300.105230630526\\
29.0816326530612	300.085063437228\\
29.1836734693878	300.067088993767\\
29.2857142857143	300.051283406946\\
29.3877551020408	300.03762595274\\
29.4897959183673	300.026098935937\\
29.5918367346939	300.016687572623\\
29.6938775510204	300.009379893617\\
29.7959183673469	300.00416666739\\
29.8979591836735	300.001041341259\\
30	300\\
30	300\\
30.1020408163265	299.998958658741\\
30.2040816326531	299.99583333261\\
30.3061224489796	299.990620106383\\
30.4081632653061	299.983312427377\\
30.5102040816327	299.973901064063\\
30.6122448979592	299.96237404726\\
30.7142857142857	299.948716593054\\
30.8163265306122	299.932911006233\\
30.9183673469388	299.914936562772\\
31.0204081632653	299.894769369474\\
31.1224489795918	299.872382198495\\
31.2244897959184	299.847744293967\\
31.3265306122449	299.820821147354\\
31.4285714285714	299.7915742375\\
31.530612244898	299.759960730486\\
31.6326530612245	299.725933133432\\
31.734693877551	299.689438895133\\
31.8367346938776	299.650419944941\\
31.9387755102041	299.608812159443\\
32.0408163265306	299.564544744152\\
32.1428571428571	299.517539514526\\
32.2448979591837	299.467710056937\\
32.3469387755102	299.414960745466\\
32.4489795918367	299.359185584346\\
32.5510204081633	299.300266837899\\
32.6530612244898	299.238073399448\\
32.7551020408163	299.172458836788\\
32.8571428571429	299.103259033241\\
32.9591836734694	299.030289317986\\
33.0612244897959	298.953340944451\\
33.1632653061224	298.872176726737\\
33.265306122449	298.786525574547\\
33.3673469387755	298.696075566587\\
33.469387755102	298.600465053955\\
33.5714285714286	298.499271061119\\
33.6734693877551	298.391993905836\\
33.7755102040816	298.278036409022\\
33.8775510204082	298.15667516291\\
33.9795918367347	298.027019790651\\
34.0816326530612	297.887953407772\\
34.1836734693878	297.738042421428\\
34.2857142857143	297.575393768189\\
34.3877551020408	297.397416351933\\
34.4897959183673	297.200393740086\\
34.5918367346939	296.978644839762\\
34.6938775510204	296.722647246544\\
34.7959183673469	296.413919026587\\
34.8979591836735	296.004985489979\\
35	295\\
35	295\\
35	293.995983935743\\
35	292.991967871486\\
35	291.987951807229\\
35	290.983935742972\\
35	289.979919678715\\
35	288.975903614458\\
35	287.971887550201\\
35	286.967871485944\\
35	285.963855421687\\
35	284.95983935743\\
35	283.955823293173\\
35	282.951807228916\\
35	281.947791164659\\
35	280.943775100402\\
35	279.939759036145\\
35	278.935742971888\\
35	277.931726907631\\
35	276.927710843374\\
35	275.923694779116\\
35	274.919678714859\\
35	273.915662650602\\
35	272.911646586345\\
35	271.907630522088\\
35	270.903614457831\\
35	269.899598393574\\
35	268.895582329317\\
35	267.89156626506\\
35	266.887550200803\\
35	265.883534136546\\
35	264.879518072289\\
35	263.875502008032\\
35	262.871485943775\\
35	261.867469879518\\
35	260.863453815261\\
35	259.859437751004\\
35	258.855421686747\\
35	257.85140562249\\
35	256.847389558233\\
35	255.843373493976\\
35	254.839357429719\\
35	253.835341365462\\
35	252.831325301205\\
35	251.827309236948\\
35	250.823293172691\\
35	249.819277108434\\
35	248.815261044177\\
35	247.81124497992\\
35	246.807228915663\\
35	245.803212851406\\
35	244.799196787149\\
35	243.795180722892\\
35	242.791164658635\\
35	241.787148594378\\
35	240.78313253012\\
35	239.779116465863\\
35	238.775100401606\\
35	237.771084337349\\
35	236.767068273092\\
35	235.763052208835\\
35	234.759036144578\\
35	233.755020080321\\
35	232.751004016064\\
35	231.746987951807\\
35	230.74297188755\\
35	229.738955823293\\
35	228.734939759036\\
35	227.730923694779\\
35	226.726907630522\\
35	225.722891566265\\
35	224.718875502008\\
35	223.714859437751\\
35	222.710843373494\\
35	221.706827309237\\
35	220.70281124498\\
35	219.698795180723\\
35	218.694779116466\\
35	217.690763052209\\
35	216.686746987952\\
35	215.682730923695\\
35	214.678714859438\\
35	213.674698795181\\
35	212.670682730924\\
35	211.666666666667\\
35	210.66265060241\\
35	209.658634538153\\
35	208.654618473896\\
35	207.650602409639\\
35	206.646586345382\\
35	205.642570281125\\
35	204.638554216867\\
35	203.63453815261\\
35	202.630522088353\\
35	201.626506024096\\
35	200.622489959839\\
35	199.618473895582\\
35	198.614457831325\\
35	197.610441767068\\
35	196.606425702811\\
35	195.602409638554\\
35	194.598393574297\\
35	193.59437751004\\
35	192.590361445783\\
35	191.586345381526\\
35	190.582329317269\\
35	189.578313253012\\
35	188.574297188755\\
35	187.570281124498\\
35	186.566265060241\\
35	185.562248995984\\
35	184.558232931727\\
35	183.55421686747\\
35	182.550200803213\\
35	181.546184738956\\
35	180.542168674699\\
35	179.538152610442\\
35	178.534136546185\\
35	177.530120481928\\
35	176.526104417671\\
35	175.522088353414\\
35	174.518072289157\\
35	173.5140562249\\
35	172.510040160643\\
35	171.506024096386\\
35	170.502008032129\\
35	169.497991967871\\
35	168.493975903614\\
35	167.489959839357\\
35	166.4859437751\\
35	165.481927710843\\
35	164.477911646586\\
35	163.473895582329\\
35	162.469879518072\\
35	161.465863453815\\
35	160.461847389558\\
35	159.457831325301\\
35	158.453815261044\\
35	157.449799196787\\
35	156.44578313253\\
35	155.441767068273\\
35	154.437751004016\\
35	153.433734939759\\
35	152.429718875502\\
35	151.425702811245\\
35	150.421686746988\\
35	149.417670682731\\
35	148.413654618474\\
35	147.409638554217\\
35	146.40562248996\\
35	145.401606425703\\
35	144.397590361446\\
35	143.393574297189\\
35	142.389558232932\\
35	141.385542168675\\
35	140.381526104418\\
35	139.377510040161\\
35	138.373493975904\\
35	137.369477911647\\
35	136.36546184739\\
35	135.361445783133\\
35	134.357429718875\\
35	133.353413654618\\
35	132.349397590361\\
35	131.345381526104\\
35	130.341365461847\\
35	129.33734939759\\
35	128.333333333333\\
35	127.329317269076\\
35	126.325301204819\\
35	125.321285140562\\
35	124.317269076305\\
35	123.313253012048\\
35	122.309236947791\\
35	121.305220883534\\
35	120.301204819277\\
35	119.29718875502\\
35	118.293172690763\\
35	117.289156626506\\
35	116.285140562249\\
35	115.281124497992\\
35	114.277108433735\\
35	113.273092369478\\
35	112.269076305221\\
35	111.265060240964\\
35	110.261044176707\\
35	109.25702811245\\
35	108.253012048193\\
35	107.248995983936\\
35	106.244979919679\\
35	105.240963855422\\
35	104.236947791165\\
35	103.232931726908\\
35	102.228915662651\\
35	101.224899598394\\
35	100.220883534137\\
35	99.2168674698795\\
35	98.2128514056225\\
35	97.2088353413655\\
35	96.2048192771084\\
35	95.2008032128514\\
35	94.1967871485944\\
35	93.1927710843374\\
35	92.1887550200803\\
35	91.1847389558233\\
35	90.1807228915663\\
35	89.1767068273092\\
35	88.1726907630522\\
35	87.1686746987952\\
35	86.1646586345381\\
35	85.1606425702811\\
35	84.1566265060241\\
35	83.1526104417671\\
35	82.1485943775101\\
35	81.144578313253\\
35	80.140562248996\\
35	79.136546184739\\
35	78.1325301204819\\
35	77.1285140562249\\
35	76.1244979919679\\
35	75.1204819277108\\
35	74.1164658634538\\
35	73.1124497991968\\
35	72.1084337349398\\
35	71.1044176706827\\
35	70.1004016064257\\
35	69.0963855421687\\
35	68.0923694779117\\
35	67.0883534136546\\
35	66.0843373493976\\
35	65.0803212851406\\
35	64.0763052208835\\
35	63.0722891566265\\
35	62.0682730923695\\
35	61.0642570281125\\
35	60.0602409638554\\
35	59.0562248995984\\
35	58.0522088353414\\
35	57.0481927710843\\
35	56.0441767068273\\
35	55.0401606425703\\
35	54.0361445783132\\
35	53.0321285140562\\
35	52.0281124497992\\
35	51.0240963855422\\
35	50.0200803212851\\
35	49.0160642570281\\
35	48.0120481927711\\
35	47.0080321285141\\
35	46.004016064257\\
35	45\\
35	45\\
35.0671140939597	43.8433766483481\\
35.1342281879195	43.3670436160273\\
35.2013422818792	43.0034262039405\\
35.2684563758389	42.6984660134923\\
35.3355704697987	42.4311866833551\\
35.4026845637584	42.1908121427575\\
35.4697986577181	41.9709304439203\\
35.5369127516779	41.7673571600558\\
35.6040268456376	41.5771812080537\\
35.6711409395973	41.3982797677856\\
35.738255033557	41.2290478421798\\
35.8053691275168	41.0682366400331\\
35.8724832214765	40.9148515990515\\
35.9395973154362	40.7680851622995\\
36.006711409396	40.627270837782\\
36.0738255033557	40.4918508392626\\
36.1409395973154	40.3613526991604\\
36.2080536912752	40.2353719867654\\
36.2751677852349	40.1135592887873\\
36.3422818791946	39.9956102329377\\
36.4093959731544	39.8812577273374\\
36.4765100671141	39.7702658419388\\
36.5436241610738	39.6624249259776\\
36.6107382550336	39.5575476690703\\
36.6778523489933	39.4554658919939\\
36.744966442953	39.3560279082876\\
36.8120805369128	39.2590963371611\\
36.8791946308725	39.1645462767022\\
36.9463087248322	39.072263767315\\
37.0134228187919	38.9821444908833\\
37.0805369127517	38.8940926628624\\
37.1476510067114	38.8080200833982\\
37.2147651006711	38.7238453203994\\
37.2818791946309	38.6414930027774\\
37.3489932885906	38.5608932061995\\
37.4161073825503	38.4819809169509\\
37.4832214765101	38.4046955620835\\
37.5503355704698	38.3289805960863\\
37.6174496644295	38.254783135972\\
37.6845637583893	38.1820536380133\\
37.751677852349	38.1107456104545\\
37.8187919463087	38.0408153574144\\
37.8859060402685	37.9722217499333\\
37.9530201342282	37.9049260207235\\
38.0201342281879	37.8388915796864\\
38.0872483221477	37.7740838476807\\
38.1543624161074	37.7104701063788\\
38.2214765100671	37.6480193623468\\
38.2885906040268	37.5867022237304\\
38.3557046979866	37.5264907881476\\
38.4228187919463	37.4673585405632\\
38.489932885906	37.4092802600812\\
38.5570469798658	37.3522319347183\\
38.6241610738255	37.2961906833387\\
38.6912751677852	37.2411346840278\\
38.758389261745	37.1870431082656\\
38.8255033557047	37.1338960603367\\
38.8926174496644	37.0816745214751\\
38.9597315436242	37.0303602983001\\
39.0268456375839	36.9799359751465\\
39.0939597315436	36.9303848699351\\
39.1610738255034	36.8816909932685\\
39.2281879194631	36.8338390104671\\
39.2953020134228	36.7868142062935\\
39.3624161073826	36.7406024521337\\
39.4295302013423	36.6951901754319\\
39.496644295302	36.6505643311917\\
39.5637583892617	36.6067123753753\\
39.6308724832215	36.5636222400504\\
39.6979865771812	36.5212823101456\\
39.7651006711409	36.4796814016898\\
39.8322147651007	36.4388087414218\\
39.8993288590604	36.3986539476669\\
39.9664429530201	36.359207012386\\
40.0335570469799	36.3204582843104\\
40.1006711409396	36.2823984530848\\
40.1677852348993	36.2450185343461\\
40.2348993288591	36.208309855671\\
40.3020134228188	36.1722640433334\\
40.3691275167785	36.1368730098152\\
40.4362416107383	36.1021289420197\\
40.503355704698	36.0680242901401\\
40.5704697986577	36.0345517571403\\
40.6375838926174	36.0017042888084\\
40.7046979865772	35.969475064345\\
40.7718120805369	35.9378574874533\\
40.8389261744966	35.9068451778983\\
40.9060402684564	35.876431963508\\
40.9731543624161	35.8466118725867\\
41.0402684563758	35.8173791267184\\
41.1073825503356	35.7887281339346\\
41.1744966442953	35.7606534822266\\
41.241610738255	35.7331499333808\\
41.3087248322148	35.7062124171201\\
41.3758389261745	35.6798360255322\\
41.4429530201342	35.6540160077696\\
41.510067114094	35.6287477650065\\
41.5771812080537	35.6040268456376\\
41.6442953020134	35.5798489407064\\
41.7114093959732	35.5562098795501\\
41.7785234899329	35.5331056256507\\
41.8456375838926	35.5105322726799\\
41.9127516778523	35.4884860407296\\
41.9798657718121	35.4669632727169\\
42.0469798657718	35.4459604309568\\
42.1140939597315	35.4254740938915\\
42.1812080536913	35.4055009529718\\
42.248322147651	35.3860378096805\\
42.3154362416107	35.3670815726934\\
42.3825503355705	35.3486292551691\\
42.4496644295302	35.3306779721639\\
42.5167785234899	35.3132249381644\\
42.5838926174497	35.2962674647337\\
42.6510067114094	35.2798029582649\\
42.7181208053691	35.2638289178389\\
42.7852348993289	35.2483429331806\\
42.8523489932886	35.2333426827101\\
42.9194630872483	35.218825931685\\
42.9865771812081	35.2047905304293\\
43.0536912751678	35.1912344126469\\
43.1208053691275	35.1781555938154\\
43.1879194630872	35.1655521696568\\
43.255033557047	35.1534223146838\\
43.3221476510067	35.1417642808169\\
43.3892617449664	35.130576396072\\
43.4563758389262	35.1198570633139\\
43.5234899328859	35.1096047590751\\
43.5906040268456	35.0998180324372\\
43.6577181208054	35.0904955039727\\
43.7248322147651	35.0816358647457\\
43.7919463087248	35.0732378753696\\
43.8590604026846	35.0653003651204\\
43.9261744966443	35.0578222311033\\
43.993288590604	35.0508024374731\\
44.0604026845638	35.0442400147038\\
44.1275167785235	35.0381340589104\\
44.1946308724832	35.0324837312175\\
44.261744966443	35.0272882571776\\
44.3288590604027	35.0225469262343\\
44.3959731543624	35.0182590912332\\
44.4630872483221	35.0144241679768\\
44.5302013422819	35.0110416348247\\
44.5973154362416	35.0081110323367\\
44.6644295302013	35.0056319629604\\
44.7315436241611	35.0036040907599\\
44.7986577181208	35.0020271411887\\
44.8657718120805	35.0009009009027\\
44.9328859060403	35.0002252176166\\
45	35\\
45	35\\
45.1006711409396	34.9996621735752\\
45.2013422818792	34.9986486486459\\
45.3020134228188	34.996959288217\\
45.4026845637584	34.9945938638601\\
45.503355704698	34.9915520555594\\
45.6040268456376	34.9878334514949\\
45.7046979865772	34.983437547763\\
45.8053691275168	34.9783637480348\\
45.9060402684564	34.9726113631502\\
46.006711409396	34.9661796106485\\
46.1073825503356	34.9590676142336\\
46.2080536912752	34.9512744031737\\
46.3087248322148	34.9427989116345\\
46.4093959731544	34.9336399779443\\
46.510067114094	34.9237963437904\\
46.6107382550336	34.913266653345\\
46.7114093959732	34.9020494523195\\
46.8120805369128	34.8901431869456\\
46.9127516778523	34.8775462028815\\
47.0134228187919	34.864256744041\\
47.1140939597315	34.8502729513443\\
47.2147651006711	34.8355928613874\\
47.3154362416107	34.8202144050292\\
47.4161073825503	34.804135405892\\
47.5167785234899	34.7873535787747\\
47.6174496644295	34.7698665279744\\
47.7181208053691	34.7516717455148\\
47.8187919463087	34.7327666092769\\
47.9194630872483	34.7131483810296\\
48.0201342281879	34.6928142043561\\
48.1208053691275	34.6717611024725\\
48.2214765100671	34.6499859759348\\
48.3221476510067	34.6274856002291\\
48.4228187919463	34.6042566232417\\
48.5234899328859	34.5802955626027\\
48.6241610738255	34.5555988028995\\
48.7248322147651	34.5301625927533\\
48.8255033557047	34.5039830417542\\
48.9261744966443	34.4770561172463\\
49.0268456375839	34.4493776409599\\
49.1275167785235	34.4209432854792\\
49.2281879194631	34.3917485705423\\
49.3288590604027	34.3617888591627\\
49.4295302013423	34.3310593535648\\
49.5302013422819	34.2995550909246\\
49.6308724832215	34.2672709389057\\
49.7315436241611	34.2342015909801\\
49.8322147651007	34.2003415615239\\
49.9328859060403	34.1656851806748\\
50.0335570469799	34.1302265889404\\
50.1342281879195	34.0939597315436\\
50.2348993288591	34.0568783524903\\
50.3355704697987	34.0189759883457\\
50.4362416107383	33.9802459617017\\
50.5369127516779	33.9406813743198\\
50.6375838926174	33.9002750999288\\
50.738255033557	33.8590197766602\\
50.8389261744966	33.8169077990981\\
50.9395973154362	33.7739313099225\\
51.0402684563758	33.73008219112\\
51.1409395973154	33.6853520547381\\
51.241610738255	33.6397322331525\\
51.3422818791946	33.5932137688201\\
51.4429530201342	33.5457874034824\\
51.5436241610738	33.4974435667874\\
51.6442953020134	33.4481723642896\\
51.744966442953	33.3979635647899\\
51.8456375838926	33.3468065869704\\
51.9463087248322	33.2946904852772\\
52.0469798657718	33.241603935\\
52.1476510067114	33.1875352164935\\
52.248322147651	33.1324721984808\\
52.3489932885906	33.0764023203728\\
52.4496644295302	33.0193125735345\\
52.5503355704698	32.961189481421\\
52.6510067114094	32.9020190784996\\
52.751677852349	32.8417868878674\\
52.8523489932886	32.7804778974653\\
52.9530201342282	32.7180765347815\\
53.0536912751678	32.6545666399244\\
53.1543624161074	32.589931436937\\
53.255033557047	32.5241535032124\\
53.3557046979866	32.4572147368521\\
53.4563758389262	32.3890963217995\\
53.5570469798658	32.3197786905597\\
53.6577181208054	32.2492414842993\\
53.758389261745	32.1774635100973\\
53.8590604026846	32.1044226950973\\
53.9597315436242	32.0300960372803\\
54.0604026845638	31.9544595525498\\
54.1610738255034	31.8774882177873\\
54.261744966443	31.7991559094949\\
54.3624161073826	31.7194353376015\\
54.4630872483221	31.6382979739583\\
54.5637583892617	31.555713974992\\
54.6644295302013	31.4716520979226\\
54.7651006711409	31.3860796098781\\
54.8657718120805	31.2989621891551\\
54.9664429530201	31.2102638177786\\
55.0671140939597	31.1199466644043\\
55.1677852348993	31.0279709564799\\
55.2684563758389	30.9342948404318\\
55.3691275167785	30.838874228479\\
55.4697986577181	30.7416626304704\\
55.5704697986577	30.6426109689148\\
55.6711409395973	30.5416673751001\\
55.7718120805369	30.4387769638784\\
55.8724832214765	30.3338815843182\\
55.9731543624161	30.2269195429801\\
56.0738255033557	30.1178252960421\\
56.1744966442953	30.0065291058705\\
56.2751677852349	29.8929566568747\\
56.3758389261745	29.7770286245737\\
56.4765100671141	29.6586601907008\\
56.5771812080537	29.5377604958338\\
56.6778523489933	29.4142320194008\\
56.7785234899329	29.2879698749027\\
56.8791946308725	29.1588610057065\\
56.9798657718121	29.0267832636751\\
57.0805369127517	28.8916043490275\\
57.1812080536913	28.7531805849466\\
57.2818791946309	28.6113554942583\\
57.3825503355705	28.4659581375686\\
57.4832214765101	28.3168011620092\\
57.5838926174497	28.1636784963946\\
57.6845637583893	28.0063626110337\\
57.7852348993289	27.8446012370919\\
57.8859060402685	27.678113408994\\
57.9865771812081	27.5065846505935\\
58.0872483221477	27.3296610668191\\
58.1879194630872	27.1469420198519\\
58.2885906040268	26.9579709512594\\
58.3892617449664	26.762223741106\\
58.489932885906	26.5590937433269\\
58.5906040268456	26.3478722565508\\
58.6912751677852	26.1277226014228\\
58.7919463087248	25.8976450399504\\
58.8926174496644	25.6564282367303\\
58.993288590604	25.4025803483216\\
59.0939597315436	25.1342281879195\\
59.1946308724832	24.8489642599164\\
59.2953020134228	24.5436043341195\\
59.3959731543624	24.2137817858637\\
59.496644295302	23.8532199749674\\
59.5973154362416	23.4523009797615\\
59.6979865771812	22.9948606940893\\
59.7986577181208	22.449434575959\\
59.8993288590604	21.7349350274779\\
60	20\\
60	20\\
59.8393574297189	16.4187153719505\\
59.6787148594378	14.9403965468197\\
59.5180722891566	13.8095264681132\\
59.3574297188755	12.8590808768216\\
59.1967871485944	12.0242945623734\\
59.0361445783133	11.2719179391447\\
58.8755020080321	10.5821794653174\\
58.714859437751	9.9421841975913\\
58.5542168674699	9.34296656024292\\
58.3935742971888	8.77799248412981\\
58.2329317269076	8.24232456794201\\
58.0722891566265	7.73212331435386\\
57.9116465863454	7.24433251013314\\
57.7510040160643	6.77647188610013\\
57.5903614457831	6.32649546109772\\
57.429718875502	5.8926917911003\\
57.2690763052209	5.47361189573873\\
57.1084337349398	5.06801601462201\\
56.9477911646586	4.67483350652354\\
56.7871485943775	4.29313212969599\\
56.6265060240964	3.92209415169901\\
56.4658634538153	3.56099751907979\\
56.3052208835341	3.20920083504668\\
56.144578313253	2.86613124372899\\
55.9839357429719	2.53127456150077\\
55.8232931726908	2.20416716578155\\
55.6626506024096	1.8843892730482\\
55.5020080321285	1.57155932568676\\
55.3413654618474	1.2653292718516\\
55.1807228915663	0.965380570475659\\
55.0200803212851	0.671420789650959\\
54.859437751004	0.383180694009642\\
54.6987951807229	0.100411737771005\\
54.5381526104418	-0.17711610359207\\
54.3775100401606	-0.449616217392787\\
54.2168674698795	-0.717286997787642\\
54.0562248995984	-0.980313286220838\\
53.8955823293173	-1.23886763828229\\
53.7349397590361	-1.49311144190627\\
53.574297188755	-1.7431959077123\\
53.4136546184739	-1.98926294893346\\
53.2530120481928	-2.23144596563045\\
53.0923694779116	-2.46987054563024\\
52.9317269076305	-2.70465509276113\\
52.7710843373494	-2.93591139140394\\
52.6104417670683	-3.16374511508554\\
52.4497991967871	-3.38825628575572\\
52.289156626506	-3.60953968947641\\
52.1285140562249	-3.82768525348106\\
51.9678714859438	-4.04277838890886\\
51.8072289156626	-4.25490030296216\\
51.6465863453815	-4.46412828376094\\
51.4859437751004	-4.6705359607613\\
51.3253012048193	-4.87419354325579\\
51.1646586345382	-5.07516803917187\\
51.004016064257	-5.27352345612483\\
50.8433734939759	-5.46932098645554\\
50.6827309236948	-5.66261917778745\\
50.5220883534137	-5.85347409046627\\
50.3614457831325	-6.04193944309647\\
50.2008032128514	-6.22806674725807\\
50.0401606425703	-6.41190543237218\\
49.8795180722892	-6.59350296158341\\
49.718875502008	-6.77290493943731\\
49.5582329317269	-6.95015521205351\\
49.3975903614458	-7.12529596042465\\
49.2369477911647	-7.29836778741015\\
49.0763052208835	-7.46940979893855\\
48.9156626506024	-7.63845967988369\\
48.7550200803213	-7.80555376503621\\
48.5943775100402	-7.97072710555286\\
48.433734939759	-8.13401353123166\\
48.2730923694779	-8.29544570892938\\
48.1124497991968	-8.45505519740983\\
47.9518072289157	-8.61287249888655\\
47.7911646586345	-8.76892710750025\\
47.6305220883534	-8.9232475549516\\
47.4698795180723	-9.07586145349081\\
47.3092369477912	-9.22679553644905\\
47.14859437751	-9.37607569648183\\
46.9879518072289	-9.5237270216802\\
46.8273092369478	-9.66977382969368\\
46.6666666666667	-9.8142396999972\\
46.5060240963855	-9.9571475044242\\
46.3453815261044	-10.0985194360785\\
46.1847389558233	-10.2383770367292\\
46.0240963855422	-10.3767412227844\\
45.863453815261	-10.5136323099342\\
45.7028112449799	-10.6490700365433\\
45.5421686746988	-10.7830735858724\\
45.3815261044177	-10.9156616071978\\
45.2208835341365	-11.0468522358953\\
45.0602409638554	-11.1766631125511\\
44.8995983935743	-11.3051114011548\\
44.7389558232932	-11.4322138064303\\
44.578313253012	-11.5579865903516\\
44.4176706827309	-11.6824455878915\\
44.2570281124498	-11.8056062220455\\
44.0963855421687	-11.9274835181723\\
43.9357429718875	-12.0480921176867\\
43.7751004016064	-12.1674462911424\\
43.6144578313253	-12.2855599507359\\
43.4538152610442	-12.4024466622632\\
43.2931726907631	-12.518119656559\\
43.1325301204819	-12.6325918404438\\
42.9718875502008	-12.7458758072055\\
42.8112449799197	-12.8579838466405\\
42.6506024096386	-12.9689279546739\\
42.4899598393574	-13.0787198425829\\
42.3293172690763	-13.1873709458422\\
42.1686746987952	-13.2948924326089\\
42.0080321285141	-13.4012952118671\\
41.8473895582329	-13.5065899412461\\
41.6867469879518	-13.6107870345305\\
41.5261044176707	-13.7138966688744\\
41.3654618473896	-13.8159287917357\\
41.2048192771084	-13.916893127543\\
41.0441767068273	-14.0167991841065\\
40.8835341365462	-14.115656258787\\
40.7228915662651	-14.2134734444315\\
40.5622489959839	-14.3102596350881\\
40.4016064257028	-14.4060235315086\\
40.2409638554217	-14.500773646449\\
40.0803212851406	-14.5945183097775\\
39.9196787148594	-14.6872656733963\\
39.7590361445783	-14.7790237159879\\
39.5983935742972	-14.8698002475917\\
39.4377510040161	-14.9596029140188\\
39.2771084337349	-15.0484392011117\\
39.1164658634538	-15.1363164388558\\
38.9558232931727	-15.223241805348\\
38.7951807228916	-15.3092223306298\\
38.6345381526104	-15.394264900389\\
38.4738955823293	-15.4783762595358\\
38.3132530120482	-15.5615630156593\\
38.1526104417671	-15.6438316423677\\
37.9919678714859	-15.7251884825177\\
37.8313253012048	-15.8056397513382\\
37.6706827309237	-15.8851915394503\\
37.5100401606426	-15.9638498157898\\
37.3493975903614	-16.0416204304352\\
37.1887550200803	-16.118509117344\\
37.0281124497992	-16.1945214970027\\
36.8674698795181	-16.2696630789914\\
36.7068273092369	-16.3439392644679\\
36.5461847389558	-16.4173553485745\\
36.3855421686747	-16.4899165227681\\
36.2248995983936	-16.5616278770797\\
36.0642570281125	-16.6324944023027\\
35.9036144578313	-16.7025209921152\\
35.7429718875502	-16.7717124451365\\
35.5823293172691	-16.8400734669215\\
35.421686746988	-16.9076086718955\\
35.2610441767068	-16.9743225852294\\
35.1004016064257	-17.04021964466\\
34.9397590361446	-17.1053042022557\\
34.7791164658635	-17.1695805261297\\
34.6184738955823	-17.2330528021028\\
34.4578313253012	-17.2957251353173\\
34.2971887550201	-17.3576015518037\\
34.136546184739	-17.418686000002\\
33.9759036144578	-17.4789823522382\\
33.8152610441767	-17.5384944061593\\
33.6546184738956	-17.5972258861256\\
33.4939759036145	-17.655180444564\\
33.3333333333333	-17.7123616632825\\
33.1726907630522	-17.7687730547472\\
33.0120481927711	-17.8244180633226\\
32.85140562249	-17.8793000664774\\
32.6907630522088	-17.9334223759562\\
32.5301204819277	-17.9867882389177\\
32.3694779116466	-18.0394008390417\\
32.2088353413655	-18.0912632976044\\
32.0481927710843	-18.1423786745241\\
31.8875502008032	-18.1927499693775\\
31.7269076305221	-18.2423801223878\\
31.566265060241	-18.2912720153856\\
31.4056224899598	-18.3394284727423\\
31.2449799196787	-18.3868522622788\\
31.0843373493976	-18.4335460961482\\
30.9236947791165	-18.4795126316941\\
30.7630522088353	-18.5247544722855\\
30.6024096385542	-18.5692741681289\\
30.4417670682731	-18.6130742170567\\
30.281124497992	-18.6561570652954\\
30.1204819277108	-18.6985251082115\\
29.9598393574297	-18.740180691037\\
29.7991967871486	-18.7811261095747\\
29.6385542168675	-18.8213636108845\\
29.4779116465863	-18.8608953939497\\
29.3172690763052	-18.8997236103256\\
29.1566265060241	-18.9378503647695\\
28.995983935743	-18.9752777158528\\
28.8353413654618	-19.0120076765572\\
28.6746987951807	-19.0480422148523\\
28.5140562248996	-19.0833832542584\\
28.3534136546185	-19.1180326743924\\
28.1927710843373	-19.1519923114986\\
28.0321285140562	-19.185263958964\\
27.8714859437751	-19.2178493678191\\
27.710843373494	-19.2497502472238\\
27.5502008032129	-19.2809682649392\\
27.3895582329317	-19.3115050477861\\
27.2289156626506	-19.3413621820888\\
27.0682730923695	-19.3705412141068\\
26.9076305220884	-19.3990436504532\\
26.7469879518072	-19.4268709584995\\
26.5863453815261	-19.4540245667695\\
26.425702811245	-19.4805058653199\\
26.2650602409639	-19.5063162061093\\
26.1044176706827	-19.5314569033555\\
25.9437751004016	-19.5559292338816\\
25.7831325301205	-19.57973443745\\
25.6224899598394	-19.6028737170866\\
25.4618473895582	-19.6253482393931\\
25.3012048192771	-19.6471591348494\\
25.140562248996	-19.6683074981048\\
24.9799196787149	-19.6887943882596\\
24.8192771084337	-19.7086208291364\\
24.6586345381526	-19.7277878095412\\
24.4979919678715	-19.7462962835151\\
24.3373493975904	-19.7641471705759\\
24.1767068273092	-19.7813413559511\\
24.0160642570281	-19.7978796908004\\
23.855421686747	-19.8137629924297\\
23.6947791164659	-19.828992044496\\
23.5341365461847	-19.8435675972035\\
23.3734939759036	-19.8574903674898\\
23.2128514056225	-19.870761039205\\
23.0522088353414	-19.8833802632809\\
22.8915662650602	-19.895348657892\\
22.7309236947791	-19.9066668086085\\
22.570281124498	-19.9173352685403\\
22.4096385542169	-19.9273545584732\\
22.2489959839357	-19.9367251669969\\
22.0883534136546	-19.9454475506243\\
21.9277108433735	-19.9535221339038\\
21.7670682730924	-19.9609493095225\\
21.6064257028112	-19.9677294384025\\
21.4457831325301	-19.9738628497884\\
21.285140562249	-19.9793498413276\\
21.1244979919679	-19.9841906791429\\
20.9638554216867	-19.9883855978969\\
20.8032128514056	-19.9919348008487\\
20.6425702811245	-19.994838459904\\
20.4819277108434	-19.9970967156558\\
20.3212851405622	-19.998709677419\\
20.1606425702811	-19.999677423257\\
20	-20\\
20	-20\\
18.75	-20\\
17.5	-20\\
16.25	-20\\
15	-20\\
15	-20\\
14.8657718120805	-19.9995495647669\\
14.7315436241611	-19.9981981981945\\
14.5973154362416	-19.9959457176226\\
14.4630872483221	-19.9927918184801\\
14.3288590604027	-19.9887360740792\\
14.1946308724832	-19.9837779353265\\
14.0604026845638	-19.9779167303507\\
13.9261744966443	-19.9711516640464\\
13.7919463087248	-19.9634818175336\\
13.6577181208054	-19.9549061475314\\
13.5234899328859	-19.9454234856448\\
13.3892617449664	-19.9350325375649\\
13.255033557047	-19.9237318821793\\
13.1208053691275	-19.9115199705924\\
12.9865771812081	-19.8983951250539\\
12.8523489932886	-19.8843555377933\\
12.7181208053691	-19.8693992697593\\
12.5838926174497	-19.8535242492608\\
12.4496644295302	-19.8367282705086\\
12.3154362416107	-19.8190089920546\\
12.1812080536913	-19.8003639351257\\
12.0469798657718	-19.7807904818499\\
11.9127516778523	-19.7602858733723\\
11.7785234899329	-19.738847207856\\
11.6442953020134	-19.7164714383662\\
11.510067114094	-19.6931553706325\\
11.3758389261745	-19.6688956606864\\
11.241610738255	-19.6436888123692\\
11.1073825503356	-19.6175311747062\\
10.9731543624161	-19.5904189391414\\
10.8389261744966	-19.56234813663\\
10.7046979865772	-19.5333146345797\\
10.5704697986577	-19.5033141336388\\
10.4362416107383	-19.4723421643223\\
10.3020134228188	-19.4403940834703\\
10.1677852348993	-19.4074650705326\\
10.0335570469799	-19.3735501236711\\
9.8993288590604	-19.3386440556722\\
9.76510067114094	-19.3027414896618\\
9.63087248322148	-19.2658368546132\\
9.49664429530201	-19.2279243806389\\
9.36241610738255	-19.1889980940564\\
9.22818791946309	-19.1490518122169\\
9.09395973154362	-19.1080791380864\\
8.95973154362416	-19.0660734545661\\
8.8255033557047	-19.0230279185409\\
8.69127516778524	-18.9789354546401\\
8.55704697986577	-18.9337887486985\\
8.42281879194631	-18.8875802408997\\
8.28859060402685	-18.8403021185873\\
8.15436241610738	-18.7919463087248\\
8.02013422818792	-18.7425044699871\\
7.88590604026846	-18.6919679844609\\
7.75167785234899	-18.6403279489357\\
7.61744966442953	-18.5875751657597\\
7.48322147651007	-18.5337001332384\\
7.3489932885906	-18.4786930355469\\
7.21476510067114	-18.4225437321308\\
7.08053691275168	-18.3652417465633\\
6.94630872483221	-18.3067762548267\\
6.81208053691275	-18.2471360729841\\
6.67785234899329	-18.1863096442033\\
6.54362416107383	-18.1242850250934\\
6.40939597315436	-18.0610498713099\\
6.2751677852349	-17.9965914223832\\
6.14093959731544	-17.9308964857195\\
6.00671140939597	-17.8639514197199\\
5.87248322147651	-17.7957421159606\\
5.73825503355705	-17.7262539803696\\
5.60402684563758	-17.6554719133333\\
5.46979865771812	-17.5833802886581\\
5.33557046979866	-17.5099629313078\\
5.20134228187919	-17.4352030938304\\
5.06711409395973	-17.3590834313793\\
4.93288590604027	-17.281585975228\\
4.79865771812081	-17.2026921046662\\
4.66442953020134	-17.1223825171565\\
4.53020134228188	-17.0406371966204\\
4.39597315436242	-16.9574353797087\\
4.26174496644295	-16.8727555198991\\
4.12751677852349	-16.7865752492494\\
3.99328859060403	-16.6988713376166\\
3.85906040268456	-16.6096196491361\\
3.7248322147651	-16.5187950957327\\
3.59060402684564	-16.426371587413\\
3.45637583892617	-16.3323219790657\\
3.32214765100671	-16.236618013463\\
3.18791946308725	-16.1392302601297\\
3.05369127516779	-16.040128049707\\
2.91946308724832	-15.9392794033998\\
2.78523489932886	-15.8366509570498\\
2.6510067114094	-15.7322078793265\\
2.51677852348993	-15.6259137834687\\
2.38255033557047	-15.5177306319445\\
2.24832214765101	-15.4076186333227\\
2.11409395973154	-15.2955361305635\\
1.97986577181208	-15.1814394798375\\
1.84563758389262	-15.0652829188735\\
1.71140939597315	-14.9470184237048\\
1.57718120805369	-14.8265955525391\\
1.44295302013423	-14.7039612753065\\
1.30872483221476	-14.5790597872424\\
1.1744966442953	-14.4518323046387\\
1.04026845637584	-14.3222168406272\\
0.906040268456376	-14.190147958553\\
0.771812080536913	-14.0555565001335\\
0.63758389261745	-13.9183692851712\\
0.503355704697986	-13.778508779091\\
0.369127516778523	-13.6358927239735\\
0.234899328859061	-13.4904337280561\\
0.100671140939598	-13.3420388078273\\
-0.0335570469798654	-13.1906088758329\\
-0.167785234899329	-13.0360381660982\\
-0.302013422818792	-12.8782135876011\\
-0.436241610738255	-12.7170139944451\\
-0.570469798657719	-12.5523093592011\\
-0.704697986577182	-12.3839598332035\\
-0.838926174496644	-12.2118146742753\\
-0.973154362416107	-12.0357110182335\\
-1.10738255033557	-11.85547246537\\
-1.24161073825503	-11.6709074465955\\
-1.3758389261745	-11.4818073256778\\
-1.51006711409396	-11.2879441834248\\
-1.64429530201342	-11.0890682160123\\
-1.77852348993289	-10.8849046618595\\
-1.91275167785235	-10.6751501480449\\
-2.04697986577181	-10.4594683161225\\
-2.18120805369128	-10.2374845453253\\
-2.31543624161074	-10.0087795341247\\
-2.4496644295302	-9.77288142242545\\
-2.58389261744966	-9.52925602646918\\
-2.71812080536913	-9.27729460167918\\
-2.85234899328859	-9.01629832147472\\
-2.98657718120806	-8.74545832443593\\
-3.12080536912752	-8.46382967540104\\
-3.25503355704698	-8.17029680189702\\
-3.38926174496644	-7.86352671993386\\
-3.52348993288591	-7.54190431564035\\
-3.65771812080537	-7.20344046442885\\
-3.79194630872483	-6.84563758389262\\
-3.9261744966443	-6.46528567988848\\
-4.06040268456376	-6.05813911215933\\
-4.19463087248322	-5.61837571448493\\
-4.32885906040269	-5.13762663328983\\
-4.46308724832215	-4.60306797301538\\
-4.59731543624161	-3.99314759211906\\
-4.73154362416107	-3.26591276794536\\
-4.86577181208054	-2.31324670330389\\
-5	-0\\
};

\node[right, align=left]
at (axis cs:1,0) {$\leftarrow\text{ finish/start}$};
\addplot [color=red]
  table[row sep=crcr]{%
-2.5	0\\
-2.50000434304876	0.16104733037824\\
-2.50003501767018	0.592433252200396\\
-2.50011269744499	1.25000639169794\\
-2.50026611494843	2.08914287335502\\
-2.50053090240134	3.07443355359123\\
-2.50094232630881	4.19593794301175\\
-2.5015458360844	5.44242182896933\\
-2.5023861451317	6.8013003926219\\
-2.50350901673793	8.26005902893635\\
-2.50495577711865	9.81175142134889\\
-2.50675411964068	11.4532984139585\\
-2.50892448388566	13.1813819962147\\
-2.51149698391108	14.9928364291136\\
-2.51451188993621	16.8840927951956\\
-2.51801174214229	18.8513579130648\\
-2.52203116747758	20.8907592796854\\
-2.52659604588063	22.9988723972649\\
-2.53173892954965	25.173857331165\\
-2.53749384944114	27.4143350382192\\
-2.54388402002007	29.718866931602\\
-2.5509291347488	32.0862227887059\\
-2.5586562966307	34.514948477356\\
-2.56710080085691	37.0034765902002\\
-2.5762962041022	39.5502105111005\\
-2.58627627400833	42.1535293821034\\
-2.59707838426438	44.811793107885\\
-2.60873489971928	47.5233483018276\\
-2.62127518244725	50.2871755367837\\
-2.63473148029349	53.10277979247\\
-2.64913952898472	55.9694086747284\\
-2.66454311807522	58.8864963908789\\
-2.68099232187231	61.8533135908527\\
-2.69853760215136	64.8690549992787\\
-2.71722300518789	67.9329043089297\\
-2.73708330394528	71.0440352718348\\
-2.7581405157462	74.2016128165141\\
-2.78039912196609	77.4047941928491\\
-2.80384831227795	80.6527300646217\\
-2.82846392471334	83.9445656018728\\
-2.85420243296236	87.2794416391745\\
-2.88099353378785	90.6564958381507\\
-2.90872866194661	94.0749482104396\\
-2.93724676764693	97.5341482115895\\
-2.96631663203454	101.033236922388\\
-2.99562813333319	104.57134507101\\
-3.0247945170155	108.147604869301\\
-3.0533401025709	111.761150796697\\
-3.08067861494135	115.411120209883\\
-3.10611181472887	119.096653575458\\
-3.12882124168089	122.816894428065\\
-3.14783839577749	126.570988870401\\
-3.16204825986905	130.358084581487\\
-3.17020669186452	134.177329505645\\
-3.1709171881654	138.027869654191\\
-3.16265702635582	141.908847026661\\
-3.14382258600193	145.819436972906\\
-3.11271753519441	149.75896736751\\
-3.06760153053367	153.726576036576\\
-3.00677262908265	157.721362832681\\
-2.92860605685777	161.742421001319\\
-2.8316197585692	165.788841835773\\
-2.71460100338701	169.859725914511\\
-2.57668664361873	173.954194709817\\
-2.41743067054376	178.071403524577\\
-2.23693526464123	182.210561153052\\
-2.03605804772981	186.370274356184\\
-1.82153279040895	190.47090926953\\
-1.59993114847473	194.448541309141\\
-1.37469568863419	198.303288999651\\
-1.15018971291585	202.035486523488\\
-0.931623641055622	205.645502459353\\
-0.724812011434218	209.133676770896\\
-0.535911197589214	212.500247074047\\
-0.370962534140666	215.745270764039\\
-0.235403553683077	218.86856720473\\
-0.135396358497349	221.868043138242\\
-0.0742296698697296	224.737125131328\\
-0.055212846605588	227.46994328304\\
-0.0811461442148769	230.065488246332\\
-0.155007201159079	232.525511608356\\
-0.273275027851965	234.855275497393\\
-0.421746096976666	237.058622639636\\
-0.605649528886917	239.133565584005\\
-0.825339987820848	241.078488729471\\
-1.07654801228027	242.892999798555\\
-1.35561714584212	244.576475041427\\
-1.66249493199461	246.126286957731\\
-1.99612158440977	247.537616148603\\
-2.34957872041673	248.803844147331\\
-2.71413376224882	249.920489040637\\
-3.08349315079103	250.886875811107\\
-3.45072693356977	251.703174312237\\
-3.80791765946599	252.370262673005\\
-4.14670567274281	252.891204207713\\
-4.50864623502675	253.351954261405\\
-4.92323235313689	253.803913342287\\
-5.38056799660124	254.228290211102\\
-5.87382161635496	254.613277647118\\
-6.39789640333304	254.951970221406\\
-6.94886155428749	255.240409962619\\
-7.52225706695264	255.475444511979\\
-8.11351703440129	255.65474409342\\
-8.71782758656211	255.776583592764\\
-9.32999061636193	255.839725724104\\
-9.94446375265539	255.843450590731\\
-10.5556038930792	255.787626564465\\
-11.1580283065173	255.672717491615\\
-11.7461979561795	255.499733859215\\
-12.3145986439935	255.270246384777\\
-12.8578399894215	254.986380944013\\
-13.3707458489915	254.650799456277\\
-13.8483070009263	254.266732513195\\
-14.2847382264994	253.838712878477\\
-14.6816683345067	253.365369242869\\
-15.0802069419576	252.814450090741\\
-15.5455082935656	252.172266113235\\
-16.1026340326918	251.434202459017\\
-16.7550328819315	250.612361104169\\
-17.5043623807544	249.723793413894\\
-18.2460648439003	248.843563428951\\
-18.9162228309623	248.047686968787\\
-19.5344698340988	247.359269452557\\
-20.1059786501861	246.788275950113\\
-20.6290274916642	246.338582487602\\
-21.1006277732453	246.00797563887\\
-21.6035688902265	245.719815571837\\
-22.1577076050592	245.458798829626\\
-22.7438926815495	245.241218912906\\
-23.3483358186563	245.072208100733\\
-23.96280158995	244.954104432645\\
-24.5823040000165	244.889468203118\\
-25.2023759632066	244.879607721058\\
-25.8190219794213	244.924825866716\\
-26.427715805756	245.024855133894\\
-27.0238926541252	245.178860590743\\
-27.6032968725203	245.385602613521\\
-28.1614363958568	245.643438733833\\
-28.6939139223191	245.950318158467\\
-29.196477722069	246.303792013725\\
-29.6650684684561	246.701036284059\\
-30.0958153127256	247.138861426494\\
-30.4851776279824	247.613538574135\\
-30.8542981471063	248.145005162832\\
-31.2621628176171	248.806629941154\\
-31.6991574064821	249.608590599469\\
-32.1458262944908	250.542934167584\\
-32.5859299192447	251.602086369954\\
-33.0069470254382	252.777385421718\\
-33.3999691799832	254.057748404782\\
-33.7584774376353	255.434322896269\\
-34.0785972773326	256.910662048538\\
-34.3544573685719	258.484046625893\\
-34.5809119061637	260.149778872322\\
-34.7551085207008	261.902969573623\\
-34.876153719014	263.7388288714\\
-34.9441759282351	265.652364686066\\
-34.9611795851212	267.638984654663\\
-34.9322454758543	269.694610608875\\
-34.8654345112879	271.816263242255\\
-34.7702612075714	274.002803793867\\
-34.6564335289619	276.253453302316\\
-34.5330503218191	278.567426817532\\
-34.4076058781473	280.943936618583\\
-34.2854606462534	283.381720188645\\
-34.1703901495036	285.879310295788\\
-34.0652562731652	288.435129738034\\
-33.972393706026	291.047529145457\\
-33.8938158001773	293.714809863538\\
-33.8312652902463	296.435256405095\\
-33.7861422008324	299.207955507836\\
-33.7594349548451	302.032212000049\\
-33.7517207260333	304.907164209306\\
-33.7631958886681	307.832134130856\\
-33.793698367919	310.806263908963\\
-33.8427024325625	313.828673413388\\
-33.9092944771267	316.89850407533\\
-33.9921303076708	320.014925516333\\
-34.0893458514176	323.177142917198\\
-34.1984526416946	326.384401992346\\
-34.3162742472599	329.6359861484\\
-34.4389139585306	332.931204230171\\
-34.5617057515973	336.269368879645\\
-34.679203523735	339.649761307364\\
-34.7852468960901	343.07169607943\\
-34.8730940809665	346.5344260274\\
-34.9356461331322	350.036810173471\\
-34.9657785021248	353.577456934839\\
-34.9567327518811	357.154713097769\\
-34.9024388438493	360.766686543022\\
-34.7977322021241	364.411303785688\\
-34.638451931065	368.086390471147\\
-34.4216471918593	371.787304928308\\
-34.1511166551178	375.460448516808\\
-33.839681168608	379.010495617952\\
-33.4899018014161	382.422865920943\\
-33.102752944199	385.694952518279\\
-32.6785521709381	388.827937808905\\
-32.222391503357	391.826818955917\\
-31.7410227253384	394.693835377041\\
-31.2337355278092	397.427682871011\\
-30.7021434017241	400.02792155373\\
-30.1505114237799	402.495256659631\\
-29.5818434358584	404.829709047556\\
-28.997743480422	407.030334688659\\
-28.3991693317234	409.095419822921\\
-27.7866171772594	411.02226836458\\
-27.1617399280238	412.808401716826\\
-26.5275721950489	414.452108418515\\
-25.8882902487174	415.952610200598\\
-25.2489330560805	417.309983359687\\
-24.6152706591187	418.524886798994\\
-23.9935899767506	419.598274183426\\
-23.3893420694438	420.532395539841\\
-22.7610008657885	421.392247066719\\
-22.0835693188882	422.222696598376\\
-21.366162441657	423.017022625089\\
-20.6112783914274	423.773960140023\\
-19.8177318677177	424.494880101137\\
-18.9850871760725	425.178084263078\\
-18.1149225920552	425.81963298127\\
-17.2100768300383	426.415865309355\\
-16.2721555740804	426.965192397012\\
-15.2992066037831	427.468016888766\\
-14.2839893897193	427.926493035951\\
-13.2165046589262	428.342812358742\\
-12.0899878353179	428.716756550651\\
-10.9079439359575	429.044424651752\\
-9.69088184451813	429.318851532882\\
-8.47996661956121	429.532573271698\\
-7.31392749541215	429.683161357648\\
-6.20770989225377	429.773095340717\\
-5.14741229414892	429.805753195128\\
-4.10055307129393	429.782234968093\\
-3.05435842795979	429.701886140543\\
-2.01072100178908	429.563632039034\\
-0.972788379575986	429.366072444701\\
0.0557480485805666	429.108220045852\\
1.07129119515285	428.790494878722\\
2.07097081685627	428.414992903804\\
3.05125008952437	427.983616672049\\
4.00822392564165	427.497385092275\\
4.93797217251875	426.957693652547\\
5.84172767551464	426.363404628974\\
6.76697926225755	425.678512603662\\
7.72268865987445	424.888953535566\\
8.69723022737316	423.997117078136\\
9.67972717341997	423.006413876006\\
10.6634291077538	421.916696278191\\
11.6420805104973	420.7273544968\\
12.6066675362486	419.440858888048\\
13.5527403059534	418.060553639614\\
14.4827065934538	416.594543351484\\
15.3905275341922	415.044742053392\\
16.2701187078061	413.413163846449\\
17.1167051195172	411.70284903088\\
17.9258938845378	409.917108445688\\
18.6948098686619	408.05898292981\\
19.4225361447224	406.131334897633\\
20.1082403971789	404.136283682736\\
20.7498992424801	402.076128215804\\
21.3439902919288	399.953251406264\\
21.886199611095	397.769493593805\\
22.3725752618856	395.52692091905\\
22.8016430677745	393.225967070589\\
23.1768941083186	390.859450659641\\
23.4973229324833	388.428914142315\\
23.7597837033407	385.93731449346\\
23.962010857014	383.387615368995\\
24.1023971642055	380.782822450827\\
24.1802576376053	378.125904092035\\
24.1958962699593	375.419618885117\\
24.1503937383288	372.665956251862\\
24.0452071433136	369.865974852517\\
23.8823280291801	367.020884056082\\
23.6652443105363	364.131405390295\\
23.4001367089597	361.197682064225\\
23.0963037449987	358.219393991272\\
22.7654107184006	355.195759153747\\
22.420722718604	352.12581459038\\
22.0751783218904	349.008873920535\\
21.73887851358	345.844845276061\\
21.4203899184623	342.65791252107\\
21.1340348031155	339.572996328007\\
20.8819280448204	336.608388696315\\
20.6620433313506	333.764355526509\\
20.4716219450159	331.041158962932\\
20.3085736210225	328.439025231239\\
20.1743672266683	325.958302830748\\
20.0786926220775	323.60214447781\\
20.0344011440769	321.373933843733\\
20.0452431527416	319.274063796015\\
20.109972829793	317.30333507728\\
20.2237229178543	315.462727441558\\
20.3816774806037	313.753921195885\\
20.5794371733435	312.178851836505\\
20.8116714574422	310.738819095852\\
21.0800862869906	309.4108314723\\
21.4090388076992	308.10559402058\\
21.7991172529484	306.817103366878\\
22.2480858947572	305.550390017823\\
22.7508954846363	304.310698312694\\
23.3002499291666	303.104318230382\\
23.8876583724499	301.93876231793\\
24.5027651524926	300.824224130199\\
25.1332110883818	299.776426657421\\
25.7607757996645	298.817586337772\\
26.3670836341666	297.962975553082\\
26.9419997609506	297.214625245059\\
27.4724082517676	296.573930979033\\
27.9495499652997	296.034791513359\\
28.3665300747125	295.55060352833\\
28.782953562523	295.025405529006\\
29.2472812701581	294.388214080657\\
29.7620359226734	293.622233034167\\
30.3042758862227	292.730429775225\\
30.8551900403699	291.715527208622\\
31.3991713074407	290.584754937031\\
31.9239675470456	289.347402687536\\
32.4196816044277	288.012095448153\\
32.8780757461663	286.585385092215\\
33.2928222989327	285.072019963109\\
33.65863903872	283.476889931842\\
33.9711301394631	281.805011937851\\
34.2257541528624	280.059542144212\\
34.4184736778211	278.237133707972\\
34.5674584072719	276.336005378402\\
34.6703190880079	274.360019350655\\
34.7201830866015	272.313712222667\\
34.724978020507	270.200569907354\\
34.6942082556283	268.022182596812\\
34.6336302281186	265.779819162474\\
34.5489627769346	263.47463109819\\
34.4459584858991	261.107576854161\\
34.3301135256145	258.679843222553\\
34.2065745727549	256.192694845378\\
34.0799150554597	253.647470900008\\
33.9536693787073	251.045625395702\\
33.8302034771446	248.388709409896\\
33.7110247554329	245.678334308094\\
33.5970942390699	242.915470951293\\
33.4890940647302	240.100629153848\\
33.3875874877925	237.234565739342\\
33.2930245691288	234.317856163262\\
33.2056908069306	231.351250388\\
33.1256460464878	228.335574547525\\
33.0527709382616	225.271666974462\\
32.9868282338783	222.160374937695\\
32.9274788384133	219.002552513793\\
32.8743127397314	215.799058967492\\
32.8268825883766	212.550757642404\\
32.7846781788586	209.258515164458\\
32.7471131791067	205.92320055329\\
32.7135575343747	202.545684310791\\
32.683330302199	199.126751071957\\
32.6556581993741	195.667059351757\\
32.6297123406723	192.167475425456\\
32.6046390176676	188.628875739142\\
32.579512917544	185.052136707619\\
32.5533277315605	181.438134779024\\
32.5250239293392	177.787746340297\\
32.4934865742085	174.101847980937\\
32.4575630172187	170.381316576602\\
32.4160729931475	166.627029510934\\
32.3677915429591	162.839865540789\\
32.3114970797037	159.020703966168\\
32.2460286863892	155.170422757876\\
32.1702930743585	151.289897672929\\
32.0833128619919	147.379958759088\\
31.9843119179451	143.441265571085\\
31.8727567833208	139.474643016172\\
31.748434118964	135.480920715868\\
31.6115612022323	131.46089400518\\
31.4628501980245	127.415316109366\\
31.3035974646744	123.34489032504\\
31.1357810129949	119.250264720931\\
30.962119378905	115.132034149282\\
30.7861648576209	110.990748170521\\
30.6123134288991	106.826932772449\\
30.4461546275849	102.653254781696\\
30.2962705187031	98.5602969529593\\
30.169082435143	94.5894371152109\\
30.069574022849	90.7409215243673\\
30.0022340354668	87.015198189622\\
29.9708619173273	83.4130121207812\\
29.9792842733179	79.9386408579673\\
30.0301710381133	76.5997946161028\\
30.1260641708301	73.3991981436046\\
30.2696606226196	70.3336250331351\\
30.4488562223516	67.3962197031215\\
30.6587720579619	64.5860245547667\\
30.9038961906229	61.9046669900246\\
31.1834591893657	59.3530211970199\\
31.4942221564404	56.9313646609421\\
31.8331047673881	54.6400103028931\\
32.1976871723899	52.4794675301433\\
32.5857701150802	50.4503653329391\\
32.9964947581033	48.5541455615977\\
33.4311809439914	46.7950696033785\\
33.8862373035067	45.1787778016803\\
34.3538132411959	43.7101413974734\\
34.8258586458712	42.3921569808359\\
35.2966441676035	41.2251039009643\\
35.760315010772	40.2085245342856\\
36.2103397082847	39.3413230998619\\
36.6393820157439	38.6217257128965\\
37.0414321997067	38.044709150107\\
37.4841717084421	37.4968209544068\\
38.0209084509621	36.904019681585\\
38.6609128398716	36.2678721074373\\
39.400180401177	35.6071876471133\\
40.2380985038725	34.934426969712\\
41.1722455432313	34.2612852094222\\
42.198782654868	33.5983479356211\\
43.3146893936292	32.9539670211612\\
44.5097542666702	32.3139482967364\\
45.7631363215968	31.645617080021\\
47.0374953758017	30.9531846756533\\
48.2346345456209	30.2837280051236\\
49.3436444643829	29.6423895975214\\
50.352965939284	29.0153161419394\\
51.2584244489559	28.3985469577176\\
52.0598241552976	27.7938526603578\\
52.7564380008451	27.2046565753139\\
53.3478475931798	26.635577459383\\
53.8343087980538	26.0923326163064\\
54.2173614710088	25.5810749726267\\
54.5335920229387	25.0653685620237\\
54.8395620969733	24.4770784986388\\
55.1112796415766	23.8415404878315\\
55.3597803770744	23.1289208475695\\
55.6083495520226	22.3198422987254\\
55.8730892255524	21.4267241304561\\
56.1355698277679	20.4729919108497\\
56.3541463778558	19.5383062089688\\
56.5121103662451	18.6696862117306\\
56.6121452297268	17.8709812533438\\
56.6609715567314	17.1275743096629\\
56.6630342520553	16.4044986989984\\
56.6169282588326	15.6529519120177\\
56.5152389380438	14.8198063679944\\
56.3482957272278	13.9023067940812\\
56.1081782190577	12.91308267657\\
55.7935603831121	11.8613900994665\\
55.4077973826232	10.7559435896951\\
54.9504069132782	9.60686700253449\\
54.4203293198012	8.42162010145255\\
53.8164950582041	7.20519747079664\\
53.138710039533	5.96200519708245\\
52.386811093496	4.69468014956592\\
51.5611654145981	3.40765086076158\\
50.6625970356158	2.10647947860386\\
49.6917154370328	0.795369541151979\\
48.6494784025261	-0.521830867663537\\
47.5369590588209	-1.84141991435771\\
46.3553990930343	-3.15987001550887\\
45.106204124807	-4.47419856379261\\
43.7903964600734	-5.78230992030015\\
42.4091890791146	-7.08241316111936\\
40.9640368961064	-8.37266253767228\\
39.4568927259715	-9.65074650836806\\
37.9067052642462	-10.9013596848951\\
36.3879365456019	-12.0707367733399\\
34.9194068713828	-13.1523504245882\\
33.5051531460343	-14.1480292214111\\
32.1467164054754	-15.0583814270796\\
30.8450089407162	-15.883131091319\\
29.6020371996389	-16.6234197281964\\
28.4200622972808	-17.2807742823882\\
27.3011869918374	-17.8563567965636\\
26.2477865918999	-18.3516842183338\\
25.2625247541456	-18.7687817822378\\
24.3481963881513	-19.1100762525045\\
23.5074918822601	-19.3784103006238\\
22.7418936973382	-19.5774270446101\\
22.0441959885795	-19.7132625627961\\
21.3632647186377	-19.8009668469954\\
20.672813791512	-19.8472534115728\\
19.9511282147453	-19.8534478338539\\
19.1400771455569	-19.8213322157918\\
18.2334310275695	-19.7425988077854\\
17.2559410579955	-19.6111492044685\\
16.3116942543325	-19.4366740920153\\
15.4649654117725	-19.2330561092856\\
14.7186625328994	-19.0076458914571\\
14.0415507425327	-18.757402941637\\
13.3268212102874	-18.4449462780417\\
12.5375810024076	-18.0471496840726\\
11.6933590727165	-17.5618905457092\\
10.8050794028067	-16.9875269203615\\
9.88146132883091	-16.3272003513652\\
8.93174282801823	-15.5841706431571\\
7.9648955767128	-14.7608702773978\\
6.9866084939039	-13.8580292927418\\
6.00136539853772	-12.8764361044279\\
5.01302591796328	-11.8170274809488\\
4.02443028060102	-10.6798192027553\\
3.04114092157657	-9.4677969622676\\
2.06748757141518	-8.18241915851524\\
1.10732617668911	-6.82571272158247\\
0.164374981930582	-5.39977105084543\\
-0.757873710890724	-3.90684997058361\\
-1.65596977545573	-2.349188068467\\
-2.5261438069305	-0.728002118399609\\
-3.36396103959255	0.95600992679205\\
-4.1656173747443	2.70173063469576\\
-4.92937132560237	4.50668985126214\\
nan	nan\\
nan	nan\\
nan	nan\\
nan	nan\\
nan	nan\\
nan	nan\\
nan	nan\\
nan	nan\\
nan	nan\\
nan	nan\\
nan	nan\\
nan	nan\\
nan	nan\\
nan	nan\\
nan	nan\\
nan	nan\\
nan	nan\\
nan	nan\\
nan	nan\\
nan	nan\\
nan	nan\\
nan	nan\\
nan	nan\\
nan	nan\\
nan	nan\\
nan	nan\\
};

\addplot [color=red]
  table[row sep=crcr]{%
-2.5	0\\
-2.50000434304876	0.16104733037824\\
-2.50003501767018	0.592433252200396\\
-2.50011269744499	1.25000639169794\\
-2.50026611494843	2.08914287335502\\
-2.50053090240134	3.07443355359123\\
-2.50094232630881	4.19593794301175\\
-2.5015458360844	5.44242182896933\\
-2.5023861451317	6.8013003926219\\
-2.50350901673793	8.26005902893635\\
-2.50495577711865	9.81175142134889\\
-2.50675411964068	11.4532984139585\\
-2.50892448388566	13.1813819962147\\
-2.51149698391108	14.9928364291136\\
-2.51451188993621	16.8840927951956\\
-2.51801174214229	18.8513579130648\\
-2.52203116747758	20.8907592796854\\
-2.52659604588063	22.9988723972649\\
-2.53173892954965	25.173857331165\\
-2.53749384944114	27.4143350382192\\
-2.54388402002007	29.718866931602\\
-2.5509291347488	32.0862227887059\\
-2.5586562966307	34.514948477356\\
-2.56710080085691	37.0034765902002\\
-2.5762962041022	39.5502105111005\\
-2.58627627400833	42.1535293821034\\
-2.59707838426438	44.811793107885\\
-2.60873489971928	47.5233483018276\\
-2.62127518244725	50.2871755367837\\
-2.63473148029349	53.10277979247\\
-2.64913952898472	55.9694086747284\\
-2.66454311807522	58.8864963908789\\
-2.68099232187231	61.8533135908527\\
-2.69853760215136	64.8690549992787\\
-2.71722300518789	67.9329043089297\\
-2.73708330394528	71.0440352718348\\
-2.7581405157462	74.2016128165141\\
-2.78039912196609	77.4047941928491\\
-2.80384831227795	80.6527300646217\\
-2.82846392471334	83.9445656018728\\
-2.85420243296236	87.2794416391745\\
-2.88099353378785	90.6564958381507\\
-2.90872866194661	94.0749482104396\\
-2.93724676764693	97.5341482115895\\
-2.96631663203454	101.033236922388\\
-2.99562813333319	104.57134507101\\
-3.0247945170155	108.147604869301\\
-3.0533401025709	111.761150796697\\
-3.08067861494135	115.411120209883\\
-3.10611181472887	119.096653575458\\
-3.12882124168089	122.816894428065\\
-3.14783839577749	126.570988870401\\
-3.16204825986905	130.358084581487\\
-3.17020669186452	134.177329505645\\
-3.1709171881654	138.027869654191\\
-3.16265702635582	141.908847026661\\
-3.14382258600193	145.819436972906\\
-3.11271753519441	149.75896736751\\
-3.06760153053367	153.726576036576\\
-3.00677262908265	157.721362832681\\
-2.92860605685777	161.742421001319\\
-2.8316197585692	165.788841835773\\
-2.71460100338701	169.859725914511\\
-2.57668664361873	173.954194709817\\
-2.41743067054376	178.071403524577\\
-2.23693526464123	182.210561153052\\
-2.03605804772981	186.370274356184\\
-1.82153279040895	190.47090926953\\
-1.59993114847473	194.448541309141\\
-1.37469568863419	198.303288999651\\
-1.15018971291585	202.035486523488\\
-0.931623641055622	205.645502459353\\
-0.724812011434218	209.133676770896\\
-0.535911197589214	212.500247074047\\
-0.370962534140666	215.745270764039\\
-0.235403553683077	218.86856720473\\
-0.135396358497349	221.868043138242\\
-0.0742296698697296	224.737125131328\\
-0.055212846605588	227.46994328304\\
-0.0811461442148769	230.065488246332\\
-0.155007201159079	232.525511608356\\
-0.273275027851965	234.855275497393\\
-0.421746096976666	237.058622639636\\
-0.605649528886917	239.133565584005\\
-0.825339987820848	241.078488729471\\
-1.07654801228027	242.892999798555\\
-1.35561714584212	244.576475041427\\
-1.66249493199461	246.126286957731\\
-1.99612158440977	247.537616148603\\
-2.34957872041673	248.803844147331\\
-2.71413376224882	249.920489040637\\
-3.08349315079103	250.886875811107\\
-3.45072693356977	251.703174312237\\
-3.80791765946599	252.370262673005\\
-4.14670567274281	252.891204207713\\
-4.50864623502675	253.351954261405\\
-4.92323235313689	253.803913342287\\
-5.38056799660124	254.228290211102\\
-5.87382161635496	254.613277647118\\
-6.39789640333304	254.951970221406\\
-6.94886155428749	255.240409962619\\
-7.52225706695264	255.475444511979\\
-8.11351703440129	255.65474409342\\
-8.71782758656211	255.776583592764\\
-9.32999061636193	255.839725724104\\
-9.94446375265539	255.843450590731\\
-10.5556038930792	255.787626564465\\
-11.1580283065173	255.672717491615\\
-11.7461979561795	255.499733859215\\
-12.3145986439935	255.270246384777\\
-12.8578399894215	254.986380944013\\
-13.3707458489915	254.650799456277\\
-13.8483070009263	254.266732513195\\
-14.2847382264994	253.838712878477\\
-14.6816683345067	253.365369242869\\
-15.0802069419576	252.814450090741\\
-15.5455082935656	252.172266113235\\
-16.1026340326918	251.434202459017\\
-16.7550328819315	250.612361104169\\
-17.5043623807544	249.723793413894\\
-18.2460648439003	248.843563428951\\
-18.9162228309623	248.047686968787\\
-19.5344698340988	247.359269452557\\
-20.1059786501861	246.788275950113\\
-20.6290274916642	246.338582487602\\
-21.1006277732453	246.00797563887\\
-21.6035688902265	245.719815571837\\
-22.1577076050592	245.458798829626\\
-22.7438926815495	245.241218912906\\
-23.3483358186563	245.072208100733\\
-23.96280158995	244.954104432645\\
-24.5823040000165	244.889468203118\\
-25.2023759632066	244.879607721058\\
-25.8190219794213	244.924825866716\\
-26.427715805756	245.024855133894\\
-27.0238926541252	245.178860590743\\
-27.6032968725203	245.385602613521\\
-28.1614363958568	245.643438733833\\
-28.6939139223191	245.950318158467\\
-29.196477722069	246.303792013725\\
-29.6650684684561	246.701036284059\\
-30.0958153127256	247.138861426494\\
-30.4851776279824	247.613538574135\\
-30.8542981471063	248.145005162832\\
-31.2621628176171	248.806629941154\\
-31.6991574064821	249.608590599469\\
-32.1458262944908	250.542934167584\\
-32.5859299192447	251.602086369954\\
-33.0069470254382	252.777385421718\\
-33.3999691799832	254.057748404782\\
-33.7584774376353	255.434322896269\\
-34.0785972773326	256.910662048538\\
-34.3544573685719	258.484046625893\\
-34.5809119061637	260.149778872322\\
-34.7551085207008	261.902969573623\\
-34.876153719014	263.7388288714\\
-34.9441759282351	265.652364686066\\
-34.9611795851212	267.638984654663\\
-34.9322454758543	269.694610608875\\
-34.8654345112879	271.816263242255\\
-34.7702612075714	274.002803793867\\
-34.6564335289619	276.253453302316\\
-34.5330503218191	278.567426817532\\
-34.4076058781473	280.943936618583\\
-34.2854606462534	283.381720188645\\
-34.1703901495036	285.879310295788\\
-34.0652562731652	288.435129738034\\
-33.972393706026	291.047529145457\\
-33.8938158001773	293.714809863538\\
-33.8312652902463	296.435256405095\\
-33.7861422008324	299.207955507836\\
-33.7594349548451	302.032212000049\\
-33.7517207260333	304.907164209306\\
-33.7631958886681	307.832134130856\\
-33.793698367919	310.806263908963\\
-33.8427024325625	313.828673413388\\
-33.9092944771267	316.89850407533\\
-33.9921303076708	320.014925516333\\
-34.0893458514176	323.177142917198\\
-34.1984526416946	326.384401992346\\
-34.3162742472599	329.6359861484\\
-34.4389139585306	332.931204230171\\
-34.5617057515973	336.269368879645\\
-34.679203523735	339.649761307364\\
-34.7852468960901	343.07169607943\\
-34.8730940809665	346.5344260274\\
-34.9356461331322	350.036810173471\\
-34.9657785021248	353.577456934839\\
-34.9567327518811	357.154713097769\\
-34.9024388438493	360.766686543022\\
-34.7977322021241	364.411303785688\\
-34.638451931065	368.086390471147\\
-34.4216471918593	371.787304928308\\
-34.1511166551178	375.460448516808\\
-33.839681168608	379.010495617952\\
-33.4899018014161	382.422865920943\\
-33.102752944199	385.694952518279\\
-32.6785521709381	388.827937808905\\
-32.222391503357	391.826818955917\\
-31.7410227253384	394.693835377041\\
-31.2337355278092	397.427682871011\\
-30.7021434017241	400.02792155373\\
-30.1505114237799	402.495256659631\\
-29.5818434358584	404.829709047556\\
-28.997743480422	407.030334688659\\
-28.3991693317234	409.095419822921\\
-27.7866171772594	411.02226836458\\
-27.1617399280238	412.808401716826\\
-26.5275721950489	414.452108418515\\
-25.8882902487174	415.952610200598\\
-25.2489330560805	417.309983359687\\
-24.6152706591187	418.524886798994\\
-23.9935899767506	419.598274183426\\
-23.3893420694438	420.532395539841\\
-22.7610008657885	421.392247066719\\
-22.0835693188882	422.222696598376\\
-21.366162441657	423.017022625089\\
-20.6112783914274	423.773960140023\\
-19.8177318677177	424.494880101137\\
-18.9850871760725	425.178084263078\\
-18.1149225920552	425.81963298127\\
-17.2100768300383	426.415865309355\\
-16.2721555740804	426.965192397012\\
-15.2992066037831	427.468016888766\\
-14.2839893897193	427.926493035951\\
-13.2165046589262	428.342812358742\\
-12.0899878353179	428.716756550651\\
-10.9079439359575	429.044424651752\\
-9.69088184451813	429.318851532882\\
-8.47996661956121	429.532573271698\\
-7.31392749541215	429.683161357648\\
-6.20770989225377	429.773095340717\\
-5.14741229414892	429.805753195128\\
-4.10055307129393	429.782234968093\\
-3.05435842795979	429.701886140543\\
-2.01072100178908	429.563632039034\\
-0.972788379575986	429.366072444701\\
0.0557480485805666	429.108220045852\\
1.07129119515285	428.790494878722\\
2.07097081685627	428.414992903804\\
3.05125008952437	427.983616672049\\
4.00822392564165	427.497385092275\\
4.93797217251875	426.957693652547\\
5.84172767551464	426.363404628974\\
6.76697926225755	425.678512603662\\
7.72268865987445	424.888953535566\\
8.69723022737316	423.997117078136\\
9.67972717341997	423.006413876006\\
10.6634291077538	421.916696278191\\
11.6420805104973	420.7273544968\\
12.6066675362486	419.440858888048\\
13.5527403059534	418.060553639614\\
14.4827065934538	416.594543351484\\
15.3905275341922	415.044742053392\\
16.2701187078061	413.413163846449\\
17.1167051195172	411.70284903088\\
17.9258938845378	409.917108445688\\
18.6948098686619	408.05898292981\\
19.4225361447224	406.131334897633\\
20.1082403971789	404.136283682736\\
20.7498992424801	402.076128215804\\
21.3439902919288	399.953251406264\\
21.886199611095	397.769493593805\\
22.3725752618856	395.52692091905\\
22.8016430677745	393.225967070589\\
23.1768941083186	390.859450659641\\
23.4973229324833	388.428914142315\\
23.7597837033407	385.93731449346\\
23.962010857014	383.387615368995\\
24.1023971642055	380.782822450827\\
24.1802576376053	378.125904092035\\
24.1958962699593	375.419618885117\\
24.1503937383288	372.665956251862\\
24.0452071433136	369.865974852517\\
23.8823280291801	367.020884056082\\
23.6652443105363	364.131405390295\\
23.4001367089597	361.197682064225\\
23.0963037449987	358.219393991272\\
22.7654107184006	355.195759153747\\
22.420722718604	352.12581459038\\
22.0751783218904	349.008873920535\\
21.73887851358	345.844845276061\\
21.4203899184623	342.65791252107\\
21.1340348031155	339.572996328007\\
20.8819280448204	336.608388696315\\
20.6620433313506	333.764355526509\\
20.4716219450159	331.041158962932\\
20.3085736210225	328.439025231239\\
20.1743672266683	325.958302830748\\
20.0786926220775	323.60214447781\\
20.0344011440769	321.373933843733\\
20.0452431527416	319.274063796015\\
20.109972829793	317.30333507728\\
20.2237229178543	315.462727441558\\
20.3816774806037	313.753921195885\\
20.5794371733435	312.178851836505\\
20.8116714574422	310.738819095852\\
21.0800862869906	309.4108314723\\
21.4090388076992	308.10559402058\\
21.7991172529484	306.817103366878\\
22.2480858947572	305.550390017823\\
22.7508954846363	304.310698312694\\
23.3002499291666	303.104318230382\\
23.8876583724499	301.93876231793\\
24.5027651524926	300.824224130199\\
25.1332110883818	299.776426657421\\
25.7607757996645	298.817586337772\\
26.3670836341666	297.962975553082\\
26.9419997609506	297.214625245059\\
27.4724082517676	296.573930979033\\
27.9495499652997	296.034791513359\\
28.3665300747125	295.55060352833\\
28.782953562523	295.025405529006\\
29.2472812701581	294.388214080657\\
29.7620359226734	293.622233034167\\
30.3042758862227	292.730429775225\\
30.8551900403699	291.715527208622\\
31.3991713074407	290.584754937031\\
31.9239675470456	289.347402687536\\
32.4196816044277	288.012095448153\\
32.8780757461663	286.585385092215\\
33.2928222989327	285.072019963109\\
33.65863903872	283.476889931842\\
33.9711301394631	281.805011937851\\
34.2257541528624	280.059542144212\\
34.4184736778211	278.237133707972\\
34.5674584072719	276.336005378402\\
34.6703190880079	274.360019350655\\
34.7201830866015	272.313712222667\\
34.724978020507	270.200569907354\\
34.6942082556283	268.022182596812\\
34.6336302281186	265.779819162474\\
34.5489627769346	263.47463109819\\
34.4459584858991	261.107576854161\\
34.3301135256145	258.679843222553\\
34.2065745727549	256.192694845378\\
34.0799150554597	253.647470900008\\
33.9536693787073	251.045625395702\\
33.8302034771446	248.388709409896\\
33.7110247554329	245.678334308094\\
33.5970942390699	242.915470951293\\
33.4890940647302	240.100629153848\\
33.3875874877925	237.234565739342\\
33.2930245691288	234.317856163262\\
33.2056908069306	231.351250388\\
33.1256460464878	228.335574547525\\
33.0527709382616	225.271666974462\\
32.9868282338783	222.160374937695\\
32.9274788384133	219.002552513793\\
32.8743127397314	215.799058967492\\
32.8268825883766	212.550757642404\\
32.7846781788586	209.258515164458\\
32.7471131791067	205.92320055329\\
32.7135575343747	202.545684310791\\
32.683330302199	199.126751071957\\
32.6556581993741	195.667059351757\\
32.6297123406723	192.167475425456\\
32.6046390176676	188.628875739142\\
32.579512917544	185.052136707619\\
32.5533277315605	181.438134779024\\
32.5250239293392	177.787746340297\\
32.4934865742085	174.101847980937\\
32.4575630172187	170.381316576602\\
32.4160729931475	166.627029510934\\
32.3677915429591	162.839865540789\\
32.3114970797037	159.020703966168\\
32.2460286863892	155.170422757876\\
32.1702930743585	151.289897672929\\
32.0833128619919	147.379958759088\\
31.9843119179451	143.441265571085\\
31.8727567833208	139.474643016172\\
31.748434118964	135.480920715868\\
31.6115612022323	131.46089400518\\
31.4628501980245	127.415316109366\\
31.3035974646744	123.34489032504\\
31.1357810129949	119.250264720931\\
30.962119378905	115.132034149282\\
30.7861648576209	110.990748170521\\
30.6123134288991	106.826932772449\\
30.4461546275849	102.653254781696\\
30.2962705187031	98.5602969529593\\
30.169082435143	94.5894371152109\\
30.069574022849	90.7409215243673\\
30.0022340354668	87.015198189622\\
29.9708619173273	83.4130121207812\\
29.9792842733179	79.9386408579673\\
30.0301710381133	76.5997946161028\\
30.1260641708301	73.3991981436046\\
30.2696606226196	70.3336250331351\\
30.4488562223516	67.3962197031215\\
30.6587720579619	64.5860245547667\\
30.9038961906229	61.9046669900246\\
31.1834591893657	59.3530211970199\\
31.4942221564404	56.9313646609421\\
31.8331047673881	54.6400103028931\\
32.1976871723899	52.4794675301433\\
32.5857701150802	50.4503653329391\\
32.9964947581033	48.5541455615977\\
33.4311809439914	46.7950696033785\\
33.8862373035067	45.1787778016803\\
34.3538132411959	43.7101413974734\\
34.8258586458712	42.3921569808359\\
35.2966441676035	41.2251039009643\\
35.760315010772	40.2085245342856\\
36.2103397082847	39.3413230998619\\
36.6393820157439	38.6217257128965\\
37.0414321997067	38.044709150107\\
37.4841717084421	37.4968209544068\\
38.0209084509621	36.904019681585\\
38.6609128398716	36.2678721074373\\
39.400180401177	35.6071876471133\\
40.2380985038725	34.934426969712\\
41.1722455432313	34.2612852094222\\
42.198782654868	33.5983479356211\\
43.3146893936292	32.9539670211612\\
44.5097542666702	32.3139482967364\\
45.7631363215968	31.645617080021\\
47.0374953758017	30.9531846756533\\
48.2346345456209	30.2837280051236\\
49.3436444643829	29.6423895975214\\
50.352965939284	29.0153161419394\\
51.2584244489559	28.3985469577176\\
52.0598241552976	27.7938526603578\\
52.7564380008451	27.2046565753139\\
53.3478475931798	26.635577459383\\
53.8343087980538	26.0923326163064\\
54.2173614710088	25.5810749726267\\
54.5335920229387	25.0653685620237\\
54.8395620969733	24.4770784986388\\
55.1112796415766	23.8415404878315\\
55.3597803770744	23.1289208475695\\
55.6083495520226	22.3198422987254\\
55.8730892255524	21.4267241304561\\
56.1355698277679	20.4729919108497\\
56.3541463778558	19.5383062089688\\
56.5121103662451	18.6696862117306\\
56.6121452297268	17.8709812533438\\
56.6609715567314	17.1275743096629\\
56.6630342520553	16.4044986989984\\
56.6169282588326	15.6529519120177\\
56.5152389380438	14.8198063679944\\
56.3482957272278	13.9023067940812\\
56.1081782190577	12.91308267657\\
55.7935603831121	11.8613900994665\\
55.4077973826232	10.7559435896951\\
54.9504069132782	9.60686700253449\\
54.4203293198012	8.42162010145255\\
53.8164950582041	7.20519747079664\\
53.138710039533	5.96200519708245\\
52.386811093496	4.69468014956592\\
51.5611654145981	3.40765086076158\\
50.6625970356158	2.10647947860386\\
49.6917154370328	0.795369541151979\\
48.6494784025261	-0.521830867663537\\
47.5369590588209	-1.84141991435771\\
46.3553990930343	-3.15987001550887\\
45.106204124807	-4.47419856379261\\
43.7903964600734	-5.78230992030015\\
42.4091890791146	-7.08241316111936\\
40.9640368961064	-8.37266253767228\\
39.4568927259715	-9.65074650836806\\
37.9067052642462	-10.9013596848951\\
36.3879365456019	-12.0707367733399\\
34.9194068713828	-13.1523504245882\\
33.5051531460343	-14.1480292214111\\
32.1467164054754	-15.0583814270796\\
30.8450089407162	-15.883131091319\\
29.6020371996389	-16.6234197281964\\
28.4200622972808	-17.2807742823882\\
27.3011869918374	-17.8563567965636\\
26.2477865918999	-18.3516842183338\\
25.2625247541456	-18.7687817822378\\
24.3481963881513	-19.1100762525045\\
23.5074918822601	-19.3784103006238\\
22.7418936973382	-19.5774270446101\\
22.0441959885795	-19.7132625627961\\
21.3632647186377	-19.8009668469954\\
20.672813791512	-19.8472534115728\\
19.9511282147453	-19.8534478338539\\
19.1400771455569	-19.8213322157918\\
18.2334310275695	-19.7425988077854\\
17.2559410579955	-19.6111492044685\\
16.3116942543325	-19.4366740920153\\
15.4649654117725	-19.2330561092856\\
14.7186625328994	-19.0076458914571\\
14.0415507425327	-18.757402941637\\
13.3268212102874	-18.4449462780417\\
12.5375810024076	-18.0471496840726\\
11.6933590727165	-17.5618905457092\\
10.8050794028067	-16.9875269203615\\
9.88146132883091	-16.3272003513652\\
8.93174282801823	-15.5841706431571\\
7.9648955767128	-14.7608702773978\\
6.9866084939039	-13.8580292927418\\
6.00136539853772	-12.8764361044279\\
5.01302591796328	-11.8170274809488\\
4.02443028060102	-10.6798192027553\\
3.04114092157657	-9.4677969622676\\
2.06748757141518	-8.18241915851524\\
1.10732617668911	-6.82571272158247\\
0.164374981930582	-5.39977105084543\\
-0.757873710890724	-3.90684997058361\\
-1.65596977545573	-2.349188068467\\
-2.5261438069305	-0.728002118399609\\
-3.36396103959255	0.95600992679205\\
-4.1656173747443	2.70173063469576\\
-4.92937132560237	4.50668985126214\\
nan	nan\\
nan	nan\\
nan	nan\\
nan	nan\\
nan	nan\\
nan	nan\\
nan	nan\\
nan	nan\\
nan	nan\\
nan	nan\\
nan	nan\\
nan	nan\\
nan	nan\\
nan	nan\\
nan	nan\\
nan	nan\\
nan	nan\\
nan	nan\\
nan	nan\\
nan	nan\\
nan	nan\\
nan	nan\\
nan	nan\\
nan	nan\\
nan	nan\\
nan	nan\\
};

\end{axis}

\begin{axis}[%
width=4.625in,
height=10.771in,
at={(0in,0in)},
scale only axis,
xmin=0,
xmax=1,
ymin=0,
ymax=1,
axis line style={draw=none},
ticks=none,
axis x line*=bottom,
axis y line*=left,
legend style={legend cell align=left, align=left, draw=white!15!black}
]
\end{axis}
\end{tikzpicture}%} % inclusion of tex-code
		% \includegraphics[width=\columnwidth]{fig1} % inclusion of pdf
		\caption{Fahrlinie.}
		\label{fig2}
	\end{center}
\end{figure}

\begin{figure}[h]
	\begin{center}
		\scalebox{0.5}{% This file was created by matlab2tikz.
%
%The latest updates can be retrieved from
%  http://www.mathworks.com/matlabcentral/fileexchange/22022-matlab2tikz-matlab2tikz
%where you can also make suggestions and rate matlab2tikz.
%
\definecolor{mycolor1}{rgb}{0.00000,0.44700,0.74100}%
%
\begin{tikzpicture}

\begin{axis}[%
width=4.521in,
height=3.566in,
at={(0.758in,0.481in)},
scale only axis,
unbounded coords=jump,
xmin=0,
xmax=1200,
xlabel style={font=\color{white!15!black}},
xlabel={s[m]},
ymin=0,
ymax=45,
ylabel style={font=\color{white!15!black}},
ylabel={v[m/s]},
axis background/.style={fill=white},
xmajorgrids,
ymajorgrids,
legend style={legend cell align=left, align=left, draw=white!15!black}
]
\addplot [color=mycolor1]
  table[row sep=crcr]{%
0	0\\
0.16104733037824	3.20178174578114\\
0.592433252200396	5.39513322888937\\
1.25000639169794	7.6217296466141\\
2.08914287335502	9.14416872277168\\
3.07443355359124	10.5590529019268\\
4.19593794301175	11.8657862374817\\
5.44242182896933	13.0542431748942\\
6.8013003926219	14.1132920403323\\
8.26005902893635	15.0616552981787\\
9.81175142134889	15.9761141122221\\
11.4532984139585	16.8580771315051\\
13.1813819962147	17.7073683708248\\
14.9928364291136	18.523797036568\\
16.8840927951956	19.3028406613172\\
18.8513579130648	20.0434598110789\\
20.8907592796853	20.745813396668\\
22.9988723972649	21.4210272328193\\
25.173857331165	22.0830533521182\\
27.4143350382193	22.7306900260175\\
29.718866931602	23.3648092678649\\
32.0862227887059	23.9861752823881\\
34.514948477356	24.5920301830852\\
37.0034765902002	25.1820575793564\\
39.5502105111004	25.7559905341152\\
42.1535293821034	26.3136120892494\\
44.8117931078849	26.8547556553329\\
47.5233483018275	27.3797489861893\\
50.2871755367835	27.9013210240998\\
53.1027797924698	28.4152635883984\\
55.9694086747282	28.922485275689\\
58.8864963908787	29.4237742025262\\
61.8533135908525	29.9170735509163\\
64.8690549992785	30.4022677059922\\
67.9329043089296	30.8792516678635\\
71.0440352718347	31.3479311749768\\
74.2016128165139	31.8082229086171\\
77.4047941928491	32.2600545103989\\
80.6527300646216	32.7033644829613\\
83.9445656018727	33.1381023114715\\
87.2794416391743	33.5642284418463\\
90.6564958381505	33.9817141243523\\
94.0749482104394	34.3928476351888\\
97.5341482115893	34.7960828297405\\
101.033236922387	35.1905985633343\\
104.57134507101	35.5763991730063\\
108.1476048693	35.9534978830508\\
111.761150796697	36.3219154893122\\
115.411120209883	36.681679245268\\
119.096653575458	37.0328224236752\\
122.816894428064	37.3753815586217\\
126.570988870401	37.7093940529179\\
130.358084581487	38.034901365683\\
134.177329505645	38.3519442919625\\
138.027869654191	38.6605596968599\\
141.90884702666	38.9607935606348\\
145.819436972906	39.2541595909722\\
149.75896736751	39.5404087924309\\
153.726576036575	39.8183396754415\\
157.721362832681	40.0880480529678\\
161.742421001319	40.3496500779627\\
165.788841835773	40.6033146956604\\
169.859725914511	40.8492745875622\\
173.954194709818	41.0877971301842\\
178.071403524577	41.3191996306998\\
182.210561153052	41.5438621008812\\
186.370274356184	41.6489874654763\\
190.470909269531	40.4446473932039\\
194.448541309141	39.2193199762973\\
198.303288999649	37.9950728854537\\
202.035486523486	36.7716819961947\\
205.645502459448	35.5486057213405\\
209.133676770838	34.3249932418122\\
212.50024707092	33.0997584797739\\
215.745270766066	31.8718019056077\\
218.868567281434	30.6393720289528\\
221.868043166871	29.3594678381366\\
224.737123820255	28.011567847077\\
227.469940641449	26.6338802568737\\
230.065499442359	25.270890949363\\
232.525564008596	23.9499818001638\\
234.85529961968	22.700224307079\\
237.058288102297	21.4527616706782\\
239.132570995535	20.1962267489081\\
241.077758494836	18.9386559190424\\
242.895520619785	17.6864685783023\\
244.584894809556	16.4292197190669\\
246.137521513293	15.1527401047071\\
247.542941085941	13.8300432588992\\
248.8025965763	12.4418496252703\\
249.936125537365	11.0395227498293\\
250.965582241717	9.64131496435136\\
251.895642831545	8.25150756137311\\
252.714007061935	6.8750281295373\\
253.405541687067	5.71263682512892\\
254.071881474101	6.03165266283848\\
254.784910253021	6.20939238757506\\
255.524241411396	6.25968863039307\\
256.277280037608	6.25348733694842\\
257.037740111743	6.23114458743137\\
257.80369398799	6.21009479150147\\
258.574024070767	6.18903714061239\\
259.348518409367	6.17328334711278\\
260.127263810711	6.1614210210126\\
260.910210258309	6.15166007948655\\
261.697069719244	6.14290349266732\\
262.4875584614	6.1367643853865\\
263.281812346889	6.13385397500737\\
264.079820602497	6.13275869097302\\
264.881620568141	6.13181968848647\\
265.687319760285	6.131877281989\\
266.49695724073	6.13164798964924\\
267.309927068246	6.12900601548043\\
268.122646125727	6.07644109999259\\
268.941789500755	6.4274733793935\\
269.828706010229	7.31456651686067\\
270.802494623404	8.6162679358667\\
271.81130844883	9.89012693806099\\
272.805271610491	11.0933751146556\\
273.801831373824	11.9992972603418\\
274.802742122714	10.9615929780199\\
275.803282162069	9.82885620253428\\
276.806128959917	8.66282439140038\\
277.774764360925	7.48474326065264\\
278.657904088104	6.3031344751766\\
279.421724750306	5.50621022409344\\
280.210944383241	6.00156906635791\\
281.066261040248	6.2163734285084\\
281.957791238391	6.27703400180465\\
282.868423425577	6.27201206247562\\
283.788193899961	6.24450978545057\\
284.711347109407	6.21631518895783\\
285.633769366112	6.19287424060464\\
286.552497085339	6.17766334713663\\
287.464086046503	6.16364580196196\\
288.36541379427	6.15632732001277\\
289.254173670672	6.15178190067995\\
290.127918801304	6.14895698401773\\
290.984183984928	6.14704280432219\\
291.820162711289	6.14564017722092\\
292.632625712163	6.14479466508796\\
293.418175208449	6.14331249164107\\
294.174049690323	6.13768375501019\\
294.934874589151	7.13283488994207\\
295.792359719712	8.45038727559083\\
296.729947457626	9.7756310089142\\
297.730819387418	10.9358940525709\\
298.805404977369	12.0004186246773\\
299.974751226526	12.9624156239797\\
301.250235267372	13.8168417814306\\
302.62925808232	14.6681807413937\\
304.111033601312	15.5506191061334\\
305.688030139588	16.4025321990589\\
307.35457777054	17.2236109411847\\
309.107225983394	18.0182838157657\\
310.942520177962	18.7825360116915\\
312.855851249498	19.5161097905655\\
314.842475423061	20.2211497867698\\
316.898138781517	20.8990745069298\\
319.019807358294	21.5614432307185\\
321.206350088493	22.2158222710485\\
323.4569998546	22.8591196707788\\
325.770974843118	23.4908880253568\\
328.147486740435	24.1090349386565\\
330.585272404923	24.7111324213845\\
333.082864443881	25.2967936518279\\
335.638685636892	25.8657836277265\\
338.251086591048	26.4180007270321\\
340.918368619013	26.9533914274587\\
343.638816203366	27.4741557035046\\
346.411516058208	27.9911560090305\\
349.235772995391	28.5004095889702\\
352.110725332887	29.0034357366225\\
355.035695062673	29.5001952363424\\
358.009824331712	29.9892243360692\\
361.032233018532	30.4705157609402\\
364.102062569568	30.9440512065656\\
367.218482628779	31.4098034732314\\
370.380698408051	31.8677314757496\\
373.587955663303	32.3177466412526\\
376.839537854129	32.7596738136114\\
380.134753890318	33.1932238921476\\
383.47291649167	33.6179584831216\\
386.853306959542	34.0332620173156\\
390.275239962769	34.4410191578291\\
393.737968445329	34.838005646163\\
397.240351547802	35.2227907646221\\
400.780997806207	35.5944961800267\\
404.358254119533	35.9525448008827\\
407.970228469223	36.2967468606312\\
411.614847456847	36.6273815168457\\
415.289936812742	36.9451368253293\\
418.990854871674	37.1046599142681\\
422.664002605479	36.2850686567476\\
426.21405575949	34.968950942994\\
429.626437177544	33.6196734298824\\
432.898518078049	32.2691687554078\\
436.031495196497	30.9601971778016\\
439.030379724626	29.697154067597\\
441.895690818735	28.4328870680523\\
444.628094566243	27.1656790448881\\
447.274666788767	25.9043878507424\\
449.918340283509	24.6491836493278\\
452.537731226046	23.3926622330237\\
455.047027360545	22.1308452863485\\
457.40532800177	20.8565023012016\\
459.597865385403	19.5669463730319\\
461.617071494244	18.2656830094488\\
463.461839004038	16.9583610577549\\
465.136527986574	15.6509030648893\\
466.648836028041	14.3469084891095\\
468.007746317277	13.0473412101842\\
469.222150859905	11.7506232457967\\
470.301710299663	10.6432876730577\\
471.330127595384	10.6949013094533\\
472.359992662414	10.7204124300063\\
473.383256632285	10.6905022818663\\
474.40026528506	10.7004405118048\\
475.415577753078	10.7469518655905\\
476.431453023728	10.7944567667679\\
477.447593934252	10.8270238450467\\
478.463048810497	10.8499631998703\\
479.47892334571	10.8989085881422\\
480.500126170495	11.0264038955786\\
481.536661878347	11.279890945808\\
482.600817548685	11.6584179690169\\
483.701228480728	12.0854456005373\\
484.836459592921	12.4229833129322\\
485.989260135581	12.4627609779558\\
487.123773483076	12.0671507888048\\
488.207237659946	11.4227803547939\\
489.229181002588	10.7983506920829\\
490.205585330391	10.4888117068677\\
491.169396515136	10.4762558454388\\
492.135474387918	10.5107987756014\\
493.105321337459	10.5482288208412\\
494.079495161819	10.586296833282\\
495.0582270374	10.6240383547379\\
496.041777427051	10.6614386445035\\
497.030742292981	10.6970279118623\\
498.025006103179	10.7247845469661\\
499.024279381952	10.7451452607634\\
500.028186584008	10.75412850599\\
501.041893182272	11.0643048682957\\
502.125752108436	11.9697509148088\\
503.301015574746	12.8227051849973\\
504.565294382981	13.5941552757336\\
505.916421264996	14.3186413022879\\
507.357577190629	15.0606403401913\\
508.892786122285	15.7492900909924\\
510.521803698197	16.4153927250562\\
512.246584848756	17.0564361795687\\
514.06835834093	17.669042133382\\
515.983734582222	18.2566296292393\\
517.987100107693	18.8178236611887\\
520.070190279049	19.3523233863677\\
522.22181996595	19.862541908161\\
524.429600112139	20.3617828186103\\
526.681007425203	20.8522888094452\\
528.959749639379	21.3439709425732\\
531.231187474063	21.8159939084879\\
533.461726734745	22.2773012511191\\
535.671364602222	22.7284492931895\\
537.913457693952	23.1703931153305\\
540.213776851771	23.6814629539404\\
542.580616669434	24.2446050827018\\
545.011182295409	24.7907862805388\\
547.502682165626	25.3211776202664\\
550.052346174993	25.8370281705867\\
552.657141278148	26.3384939452887\\
555.314062080434	26.8268147091269\\
558.020347098428	27.3041159174651\\
560.774010558232	27.7830811233987\\
563.573994587491	28.2615644612382\\
566.419089563487	28.738926286175\\
569.308573784426	29.2191015742681\\
572.242303888508	29.699741470548\\
575.220599729919	30.1800161907452\\
578.244243028095	30.6579411158029\\
581.314196404863	31.1302081437923\\
584.431145909999	31.5934912242773\\
587.595183153389	32.046384605509\\
590.782124051305	31.5900747654082\\
593.867047563971	30.3612821675807\\
596.831661651957	29.1328914972291\\
599.675700502175	27.9055808609026\\
602.398901854662	26.6793311279513\\
605.001038743592	25.4530584322216\\
607.481763323225	24.2164360989259\\
609.83793176647	22.9259160434982\\
612.066173465271	21.6360034942454\\
614.166062470627	20.3515275806551\\
616.136654851558	19.0737237547867\\
617.976821021036	17.7964622176036\\
619.685103545796	16.5126950778398\\
621.260440920545	15.2244498149355\\
622.702549827104	13.9376986691856\\
624.034390542854	13.4547752903237\\
625.34332248457	13.4664214195394\\
626.631103750796	13.4566605282683\\
627.891543225857	13.4179936418913\\
629.133491716617	13.3284400557089\\
630.394650944854	13.1698431686161\\
631.766274333399	12.9139753946449\\
633.393976044738	12.5143163703421\\
635.139234683996	11.8925399578548\\
636.499754565108	10.9734974607574\\
637.462528667896	9.97373650664766\\
638.214236493879	8.87312961971208\\
638.864072885645	7.75265943877212\\
639.470637581346	6.63349071807396\\
640.074831429658	6.47764521602133\\
640.831453495435	7.23392629925794\\
642.031647432829	8.58827217334075\\
643.890376942235	9.85332124369018\\
645.653747644699	11.0177082187123\\
647.041899813536	12.0724060271191\\
648.303930421566	13.0176027400547\\
649.57223160409	13.8571942381231\\
650.904749477619	14.6268224873617\\
652.325694572878	15.3488324574848\\
653.838066182271	16.0383812682604\\
655.434281680448	16.6961810152411\\
657.106574376533	17.324750698085\\
658.851553288399	17.9721935118629\\
660.673313543477	18.703465019645\\
662.574038402045	19.4380122273209\\
664.549857080384	20.1370087739426\\
666.596119235451	20.8064636008379\\
668.70926814953	21.4634956943539\\
670.887698418386	22.1138917553258\\
673.130142950293	22.7546896253973\\
675.43544629078	23.3853093791091\\
677.80264196651	24.004449130261\\
680.23053500565	24.6090719455259\\
682.717853438297	25.1985156235602\\
685.263251720346	25.7719736870777\\
687.86527092757	26.3289699801577\\
690.522356683373	26.8693274971327\\
693.232895518439	27.3936849621217\\
695.995915230784	27.9144970643381\\
698.810905054305	28.427622738292\\
701.677107379874	28.9340581028902\\
704.593946124277	29.4344880930135\\
707.560670927297	29.9269414355626\\
710.576455572223	30.4113305569556\\
713.640461893861	30.8875664994521\\
716.751842955352	31.3555662688203\\
719.909745154339	31.8152552729867\\
723.113309799301	32.2665604105764\\
726.361674171972	32.7094169458251\\
729.653972359516	33.1437736100655\\
732.989336161668	33.5695858932634\\
736.366895956944	33.9868165082815\\
739.785868061712	34.397778473218\\
743.245595041155	34.8008548126189\\
746.745211777981	35.1952429345879\\
750.283843018735	35.5809351005264\\
753.860613641212	35.9579318534875\\
757.474648602943	36.3262406312051\\
761.12507299695	36.6858766483013\\
764.81101179105	37.0368623271737\\
768.531589719543	37.3792295388568\\
772.285931047298	37.7130153503084\\
776.073158728366	38.0382643863919\\
779.892395168948	38.3550411164567\\
783.74276401974	38.6634246161701\\
787.623391050818	38.9635052948424\\
791.533447579659	39.2568974300137\\
795.472275135658	39.543396949158\\
799.43904955708	39.8219230972315\\
803.432941521036	40.0926739018812\\
807.453155389974	40.3558660166134\\
811.498936945185	40.6117119613687\\
815.569581081014	40.8604128823121\\
819.664436968048	41.1021168317759\\
823.78290595953	41.3368869129848\\
827.924433536546	41.5646550148052\\
832.088487574987	41.7851859310351\\
836.262393457546	41.5639925881039\\
840.355556537172	40.3372757150852\\
844.326590060988	39.1079653216358\\
848.175240816162	37.8752666178557\\
851.901054527634	36.6385079886918\\
855.503280880703	35.3899386326885\\
858.977636999506	34.0718501613381\\
862.316408985721	32.6962991368771\\
865.516868485885	31.3378779476186\\
868.582237521319	30.0443802153907\\
871.519389117324	28.7999564811794\\
874.329295358338	27.5474729971596\\
877.010324655686	26.2911972934686\\
879.561546827257	25.0358113066372\\
881.982621096664	23.7829143170329\\
884.273450119749	22.5309663474696\\
886.433763018858	21.2790423982692\\
888.461444454647	20.0264809247219\\
890.352645613058	18.7619149211021\\
892.112825217875	17.4579811130766\\
893.779367875162	16.1042951588178\\
895.41167518687	14.7015846508847\\
897.021913528445	13.285370285717\\
898.549232916223	11.8717016204044\\
899.928812645007	10.4642365484819\\
901.132119338902	9.06636720545536\\
902.155456026023	7.68078401908718\\
903.00865587775	6.64445545990126\\
903.859099511365	7.50789835892866\\
904.818805715844	8.56343072249106\\
905.892241029692	9.48473082645676\\
907.055949785231	10.3451799762647\\
908.290630085477	11.1461763126992\\
909.568700857756	11.8817516908171\\
910.856282778005	12.5590771975742\\
912.131224951607	13.2262973346318\\
913.402036655584	13.8846969024243\\
914.717362111357	14.524762627325\\
916.136071885059	14.1535308372086\\
917.595850563454	13.2643194088773\\
919.046839075601	12.3456922302137\\
920.428170737335	11.4120900029072\\
921.707829888821	10.4936214615786\\
922.875014588312	9.57830114420636\\
923.927340525247	8.66255787015877\\
924.866674178076	7.74644365634597\\
925.696751010374	6.83423098937643\\
926.422377773688	5.94319933942247\\
927.109347320117	6.48992919482588\\
927.862305140618	6.70514543551157\\
928.647308103302	7.18765572948287\\
929.504227246684	7.95332517996963\\
930.461337195106	8.91806536821765\\
931.497780917738	9.69937208939742\\
932.567396509918	9.89751979735271\\
933.57531700004	9.22862326019071\\
934.483445402513	8.42417739077831\\
935.302517878593	7.69844673887426\\
936.057485047847	7.26587418159111\\
936.789760876454	7.2745981712911\\
937.552507973676	7.88290355244846\\
938.402726116067	8.88970364753207\\
939.347245961339	9.76677875516778\\
940.378165115312	10.5934959540535\\
941.490574578958	11.3597448778692\\
942.679223306271	12.0534649185137\\
943.93812618	12.6833829304414\\
945.264138358627	13.2892947829399\\
946.656643435375	13.8763192485226\\
948.115341271094	14.4464202842289\\
949.641507822669	15.0263769832072\\
951.234214766289	15.5587898568304\\
952.890883189405	16.0702160292674\\
954.609843860275	16.5618514523024\\
956.388993138714	17.0344773566067\\
958.22561571851	17.4880842304374\\
960.116299058281	17.9231974042476\\
962.057150981314	18.3469254703001\\
964.044355012688	18.7652711875713\\
966.073214090206	19.1750013897402\\
968.137809410039	19.5738886399532\\
970.230590912666	19.9436913356103\\
972.320912468355	19.6163968195425\\
974.309135671404	18.7030825817907\\
976.175288533542	17.7635055125144\\
977.919050601338	16.8197455432175\\
979.543379436105	15.8770103323317\\
981.052731298091	14.9343016018894\\
982.451689641169	13.9917303829559\\
983.744823990275	13.0493990816709\\
984.936604889173	12.1073130059387\\
986.031095796973	11.1656322690057\\
987.032011483256	10.2249725211776\\
987.942921929406	9.28707266761213\\
988.767389828433	8.35788593583594\\
989.509722248181	7.47048580936789\\
990.181711097816	6.8471546165466\\
990.836345797668	6.93328539078727\\
991.501964512875	6.90957182783671\\
992.202271752469	7.62868742257437\\
992.996525223409	8.66146064397048\\
993.892063752881	9.53486555628251\\
994.860869888317	9.9853925011193\\
995.79287280982	9.16159429328746\\
996.622184585147	8.24678853297985\\
997.34967831938	7.35100713611005\\
998.011561947987	7.37904216383309\\
998.71898045871	8.34271166305446\\
999.519793803602	9.30378280257925\\
1000.41063032647	10.1734222548562\\
1001.39797540804	10.9818355152899\\
1002.48858888597	11.7238367833419\\
1003.6856388386	12.3896029493782\\
1004.98964593908	13.0119266111636\\
1006.40276719311	13.6172051813959\\
1007.92607795812	14.2031124167632\\
1009.55740283493	14.7954863622647\\
1011.29032377863	15.3448398898892\\
1013.10655027728	15.8729697566598\\
1014.98331484296	16.3797694270049\\
1016.89323886109	16.8647592326037\\
1018.80603474021	17.3282643633412\\
1020.68622673202	17.770179728054\\
1022.49564956686	18.1941687322987\\
1024.2233587614	18.6090676062191\\
1025.91645142001	19.0138179666554\\
1027.64606886326	19.4095375116788\\
1029.45235483051	19.7896373900358\\
nan	nan\\
nan	nan\\
nan	nan\\
nan	nan\\
nan	nan\\
nan	nan\\
nan	nan\\
nan	nan\\
nan	nan\\
nan	nan\\
nan	nan\\
nan	nan\\
nan	nan\\
nan	nan\\
nan	nan\\
nan	nan\\
nan	nan\\
nan	nan\\
nan	nan\\
nan	nan\\
nan	nan\\
nan	nan\\
nan	nan\\
nan	nan\\
nan	nan\\
nan	nan\\
};
\addlegendentry{v}

\end{axis}
\end{tikzpicture}%} % inclusion of tex-code
		% \includegraphics[width=\columnwidth]{fig1} % inclusion of pdf
		\caption{Fahrzeuggeschwindigkeit $v$ über $s$.}
		\label{fig3}
	\end{center}
\end{figure}



%\bibliographystyle{alpha}        % Include this if you use bibtex 
%\bibliography{autosam}           % and a bib file to produce the 
%\bibliography{autosam}
                                 % bibliography (preferred). The
                                 % correct style is generated by
                                 % Elsevier at the time of printing.

%\begin{thebibliography}{3}
%
%\bibitem[(Knuth2005)]{Knuth2005}
%Knuth,~D.~E. (2005).
%\newblock The Art of Computer Programming.
%\newblock Pearson Education.
%
%\end{thebibliography}

%\appendix
\end{document}