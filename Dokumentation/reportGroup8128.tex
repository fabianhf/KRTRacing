\documentclass[pdf]{ifacconf}

\usepackage{amsmath}
\usepackage{natbib}            % you should have natbib.sty
\usepackage{graphicx}          % Include this line if your 
                               % document contains figures,
%\usepackage[dvips]{epsfig}    % or this line, depending on which
                               % you prefer.
                               
                               \usepackage{units}

% for German
% \usepackage{ngerman}           % neue Deutsche Rechtschreibung, Silbentrennung
% \usepackage[latin1]{inputenc}  % Eingabe von Umlaute im Editor
% \usepackage[T1]{fontenc}       % Trennung mit Umlauten

% to include tex code of figure created with fig2texPS - and not precompiled pdf-file, see also file ``plotFigureTest.m''
\usepackage{pstricks, pst-node, pst-plot, pst-circ}
\usepackage{moredefs}

% to include the legend into the caption. The commands are
%\mlLineLegend{red}
%\mlLineLegendDashed{red}
%\mlLineLegendDoted{red}
%\mlLineLegendDashDoted{red}
%\mlPointLegend{red}
\newlength{\mlLegendThickness}
\setlength{\mlLegendThickness}{0.4mm}
\newlength{\mlLegendHeight}
\setlength{\mlLegendHeight}{0.6ex}
\newcommand{\mlLineLegend}[1]{\mbox{\color{#1}
\protect\rule[\mlLegendHeight]{3mm}{\mlLegendThickness}\hspace*{-1mm}
}}
\newcommand{\mlLineLegendDashed}[1]{\mbox{\color{#1}
\protect\rule[\mlLegendHeight]{1.5mm}{\mlLegendThickness}\hspace*{0mm}
\protect\rule[\mlLegendHeight]{1.5mm}{\mlLegendThickness}\hspace*{-1mm}
}}
\newcommand{\mlLineLegendDoted}[1]{\mbox{\color{#1}
\protect\rule[\mlLegendHeight]{0.4mm}{\mlLegendThickness}\hspace*{0mm}
\protect\rule[\mlLegendHeight]{0.4mm}{\mlLegendThickness}\hspace*{0mm}
\protect\rule[\mlLegendHeight]{0.4mm}{\mlLegendThickness}\hspace*{0mm}
\protect\rule[\mlLegendHeight]{0.4mm}{\mlLegendThickness}\hspace*{-1mm}
}}
\newcommand{\mlLineLegendDashDoted}[1]{\mbox{\color{#1}
\protect\rule[\mlLegendHeight]{1.5mm}{\mlLegendThickness}\hspace*{0mm}
\protect\rule[\mlLegendHeight]{0.4mm}{\mlLegendThickness}\hspace*{0mm}
\protect\rule[\mlLegendHeight]{1.5mm}{\mlLegendThickness}\hspace*{0mm}
\protect\rule[\mlLegendHeight]{0.4mm}{\mlLegendThickness}\hspace*{-1mm}
}}
\newcommand{\mlPointLegend}[1]{\mbox{\color{#1}
\protect\rule[\mlLegendHeight]{0.4mm}{\mlLegendThickness}\hspace*{-0.75mm}
}}

\begin{document}

\begin{frontmatter}

\title{Report for the Course ``Projektwettbewerb Konzepte der Regelungstechnik''}

%\thanks[footnoteinfo]{Institute for Systems Theory and Automatic Control, University of Stuttgart, Germany. \textit{http://www.ist.uni-stuttgart.de}}

% include all authors, underline corresponding author
\author{\underline{John C. Doe},} 
\author{John B. Doe (Group 8128)} 
% \author{}

\begin{abstract}                          % Abstract of not more than 250 words.
This is a template for the laboratory course ``Projektwettbewerb Konzepte der Regelungstechnik'' at the Institute for Systems Theory and Automatic Control, University of Stuttgart. Use this document to write your report with \LaTeX\ (cf. \cite{Knuth2005}).
\end{abstract}

\end{frontmatter}

\section{Introduction}
We have designed a state feedback for the single-track model based on inverse kinematics, loop-shaping, feedback linearization, and robot navigation functions.

\section{Main Idea}
Our design procedure was based on the idea that accelerating a vehicle results in shorter lap times than braking a vehicle. For presentational conciseness, we have listed some important parameters in Table 1.

\begin{table}[h]
\caption{Important Parameters.}
\label{table:working plan}
\centering
\begin{tabular}{|c|c|}
\hline
\bfseries Parameter & \bfseries Value \\ \hline \hline
lap time $t_{\text{f}}$ & $\infty$ \\ \hline
control gain $k$ & $0$ \\ \hline
steering angle $\delta$ & $-\left(\operatorname{e}^{\operatorname{i}\pi}+1\right)$ \\ \hline
 \end{tabular}
\end{table}

\section{Result}
We have achieved a lap time of $t_{\text{f}}=\infty$. We have depicted a plot of vehicle velocity $v$ versus an independent curve parameter $\gamma$, with which we have parameterized the racetrack, in Fig. 1. 

\begin{figure}[h] % use \begin{figure*} for two-column figure
\begin{center}
% for details see file ``plotFigureTest.m''
% Autor: Peter Scholz
% Email: contact@peter-scholz.net
% Date:  10-Oct-2011 14:49:19
%
% This file was created by fig2texPS. Note, that the packages
% pstricks, pst-node, pst-plot, pst-circ and moredefs are required.
% A minimal example code could be:
%
% \documentclass{article}
% \usepackage{pstricks, pst-node, pst-plot, pst-circ}
% \usepackage{moredefs}
% \begin{document}
% % Autor: Peter Scholz
% Email: contact@peter-scholz.net
% Date:  10-Oct-2011 14:49:19
%
% This file was created by fig2texPS. Note, that the packages
% pstricks, pst-node, pst-plot, pst-circ and moredefs are required.
% A minimal example code could be:
%
% \documentclass{article}
% \usepackage{pstricks, pst-node, pst-plot, pst-circ}
% \usepackage{moredefs}
% \begin{document}
% % Autor: Peter Scholz
% Email: contact@peter-scholz.net
% Date:  10-Oct-2011 14:49:19
%
% This file was created by fig2texPS. Note, that the packages
% pstricks, pst-node, pst-plot, pst-circ and moredefs are required.
% A minimal example code could be:
%
% \documentclass{article}
% \usepackage{pstricks, pst-node, pst-plot, pst-circ}
% \usepackage{moredefs}
% \begin{document}
% \input{fig1.tex}
% \end{document}
%
% Global Parameters that can be changed:
\providelength{\AxesLineWidth}       \setlength{\AxesLineWidth}{0.5pt}%
\providelength{\GridLineWidth}       \setlength{\GridLineWidth}{0.4pt}%
\providelength{\GridLineDotSep}      \setlength{\GridLineDotSep}{0.4pt}%
\providelength{\MinorGridLineWidth}  \setlength{\MinorGridLineWidth}{0.4pt}%
\providelength{\MinorGridLineDotSep} \setlength{\MinorGridLineDotSep}{0.8pt}%
\providelength{\plotwidth}           \setlength{\plotwidth}{7cm}% width of the axes only
\providelength{\LineWidth}           \setlength{\LineWidth}{0.7pt}%
\providelength{\MarkerSize}          \setlength{\MarkerSize}{4pt}%
\newrgbcolor{GridColor}{0.8 0.8 0.8}%
%
% Begin Figure:-------------------------------------------
\psset{xunit=1.000000\plotwidth,yunit=0.600000\plotwidth}%
\begin{pspicture}(-0.140553,-0.157692)(1.011521,1.042308)%

% Draw bounding box for test aspects: ----
% \psframe(-0.140553,-0.157692)(1.011521,1.042308)
% Total width:  8.064516 cm
% Total height: 5.040000 cm

% Draw Grid: ----
% x-Grid:
\psline[linestyle=dashed,dash=2pt 3pt,dotsep=\GridLineDotSep,linewidth=\GridLineWidth,linecolor=GridColor](0.000000,0.000000)(0.000000,1.000000)
\psline[linestyle=dashed,dash=2pt 3pt,dotsep=\GridLineDotSep,linewidth=\GridLineWidth,linecolor=GridColor](0.250000,0.000000)(0.250000,1.000000)
\psline[linestyle=dashed,dash=2pt 3pt,dotsep=\GridLineDotSep,linewidth=\GridLineWidth,linecolor=GridColor](0.500000,0.000000)(0.500000,1.000000)
\psline[linestyle=dashed,dash=2pt 3pt,dotsep=\GridLineDotSep,linewidth=\GridLineWidth,linecolor=GridColor](0.750000,0.000000)(0.750000,1.000000)
\psline[linestyle=dashed,dash=2pt 3pt,dotsep=\GridLineDotSep,linewidth=\GridLineWidth,linecolor=GridColor](1.000000,0.000000)(1.000000,1.000000)
% y-Grid:
\psline[linestyle=dashed,dash=2pt 3pt,dotsep=\GridLineDotSep,linewidth=\GridLineWidth,linecolor=GridColor](0.000000,0.000000)(1.000000,0.000000)
\psline[linestyle=dashed,dash=2pt 3pt,dotsep=\GridLineDotSep,linewidth=\GridLineWidth,linecolor=GridColor](0.000000,0.250000)(1.000000,0.250000)
\psline[linestyle=dashed,dash=2pt 3pt,dotsep=\GridLineDotSep,linewidth=\GridLineWidth,linecolor=GridColor](0.000000,0.500000)(1.000000,0.500000)
\psline[linestyle=dashed,dash=2pt 3pt,dotsep=\GridLineDotSep,linewidth=\GridLineWidth,linecolor=GridColor](0.000000,0.750000)(1.000000,0.750000)
\psline[linestyle=dashed,dash=2pt 3pt,dotsep=\GridLineDotSep,linewidth=\GridLineWidth,linecolor=GridColor](0.000000,1.000000)(1.000000,1.000000)

% Draw Ticks: ----
% x-Ticks:
\psline[linewidth=\AxesLineWidth,linecolor=GridColor](0.000000,0.000000)(0.000000,0.020000)
\psline[linewidth=\AxesLineWidth,linecolor=GridColor](0.250000,0.000000)(0.250000,0.020000)
\psline[linewidth=\AxesLineWidth,linecolor=GridColor](0.500000,0.000000)(0.500000,0.020000)
\psline[linewidth=\AxesLineWidth,linecolor=GridColor](0.750000,0.000000)(0.750000,0.020000)
\psline[linewidth=\AxesLineWidth,linecolor=GridColor](1.000000,0.000000)(1.000000,0.020000)
% y-Ticks:
\psline[linewidth=\AxesLineWidth,linecolor=GridColor](0.000000,0.000000)(0.012000,0.000000)
\psline[linewidth=\AxesLineWidth,linecolor=GridColor](0.000000,0.250000)(0.012000,0.250000)
\psline[linewidth=\AxesLineWidth,linecolor=GridColor](0.000000,0.500000)(0.012000,0.500000)
\psline[linewidth=\AxesLineWidth,linecolor=GridColor](0.000000,0.750000)(0.012000,0.750000)
\psline[linewidth=\AxesLineWidth,linecolor=GridColor](0.000000,1.000000)(0.012000,1.000000)

{ \footnotesize % FontSizeTickLabels
% Draw x-Labels: ----
\rput[t](0.000000,-0.020000){$0$}
\rput[t](0.250000,-0.020000){$0.25$}
\rput[t](0.500000,-0.020000){$0.5$}
\rput[t](0.750000,-0.020000){$0.75$}
\rput[t](1.000000,-0.020000){$1$}
% Draw y-Labels: ----
\rput[r](-0.012000,0.000000){$0$}
\rput[r](-0.012000,0.250000){$6.9\bar{4}$}
\rput[r](-0.012000,0.500000){$13.\bar{8}$}
\rput[r](-0.012000,0.750000){$20.8\bar{3}$}
\rput[r](-0.012000,1.000000){$27.\bar{7}$}
} % End FontSizeTickLabels

% Draw Axes: ----
\psframe[linewidth=\AxesLineWidth,dimen=middle](0.000000,0.000000)(1.000000,1.000000)

{ \small % FontSizeXYlabel
% x-Label: ----
\rput[b](0.500000,-0.157692){
\begin{tabular}{c}
Curve Parameter $\gamma$\\
\end{tabular}
}

% y-Label: ----
\rput[t]{90}(-0.170553,0.500000){
\begin{tabular}{c}
Vehicle Velocity $v$\\
\end{tabular}
}
} % End FontSizeXYlabel

% New Line DATA: ----
\newrgbcolor{color180.0016}{0  0  0}
\psline[plotstyle=line,linejoin=1,linestyle=solid,linewidth=\LineWidth,linecolor=color180.0016]
(0.000000,0.000000)(0.130000,0.129634)(0.210000,0.208460)(0.270000,0.266731)(0.330000,0.324043)
(0.380000,0.370920)(0.430000,0.416871)(0.480000,0.461779)(0.520000,0.496880)(0.560000,0.531186)
(0.600000,0.564642)(0.640000,0.597195)(0.680000,0.628793)(0.710000,0.651834)(0.740000,0.674288)
(0.770000,0.696135)(0.800000,0.717356)(0.830000,0.737931)(0.860000,0.757843)(0.890000,0.777072)
(0.920000,0.795602)(0.950000,0.813416)(0.980000,0.830497)(1.000000,0.841471)

\end{pspicture}%
% \end{document}
%
% Global Parameters that can be changed:
\providelength{\AxesLineWidth}       \setlength{\AxesLineWidth}{0.5pt}%
\providelength{\GridLineWidth}       \setlength{\GridLineWidth}{0.4pt}%
\providelength{\GridLineDotSep}      \setlength{\GridLineDotSep}{0.4pt}%
\providelength{\MinorGridLineWidth}  \setlength{\MinorGridLineWidth}{0.4pt}%
\providelength{\MinorGridLineDotSep} \setlength{\MinorGridLineDotSep}{0.8pt}%
\providelength{\plotwidth}           \setlength{\plotwidth}{7cm}% width of the axes only
\providelength{\LineWidth}           \setlength{\LineWidth}{0.7pt}%
\providelength{\MarkerSize}          \setlength{\MarkerSize}{4pt}%
\newrgbcolor{GridColor}{0.8 0.8 0.8}%
%
% Begin Figure:-------------------------------------------
\psset{xunit=1.000000\plotwidth,yunit=0.600000\plotwidth}%
\begin{pspicture}(-0.140553,-0.157692)(1.011521,1.042308)%

% Draw bounding box for test aspects: ----
% \psframe(-0.140553,-0.157692)(1.011521,1.042308)
% Total width:  8.064516 cm
% Total height: 5.040000 cm

% Draw Grid: ----
% x-Grid:
\psline[linestyle=dashed,dash=2pt 3pt,dotsep=\GridLineDotSep,linewidth=\GridLineWidth,linecolor=GridColor](0.000000,0.000000)(0.000000,1.000000)
\psline[linestyle=dashed,dash=2pt 3pt,dotsep=\GridLineDotSep,linewidth=\GridLineWidth,linecolor=GridColor](0.250000,0.000000)(0.250000,1.000000)
\psline[linestyle=dashed,dash=2pt 3pt,dotsep=\GridLineDotSep,linewidth=\GridLineWidth,linecolor=GridColor](0.500000,0.000000)(0.500000,1.000000)
\psline[linestyle=dashed,dash=2pt 3pt,dotsep=\GridLineDotSep,linewidth=\GridLineWidth,linecolor=GridColor](0.750000,0.000000)(0.750000,1.000000)
\psline[linestyle=dashed,dash=2pt 3pt,dotsep=\GridLineDotSep,linewidth=\GridLineWidth,linecolor=GridColor](1.000000,0.000000)(1.000000,1.000000)
% y-Grid:
\psline[linestyle=dashed,dash=2pt 3pt,dotsep=\GridLineDotSep,linewidth=\GridLineWidth,linecolor=GridColor](0.000000,0.000000)(1.000000,0.000000)
\psline[linestyle=dashed,dash=2pt 3pt,dotsep=\GridLineDotSep,linewidth=\GridLineWidth,linecolor=GridColor](0.000000,0.250000)(1.000000,0.250000)
\psline[linestyle=dashed,dash=2pt 3pt,dotsep=\GridLineDotSep,linewidth=\GridLineWidth,linecolor=GridColor](0.000000,0.500000)(1.000000,0.500000)
\psline[linestyle=dashed,dash=2pt 3pt,dotsep=\GridLineDotSep,linewidth=\GridLineWidth,linecolor=GridColor](0.000000,0.750000)(1.000000,0.750000)
\psline[linestyle=dashed,dash=2pt 3pt,dotsep=\GridLineDotSep,linewidth=\GridLineWidth,linecolor=GridColor](0.000000,1.000000)(1.000000,1.000000)

% Draw Ticks: ----
% x-Ticks:
\psline[linewidth=\AxesLineWidth,linecolor=GridColor](0.000000,0.000000)(0.000000,0.020000)
\psline[linewidth=\AxesLineWidth,linecolor=GridColor](0.250000,0.000000)(0.250000,0.020000)
\psline[linewidth=\AxesLineWidth,linecolor=GridColor](0.500000,0.000000)(0.500000,0.020000)
\psline[linewidth=\AxesLineWidth,linecolor=GridColor](0.750000,0.000000)(0.750000,0.020000)
\psline[linewidth=\AxesLineWidth,linecolor=GridColor](1.000000,0.000000)(1.000000,0.020000)
% y-Ticks:
\psline[linewidth=\AxesLineWidth,linecolor=GridColor](0.000000,0.000000)(0.012000,0.000000)
\psline[linewidth=\AxesLineWidth,linecolor=GridColor](0.000000,0.250000)(0.012000,0.250000)
\psline[linewidth=\AxesLineWidth,linecolor=GridColor](0.000000,0.500000)(0.012000,0.500000)
\psline[linewidth=\AxesLineWidth,linecolor=GridColor](0.000000,0.750000)(0.012000,0.750000)
\psline[linewidth=\AxesLineWidth,linecolor=GridColor](0.000000,1.000000)(0.012000,1.000000)

{ \footnotesize % FontSizeTickLabels
% Draw x-Labels: ----
\rput[t](0.000000,-0.020000){$0$}
\rput[t](0.250000,-0.020000){$0.25$}
\rput[t](0.500000,-0.020000){$0.5$}
\rput[t](0.750000,-0.020000){$0.75$}
\rput[t](1.000000,-0.020000){$1$}
% Draw y-Labels: ----
\rput[r](-0.012000,0.000000){$0$}
\rput[r](-0.012000,0.250000){$6.9\bar{4}$}
\rput[r](-0.012000,0.500000){$13.\bar{8}$}
\rput[r](-0.012000,0.750000){$20.8\bar{3}$}
\rput[r](-0.012000,1.000000){$27.\bar{7}$}
} % End FontSizeTickLabels

% Draw Axes: ----
\psframe[linewidth=\AxesLineWidth,dimen=middle](0.000000,0.000000)(1.000000,1.000000)

{ \small % FontSizeXYlabel
% x-Label: ----
\rput[b](0.500000,-0.157692){
\begin{tabular}{c}
Curve Parameter $\gamma$\\
\end{tabular}
}

% y-Label: ----
\rput[t]{90}(-0.170553,0.500000){
\begin{tabular}{c}
Vehicle Velocity $v$\\
\end{tabular}
}
} % End FontSizeXYlabel

% New Line DATA: ----
\newrgbcolor{color180.0016}{0  0  0}
\psline[plotstyle=line,linejoin=1,linestyle=solid,linewidth=\LineWidth,linecolor=color180.0016]
(0.000000,0.000000)(0.130000,0.129634)(0.210000,0.208460)(0.270000,0.266731)(0.330000,0.324043)
(0.380000,0.370920)(0.430000,0.416871)(0.480000,0.461779)(0.520000,0.496880)(0.560000,0.531186)
(0.600000,0.564642)(0.640000,0.597195)(0.680000,0.628793)(0.710000,0.651834)(0.740000,0.674288)
(0.770000,0.696135)(0.800000,0.717356)(0.830000,0.737931)(0.860000,0.757843)(0.890000,0.777072)
(0.920000,0.795602)(0.950000,0.813416)(0.980000,0.830497)(1.000000,0.841471)

\end{pspicture}%
% \end{document}
%
% Global Parameters that can be changed:
\providelength{\AxesLineWidth}       \setlength{\AxesLineWidth}{0.5pt}%
\providelength{\GridLineWidth}       \setlength{\GridLineWidth}{0.4pt}%
\providelength{\GridLineDotSep}      \setlength{\GridLineDotSep}{0.4pt}%
\providelength{\MinorGridLineWidth}  \setlength{\MinorGridLineWidth}{0.4pt}%
\providelength{\MinorGridLineDotSep} \setlength{\MinorGridLineDotSep}{0.8pt}%
\providelength{\plotwidth}           \setlength{\plotwidth}{7cm}% width of the axes only
\providelength{\LineWidth}           \setlength{\LineWidth}{0.7pt}%
\providelength{\MarkerSize}          \setlength{\MarkerSize}{4pt}%
\newrgbcolor{GridColor}{0.8 0.8 0.8}%
%
% Begin Figure:-------------------------------------------
\psset{xunit=1.000000\plotwidth,yunit=0.600000\plotwidth}%
\begin{pspicture}(-0.140553,-0.157692)(1.011521,1.042308)%

% Draw bounding box for test aspects: ----
% \psframe(-0.140553,-0.157692)(1.011521,1.042308)
% Total width:  8.064516 cm
% Total height: 5.040000 cm

% Draw Grid: ----
% x-Grid:
\psline[linestyle=dashed,dash=2pt 3pt,dotsep=\GridLineDotSep,linewidth=\GridLineWidth,linecolor=GridColor](0.000000,0.000000)(0.000000,1.000000)
\psline[linestyle=dashed,dash=2pt 3pt,dotsep=\GridLineDotSep,linewidth=\GridLineWidth,linecolor=GridColor](0.250000,0.000000)(0.250000,1.000000)
\psline[linestyle=dashed,dash=2pt 3pt,dotsep=\GridLineDotSep,linewidth=\GridLineWidth,linecolor=GridColor](0.500000,0.000000)(0.500000,1.000000)
\psline[linestyle=dashed,dash=2pt 3pt,dotsep=\GridLineDotSep,linewidth=\GridLineWidth,linecolor=GridColor](0.750000,0.000000)(0.750000,1.000000)
\psline[linestyle=dashed,dash=2pt 3pt,dotsep=\GridLineDotSep,linewidth=\GridLineWidth,linecolor=GridColor](1.000000,0.000000)(1.000000,1.000000)
% y-Grid:
\psline[linestyle=dashed,dash=2pt 3pt,dotsep=\GridLineDotSep,linewidth=\GridLineWidth,linecolor=GridColor](0.000000,0.000000)(1.000000,0.000000)
\psline[linestyle=dashed,dash=2pt 3pt,dotsep=\GridLineDotSep,linewidth=\GridLineWidth,linecolor=GridColor](0.000000,0.250000)(1.000000,0.250000)
\psline[linestyle=dashed,dash=2pt 3pt,dotsep=\GridLineDotSep,linewidth=\GridLineWidth,linecolor=GridColor](0.000000,0.500000)(1.000000,0.500000)
\psline[linestyle=dashed,dash=2pt 3pt,dotsep=\GridLineDotSep,linewidth=\GridLineWidth,linecolor=GridColor](0.000000,0.750000)(1.000000,0.750000)
\psline[linestyle=dashed,dash=2pt 3pt,dotsep=\GridLineDotSep,linewidth=\GridLineWidth,linecolor=GridColor](0.000000,1.000000)(1.000000,1.000000)

% Draw Ticks: ----
% x-Ticks:
\psline[linewidth=\AxesLineWidth,linecolor=GridColor](0.000000,0.000000)(0.000000,0.020000)
\psline[linewidth=\AxesLineWidth,linecolor=GridColor](0.250000,0.000000)(0.250000,0.020000)
\psline[linewidth=\AxesLineWidth,linecolor=GridColor](0.500000,0.000000)(0.500000,0.020000)
\psline[linewidth=\AxesLineWidth,linecolor=GridColor](0.750000,0.000000)(0.750000,0.020000)
\psline[linewidth=\AxesLineWidth,linecolor=GridColor](1.000000,0.000000)(1.000000,0.020000)
% y-Ticks:
\psline[linewidth=\AxesLineWidth,linecolor=GridColor](0.000000,0.000000)(0.012000,0.000000)
\psline[linewidth=\AxesLineWidth,linecolor=GridColor](0.000000,0.250000)(0.012000,0.250000)
\psline[linewidth=\AxesLineWidth,linecolor=GridColor](0.000000,0.500000)(0.012000,0.500000)
\psline[linewidth=\AxesLineWidth,linecolor=GridColor](0.000000,0.750000)(0.012000,0.750000)
\psline[linewidth=\AxesLineWidth,linecolor=GridColor](0.000000,1.000000)(0.012000,1.000000)

{ \footnotesize % FontSizeTickLabels
% Draw x-Labels: ----
\rput[t](0.000000,-0.020000){$0$}
\rput[t](0.250000,-0.020000){$0.25$}
\rput[t](0.500000,-0.020000){$0.5$}
\rput[t](0.750000,-0.020000){$0.75$}
\rput[t](1.000000,-0.020000){$1$}
% Draw y-Labels: ----
\rput[r](-0.012000,0.000000){$0$}
\rput[r](-0.012000,0.250000){$6.9\bar{4}$}
\rput[r](-0.012000,0.500000){$13.\bar{8}$}
\rput[r](-0.012000,0.750000){$20.8\bar{3}$}
\rput[r](-0.012000,1.000000){$27.\bar{7}$}
} % End FontSizeTickLabels

% Draw Axes: ----
\psframe[linewidth=\AxesLineWidth,dimen=middle](0.000000,0.000000)(1.000000,1.000000)

{ \small % FontSizeXYlabel
% x-Label: ----
\rput[b](0.500000,-0.157692){
\begin{tabular}{c}
Curve Parameter $\gamma$\\
\end{tabular}
}

% y-Label: ----
\rput[t]{90}(-0.170553,0.500000){
\begin{tabular}{c}
Vehicle Velocity $v$\\
\end{tabular}
}
} % End FontSizeXYlabel

% New Line DATA: ----
\newrgbcolor{color180.0016}{0  0  0}
\psline[plotstyle=line,linejoin=1,linestyle=solid,linewidth=\LineWidth,linecolor=color180.0016]
(0.000000,0.000000)(0.130000,0.129634)(0.210000,0.208460)(0.270000,0.266731)(0.330000,0.324043)
(0.380000,0.370920)(0.430000,0.416871)(0.480000,0.461779)(0.520000,0.496880)(0.560000,0.531186)
(0.600000,0.564642)(0.640000,0.597195)(0.680000,0.628793)(0.710000,0.651834)(0.740000,0.674288)
(0.770000,0.696135)(0.800000,0.717356)(0.830000,0.737931)(0.860000,0.757843)(0.890000,0.777072)
(0.920000,0.795602)(0.950000,0.813416)(0.980000,0.830497)(1.000000,0.841471)

\end{pspicture}% % inclusion of tex-code
% \includegraphics[width=\columnwidth]{fig1} % inclusion of pdf
\caption{Plot of Vehicle Velocity $v$ versus Curve Parameter $\gamma$ ({\bf\color{black}---}).}
\label{fig1}
\end{center}
\end{figure}



%\bibliographystyle{alpha}        % Include this if you use bibtex 
%\bibliography{autosam}           % and a bib file to produce the 
%\bibliography{autosam}
                                 % bibliography (preferred). The
                                 % correct style is generated by
                                 % Elsevier at the time of printing.

\begin{thebibliography}{3}

\bibitem[(Knuth2005)]{Knuth2005}
Knuth,~D.~E. (2005).
\newblock The Art of Computer Programming.
\newblock Pearson Education.

\end{thebibliography}

%\appendix
\end{document}